%%% fs-run-time-conclu   Conclusions

\label {fs-conclusion-tection}

In this paper we introduced the model and corresponding implementation of distributed scalable stream processing system. Our contribution consists of the following key ideas:

\begin{itemize}
    \item Computational model is stateless from the business-logic point of view. This property allows user to not care about direct state management. Nevertheless,
 the model is MapReduce complete. MapReduce was adopted to solve a wide range of analytical and algorithmic tasks [refs??], therefore our system can be applied to them as well.
    \item Partitioning of operations is defined by business-logic. Such approach is motivated by the fact that typically user has deeper knowledge of underlaying data distribution.
    \item The collision management is implemented in optimistic manner. We suppose that failures are unavoidable, but, at the same time, quite infrequent events. Hence, unlike prevention-based collision management, optimistic approach does not impose overhead in the case of nothing is failed.
    \item The computation process is deterministic up to the time assignment at fronts.     
\end{itemize}

Additionally, we introduced a new way to determine a moment when data items can be released from stream without loosing exactly-once guarantee, which we called adaptive micro-batching.

Eventually, we showed that our system outperforms Apache Flink on the real-life computations.









