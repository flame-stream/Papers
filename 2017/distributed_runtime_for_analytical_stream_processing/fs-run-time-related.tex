%%%% fs-run-time-related  Related Work

\label {fs-related-section}

{\bf Data flow:}
One specific detail of our computational model is cyclic data flow graph support. Naiad~\cite{Murray:2013:NTD:2517349.2522738} by Microsoft Research provides an implementation of this idea. Nevertheless, Naiad applies cycles only for iterative computations and allows for each operation to have its own state. 

Another similar concept of Naiad is the usage of logical timestamps to monitor progress. However, to propagate the latest timestamp the pessimistic approach of notifications broadcasting is defined. Therefore, with the assumption of infrequent out-of-order items, our optimistic behavior is more relevant.

In our model, map and group operations are used as core processing primitives. Google Dataflow~\cite{Akidau:2015:DMP:2824032.2824076} provides the same idea. The primary distinction is that Dataflow model does not suppose any kind of operation's state. Additionally, this model provides different window types for grouping. FlameStream grouping is aligned with fixed-sized sliding window, but it is possible to implement other kind of windows by using cycle and grouping with window-affiliation hash.

{\bf State:}
According to the state management, it should be noted that the common approach is to give user the ability to handle state of almost any supported operations. Such behavior is implemented, for isntance, in Apache Flink~\cite{carbone2015apache}, Storm~\cite{apache:storm}, Samza~\cite{Noghabi:2017:SSS:3137765.3137770}, Naiad~\cite{Murray:2013:NTD:2517349.2522738}.
To the best of our knowledge, FlameStream is the only open-source stream processing system that:
\begin{itemize}
    \item Stateless in terms of business-logic
    \item Supports any MapReduce transformations 
\end{itemize}

{\bf Tracking mechanisms within stream:}
One important task that FlameStream faces is handling of the minimal global time. In Apache Storm~\cite{apache:storm} acker is used to eliminate item traces. Unlike Strorm, we use acker to track the least global time of in-flight items.
