\documentclass{llncs}
% packages should be added as needed 
\usepackage{graphicx}

\usepackage{algorithm} % for algorithms

\usepackage{algpseudocode}
\algblockdefx[Process]{Process}{EndProcess}[2][Unknown]{{\bf Process} {\it #2}}{}
\algblockdefx[Event]{Event}{EndEvent}[2][Unknown]{{\bf upon event #2 do}}{}

\usepackage{booktabs} % For formal tables
\usepackage{cite} %for multiple refs

\usepackage{amssymb} %for nice emptyset

\usepackage[T1]{fontenc}
\usepackage{inconsolata}

\usepackage{color}

\definecolor{pblue}{rgb}{0.13,0.13,1}
\definecolor{pgreen}{rgb}{0,0.5,0}
\definecolor{pred}{rgb}{0.9,0,0}
\definecolor{pgrey}{rgb}{0.46,0.45,0.48}

\usepackage{listings}
\lstset{language=Java,
  showspaces=false,
  showtabs=false,
  breaklines=true,
  showstringspaces=false,
  breakatwhitespace=true,
  commentstyle=\color{pgreen},
  keywordstyle=\color{pblue},
  stringstyle=\color{pred},
  basicstyle=\ttfamily,
  moredelim=[il][\textcolor{pgrey}]{$$},
  moredelim=[is][\textcolor{pgrey}]{\%\%}{\%\%}
}

%\usepackage{amsthm} % For claims
%\theoremstyle{remark}

%\settopmatter{printacmref=false, printccs=false, printfolios=false}
\pagestyle{plain} % removes running headers

\newcommand{\PicScale}{0.5}
\newcommand {\FlameStream} {FlameStream}
\begin{document}

\title {\FlameStream: Model and Runtime for Distributed Analytical Stream Processing}
\author{Igor E. Kuralenok \and Artem Trofimov \and Nikita Marshalkin \and Boris Novikov}
\institute{JetBrains Research, St. Petersburg, Russia}

\maketitle
\email{\string{ikuralenok, trofimov9artem, marnikitta\string}@gmail.com, borisnov@acm.org}

\begin{abstract}
A user of current distributed stream processing systems has to choose between non-deterministic computation and high latency due to a need in excessive buffering. 
%In addition, a sophisticated state management must be implemented at the business logic level to provide fault tolerance. 
We present a distributed stream processing engine based on a deterministic computational model and providing low latency. 
Experiments show that our prototype can outperform existing solutions due to low overhead of optimistic synchronization.


%In recent years, there has been a growth in research and industrial solutions in the field of distributed stream processing. 
%However, even state-of-the-art stream processing systems are inconvenient to work with in some ways. 
%Firstly, existing solutions suppose that the state of operations should be managed directly by a user, and that can be confusingly for inexperienced users of analytical processing systems. 
%Secondly, these systems do not provide for deterministic processing by default. 
%Furthermore, the most common approach for this feature is buffering. 
%Its main problem is the extra cost for blocking before each order-sensitive operation. 
%The goal of this paper is to propose a model, which addresses both recognized issues. 
%On the one hand, our model stateless from the business logic perspective. 
%Such behavior is obtained by the technique that allows a state to be a part of a stream. 
%On the other hand, it is deterministic by design. 
%We introduce an optimistic method that avoids buffering before each operation to achieve determinism with low overhead. 
%The experiments show that prototype of our system requires a low extra cost for supporting proposed features and is able to outperform alternative industrial solutions in certain conditions.
\end {abstract}

\section {Introduction}
%%% fs-run-time-intro - Introduction

\label {fs-intro-seciton}

Currently, many  
% real-life 
applications use stream processing for network monitoring, financial analytics, training machine learning models, etc. 
State-of-the-art industrial stream processing systems, such as Flink \cite{carbone2015apache}, Samza \cite{Noghabi:2017:SSS:3137765.3137770}, Storm \cite{apache:storm}, are able to provide low-latency and high-throughput in distributed environment for this kind of problems. 
However, unlike batch and micro-batch processing, stream processing is inherently non-deterministic~\cite{Zaharia:2012:DSE:2342763.2342773}. 
In particular, there is no guarantee that the messages will be processed in the same order and the system produces the same result between any two runs, even if messages are fed to the system with the same monotonically increasing timestamps. 
Although such behavior is observed in most state-of-the-art stream processing systems, it has several 
%significant 
pitfalls:

\begin{itemize}
 \item 
It is natural for the user of a software system to expect  that two independent runs within the same input data produce exactly the same result. The fact that this contract can be violated is able to cause misleadings and complicates the usage of the system.

    \item 
    The lack of determinism leads to the loss of reproducibility of the results, that in turn makes
    % the processes of 
    testing and verification excessively complicated~\cite{Zacheilas:2017:MDS:3093742.3093921}.
    \item 
    The ability to reproduce predictable results is extremely useful for providing consistency guarantees~\cite{Stonebraker:2005:RRS:1107499.1107504}. 
    The absence of this property forces the usage of heavy transactional protocols to achieve exactly-once semantics~\cite{Carbone:2017:SMA:3137765.3137777, jacques2016consistent}.
\end{itemize}

In this work, we introduce \FlameStream\ - stream processing model that is deterministic by design. This property is achieved using strong ordering. The typical way to perform in-order processing is to set up a special buffer in front of each order-sensitive operation~\cite{Li:2008:OPN:1453856.1453890}. However, extra buffering can lead to latency growth~\cite{Zacheilas:2017:MDS:3093742.3093921}, especially if the processing pipeline contains several operations that require ordered input. To avoid this issue, we introduce an optimistic approach for handling out-of-order items that requires single buffer per computational pipeline. Our approach is based on the idea that state can be streamed as an ordinary element. Such approach allows us to generalize speculative computations on the arbitrary MapReduce task. Additionally, it makes the model stateless from the business logic point of view. Our evaluation demonstrates that our method has low overhead and can outperform alternative industrial solution under normal load conditions.

Therefore, the contributions of this paper are the following:

\begin {itemize}
\item Definition of a stateful computational model that does not require state handling from the user
\item The optimistic schema for deterministic processing and the demonstration of its performance competitiveness
\end {itemize}

The rest of the paper is structured as follows: in section~\ref{fs-model} we introduce the proposed model and the optimistic approach for handling out-of-order items, the implementation details of the prototype are discussed in~\ref{fs-impl} and its performance is demonstrated in~\ref{fs-experiments-section}, the relevant prior research is mentioned in~\ref{fs-related-section}, finally we discuss the results and our plans in~\ref{fs-conclusions}.

\endinput


\section {Motivation}
%%%% fs-run-time-motivation  FlameStream motivation

\subsection{State management}

A user of a typical modern stream processing system deals with the state in the following way:

\begin{lstlisting}
public class StatefulTask {
  private final State state = new StateImpl();

  public Aggregation process(KeyValueItem input) {
    final Aggregation aggregation = state.get(input.getKey());
    aggregation.update(input.getValue());
    state.update(input.getKey(), aggregation);
    return aggregation;
  }
}
\end{lstlisting}

Regarding the code above, each input item of the stateful operation triggers the update of some aggregation that is then sent out down the stream. Although this code looks very simple, there are several pitfalls here. Usually, streaming systems provide a variety of implementations for the state functionality. For instance, Flink has 6 basic implementations~\cite{apache:flink:state}, and more than 30 internal. Storm and Samza provide the only key-value model of the state~\cite{apache:storm:state, samza:state}, but it is assumed that user should implement its own model if they need. This is explained by the fact that the knowledge of the internal structure of the state can improve the performance of the overall state management. However, the choice among such diversity of state models can be misleading and require careful investigation. Moreover, if a user wants to use state backend, that is not supported by the system, they must implement logic for all state models. This process requires the implementation of many internal interfaces that can be time demanding. Moreover, complex contracts also can cause the huge difference in logic between state implementations for distinct backends. For instance, Flink's state implementation for RocksDB writes each update to disk, while state adapter for file system uses memory caching. Therefore, the choice of the most appropriate state model within the specified state backend is a complex problem that influences the performance of data processing.

Concerning this issue, we have a purpose to implement stream processing model that avoids direct state handling. We suppose that the way of working with stateful operations can be the following:

\begin{lstlisting}
public class StatefulTask {
  public Aggregation process(Aggregation aggregation, KeyValueItem input) {
    aggregation.update(input.getValue());
    return aggregation;
  }
}
\end{lstlisting}

In this case, a user does not have to work with state manually, all internal work and optimizations are done by the system. Such approach makes the system more convenient for users and can help them to avoid sophisticated bugs. Unfortunately, existing solutions cannot be easily switched to this model, because their architectures were designed for supporting the first approach of state management. Therefore, we introduce the model that naturally maintains the proposed approach, that is detailed in further sections.

\subsection{Deterministic computations}



\section {Computational model}
%%%% fs-run-time-data-flow  FlameStream data flow

The key concept of \FlameStream\ model is a data stream. It is a sequence of discrete events described by data items, internally represented as 
$(Payload, Meta).$
The $Payload$ is processed by a user-defined code, while $Meta$ is handled with \FlameStream\ engine. In particular,
the primary purpose of the meta-information is to impose the total order on data items. 
The  $Meta$ is assigned at the entry (called {\em  front}) and is discarded at the {\em barrier} just before the exit. 

% \subsection{Computational flow}
The stream processing is specified by a logical execution graph. 
Each node of the graph represents a single operation on data items, and edges describe the routing of data items between operations.  
Our model allows cycles in the graph while such graphs are commonly assumed to be acyclic (DAGs) 
~\cite{Zaharia:2016:ASU:3013530.2934664, Carbone:2017:SMA:3137765.3137777}. 
The cycles are required for specification of certain computations (e.g. MapReduce-based) with our set of operation types (outlined below).
%Moreover, as we show further, there are cases when cycles are required, e.g., for MapReduce-based algorithms. 
Figure~\ref{logical-graph-figure} shows the example of logical execution graph.

% \subsection{Physical deployment and partitioning}
A distributed hardware environment is modeled as a set of {\em worker} processes. 
Each worker executes logical execution graph and has an assigned range of hash values used for physical routing of data items to workers. 
%An integer interval (hash range) is assigned to every worker. Intervals are not intersected and cover the range of 32-bit signed integer.
%
Each operation entry has a user-provided hash function called {\it balancing function}. This function is applied to the payload of data items and determines partitioning before each operation. After that, the data items are sent to the worker, which is responsible for the associated hash range. Therefore, load balancing explicitly depends on the user-defined balancing functions. 
%This allows the developer to determine optimal balancing which requires the knowledge of the payload distribution. The system optimizes the hash ranges assignment according to the processing statistics. 

\begin{figure}[ht]
  \centering
  \begin{minipage}[b]{.5\textwidth}
    \centering
    \includegraphics[width=0.9\linewidth]{pics/logical-graph}
    \caption{A logical execution  graph}
    \label{logical-graph-figure}
  \end{minipage}%
  \begin{minipage}[b]{.5\textwidth}
    \centering
    \includegraphics[width=\linewidth]{pics/ordering}
    \caption{The ordering model}
    \label{ordering}
  \end{minipage}
\end{figure}

% мм\subsection{Ordering model}

Data items are totally ordered according to labeles assigned to events at the entry as a part of meta-information. All operations preserve this order. Any additional items produced by an operation are placed between the item being processed and the next item. The ordering labels are dropped when items are delivered from the barrier. 

% We assume that there is a total order on data items. 
% Ordering is preserved when an item is going through the operations. 
% More precisely, the order of output items is the same as the order of corresponding input items. 
% If more than one item is generated, they are inserted in output stream sequentially. 
% Moreover, the output follows corresponding input but precedes the next item. 
% Without diving into details, it should be noted that the order of items is maintained across different fronts.

The ordering is illustrated  in Figure~\ref{ordering}. Data item with payload $1'$ is the derivative of the item with payload $1$, according to operation $F$. The same is for items with payloads $2'$ and $2$. After merge operation, the order between $1$ and $2$ is preserved. Furthermore, $1'$ follows $1$, and $2'$ follows $2$.  

%  We assume that input items of the operations are strictly ordered.

%  \subsection{Operations}

The list of available operations includes:
\begin {description}
\item [Map] applies a user-defined function to the payload of an input item and returns a (possibly empty) sequence of data items with transformed payloads. 

\item [Broadcast] replicates an input item to the specified number of operations.

\item [Merge] operation is initialized with the specified number of input nodes. It sends all incoming data to the output.

\item [Grouping] constructs a single item containing a set of consecutive items that have the same value of partition function. The maximum number of items that can be grouped is specified as a parameter  $Window Size.$ 

The output item of the grouping has the same ordering label as the last item in the output group. 
Groupings of different partitions are independent. 
Grouping is the only operation that has a state.

\end {description}

%has a {\it window size} parameter. Grouping stores input items into distinct buckets by the value of the input balancing function applied to the payload. When the next item arrives at the grouping, it is appended to the corresponding bucket. Each time the grouping outputs window-sized {\it tuple item}, which consists of the most recent (in terms of the meta-information) items of this bucket. If the size of the bucket is less than the window, all items of the bucket are taken. 

The following example illustrates  the grouping operation. 
Let the input stream be a series of integers: $ 1,2,3, \ldots$, and the  partition function returns for even numbers and 0 otherwise. If the window is set to 3, the output is 
$$(1), (2), (1|3), (2|4), (1|3|5), (2|4|6), (3|5|7), (4|6|8), \ldots$$

%The grouping operation has two important properties: the output tuple is identified by its last element, the results among items with different values of a partition function are independent.

% \subsection{User-defined parameters}

% A user can set up the following parameters:

% \begin{enumerate}
%  \item{Computational flow}
%  \item{Balancing functions of the inputs}
%  \item{Map functions}
%  \item{Grouping windows}
%\end{enumerate}

%These parameters can produce more than one graph, which can yield equivalent results. Choosing among them is a performance optimization problem that relies on the system.
%It is important to mention that there are no parameters for state-management. 
%Therefore, business-logic is stateless. Nevertheless, the operations set is enough to implement any MapReduce transformation as shown in the next section.

An important special case of grouping with $Window Size = 2$  provides for realization of stateful calculations with drifting state techniques manifested in section~\ref{motivation-section}.  
Indeed, consider a map operation that follows the grouping and sends its output to the grouping input. This map operation receives a pair of its previous output considered as the state object, and new incoming item from the source stream. The map operation calculates new state object and sends it back as the grouping input. 
More details are provided in the appendix~\ref{fs-drifting}.

% \subsection{Drifting state}

% Most of the computational pipelines require state carrying between consecutive data items. For example, moving average calculation needs partial sums to be stored. The state management in distributed systems is a complicated topic that requires delicate treatment. Some systems forces that all state management is put in business-logic: making state persistent, replication, ensuring state consistency, etc. Others shifts some difficulties to the system's concern, e.g., Flink provides a state API that guarantees consistency in case of failures \cite{Carbone:2017:SMA:3137765.3137777}.

%The core idea of {\it drifting state} is to fully integrate state management into the system internals by making a state a part of the heterogeneous data flow. 
% In order to do state updates, the drifting state is grouped with new elements, combined, and the new state is emitted. 
% WRONG!!!  The limited number of such combines can be implemented with DAG by repetition of grouping and map operations. 
%As stream processing is aimed to infinite sequences, there is a need for a cycle in the computational pipeline. 
%The exact structure of such cycle is described in appendix~\ref{fs-drifting} with an example of typical stateful transformation, MapReduce.

%The complex graph with multiple stateful operators can be constructed using a cycle for each stateful component. 
%It is important to say that such tangled structures may not be exposed to the end-user and can be hidden by some high-level API.



% \section{Drifting state}
% %% It is just an empty TeX file.
%% Write your code here.

\label{fs-drifting}

%\subsection{MapReduce transformation}
The map stage of MapReduce can be expressed in terms of our map operation. 
The generic reduce stage can be presented as

\begin {tabbing}
1234\=1234\= \kill
{\bf for} $mapped \in values$ {\bf do}   \\
\>$accumulator$ := combine ($mapped$, $accumulator$); \\
{\bf end for} \\
{\bf return } $accumulator$;
\end {tabbing}

The {\it accumulator} is an explicit state that should be kept between subsequent iterations.
%
%\begin{algorithm}
%\caption{Generic reduce stage}
%\label{reduce}
%\begin{algorithmic}
%  \Function{reduce}{$key$, $values$}
%    \State $accumulator$ \Comment{reduce's state}
%    \ForAll{$v \in values$} 
%      \State \Call{combine}{$v$, $accumulator$}
%    \EndFor
%    \State \Return \Call{map}{$accumulator$}
%  \EndFunction
%\%\end{algorithmic}
%\end{algorithm}
%
To implement reduce stage we apply the drifting state technique and make the accumulator value a part of the stream. 
Figure~\ref{mapreduce-graph-figure} shows a generic graph for MapReduce transformation. 
Map and reduce stages are highlighted with a dashed line. 

\begin{figure}[ht]
  \centering
  \includegraphics[width=0.6\textwidth]{pics/mapreduce}
  \caption{Logical graph for MapReduce transformations}
  \label {mapreduce-graph-figure}
\end{figure}

There are four types of data items in this stream: {\it input,} {\it mapped,}   {\it accumulator,} and {\it  reduced.} 
% Mapped, accumulator and reduced items have the key-value structure of a payload. 
The operations of the stream have the following purposes:

\begin{itemize}
  \item The first map operation outputs mapped items according to map stage of MapReduce model.
  
  \item The grouping with $WindowSize=2$ groups the $accumulator$  with next $mapped$  item. 
  % The hash function is designed to return distinct values for payloads with distinct keys
  
  \item The combine map  produces new state of $accumulator$ to be sent  to grouping.
  
  %The second map implements the actual combining. It accepts inputs that have a form of: \textit{(mapped item)} or \textit{(accumulator item, mapped item)}. The first kind is transformed into some initial value. 
%  The second one is combined into the new accumulator item as specified by reduce stage of MapReduce. 
 % The tuples with structure \textit{(mapped item, accumulator item)} are filtered out
  \item The final map converts $accumulator$ into final reduce output.
\end{itemize}

Ordering rules  guarantee that each $accumulator$  item always arrives at the grouping right before next not yet combined mapped item.
%Hence, each mapped item that has not been combined yet would be grouped with the right accumulator item. 
%Additionally, when combine map accepts tuple {\it (mapped item, accumulator item)}, 
%then it means that mapped item was generated before accumulator item, and therefore, it had been already combined. 
The cycle gives the ability for new accumulator items to get back in the grouping operation. 
%The accumulator map transforms the accumulator item into the final reduced item right before sink. 
Thereby, the stream reacts to each input item by generating new reduced item, which contains the actual value of the reduce stage.

% \subsection{Example: word count}

% We illustrate the MapReduce algorithm with an example of word counting. Map stage of this algorithm transforms each input word into key-value pair where the word is a key, and the value is 1. Reduce stage sums all values into the final result for the specific key. In this case, the accumulator map is omitted, because the accumulator is the actual result of the reduce stage.

% The example of input/output items, which are generated/ transformed by the part of the logical graph, is shown in Figure~\ref {word-count-figure}. According to our graph for MapReduce transformations, the item {\it m[dog, 1]} represents mapped item with key "dog" and value 1. The item {\it a[dog, 1]} describes accumulator item with key "dog" and value 1. Figure shows how the model reacts on two consequent input items containing word "dog". The meta-information of items is omitted for simplification. More precisely, there are four stages separated by dotted lines:

% \begin{enumerate}
%     \item New mapped item with key "dog" arrives at grouping with an empty state. Grouping outputs tuple with this single item. Combine map transforms it into the first accumulator item for key "dog" and value 1.
%     \item The accumulator item arrives at grouping after it went through the cycle. It is grouped in the tuple with the mapped item that has been already in the state with key "dog". However, combine map drops this tuple, because of the order of items.
%     \item New mapped item with key "dog" arrives at grouping. It is inserted right after the accumulator in the bucket for key "dog". The grouping operation outputs tuple containing the accumulator item and new mapped item. Map operation combines reduced and mapped items into new reduced items with key "dog" and value 2.
%     \item New accumulator item arrives at grouping through the cycle, but new generated tuple is not accepted by combine map, as well as in step 2.
% \end{enumerate}

% \begin{figure}[ht]
%   \centering
%   \includegraphics[scale=0.5]{pics/wordcount}
%   \caption{The example of input/output items, which are generated/ transformed by the part of the logical graph for word counting}
%   \label {word-count-figure}
% \end{figure}


\section {Deterministic computations}
%%%% fs-run-mapreduce Optimistic collision management
\label {fs-collision}

Deterministc execution is a desired property of any kind of system. If there are multiple, often exponential, possible outcomes the developer needs to reason about which results are valid, which are equivalent and which are considered to be invalid.

To reduce outputs to only one possible result we impose two restrictions on our model: 

\begin{itemize}
  \item We require {\it map} function to be pure: return value is only determined by its input values, without observable side effects~\ref{wiki}
  \item We impose a strict ordering requirements on the {\it grouping}'s input
\end{itemize}

While the first requiremet can be easily satisfied by moderating the provided business-logic, the second one is foreign to the distributed systems. Because of asynchrony and the possible existence of multiple paths between two nodes it is hard to deliver the right order. 

In this section we review common approaches for order enforcing, then we introduce our approach.

\subsection{Existing solutions}

There are two most common methods that are used to implement order-sensitive operators: in-order processing (IOP) \cite{Arasu:2006:CCQ:1146461.1146463, Cranor:2003:GSD:872757.872838, hammad2004optimizing} and out-of-order processing (OOP) \cite{Li:2008:OPN:1453856.1453890}.

According to IOP approach, each operation must enforce the total order on output elements that can be violated due to asynchronous nature of execution. Buffering is usually used to fulfill this requirement. For example, the implementation of the merge operator, in presence of a skew between input streams, must buffer the earlier one to enforce order on the output.

OOP is an approach that does not require order maintenance if it is not needed. In the case of ordering requirements, OOP buffers input items until a special condition is satisfied. This condition is supported by progress indicators such as punctuations \cite{Tucker:2003:EPS:776752.776780}, low watermarks \cite{Akidau:2013:MFS:2536222.2536229}, or heartbeats \cite{Srivastava:2004:FTM:1055558.1055596}. They go through the stream as ordinal items, but do not trigger business-logic of the operations. Each progress indicator carries meta-information and promises that there are no any elements with lesser meta-information. Therefore, indicators must be monotonic, but data items between two consecutive indicators can be arbitrarily reordered. Data sources periodically yield them.

While this methods are commonly adapted in practice we find them to have unpredictible latencies and to have bursty loads. In our system we employ an optimistic approach for handling out-of-order items.

\subsection{Optimistic approach}

In order to implement ordering model, we accept the fact that grouping can produce incorrect tuples. However, we guarantee that all correct tuples are eventually produced. The correctness of tuple means that this tuple would be generated if the order assumption was satisfied. 

To eventually produce all correct tuples, we use an approach called {\it replay}. If an item arrives the grouping operation, according to the meta-information order, nothing is replayed and only the most recent window is produced. However, if an item is out-of-order, it is inserted in the bucket at the correct location, and all tuples, which contain this element, are reproduced. Thereby, replay guarantees that eventually all correct tuples are generated. At the same time, for tuples, that has been produced but became invalid, {\it tombstones} are sent.

Tombstones are ordinal data items but with a special flag in its meta-information. This flag means that tuples with such meta-information are invalid, and they should not leave the system. Tombstones have the same payloads as invalid items in order to go along the same path in the computational pipeline.

The example of replay is shown in Figure~\ref{grouping-replaying}, The green item is out-of-order. The output consists of the new valid items {\it (1, 2) and (2, 3)}, and the tombstone {\it (1, 3)} for the previously generated item.

\begin{figure}[htbp]
  \centering
  \includegraphics[width=0.48\textwidth]{pics/grouping-replaying}
  \caption{The replay in groupig with window = 2. The new items are generated on insertion}
  \label {grouping-replaying}
\end{figure}

In the case of the right order of input items, there are no redundant items produced.

The barrier filters out invalid elements, when corresponding tombstones arrive. It is partially flushed for some meta-information interval when there is a guarantee that there are no any out-of-order items and tombstones further up the stream for this range. The exact technique for providing such guarantee is defined in Section~\ref{fs-impl}.

\subsection{Advantages and limitations}

The proposed architecture's performance depends on how often reorderings are observed during the runtime. In the case when the order naturally preserved there is almost no overhead: when the watermark arrives, all computations are already done. The probability of reordering could be managed on a business-logic level and optimized by the developer. In experiments section it is shown that the computational nodes count is one of such parameters. Regarding the weaknesses, this method can generate additional items, which lead to extra network traffic and computations. Experiments, which are shown in the section~\ref{fs-experiments-section} demonstrate that the number of extra items is low.



\section {Implementation}
%%% fs-run-time-impl Implementation
\label{fs-impl}

\FlameStream\ is implemented in Java, using Akka framework for messaging and Apache Zookeeper for cluster state management.

\subsection{Ordering model}
The meta-information of data item is implemented as a tuple of a {\it global time}, a {\it trace}, {\it child ids} and a {\it tombstone flag}.

\[Meta := (GlobalTime, ChildIds[], Trace, IsTombstone)\]

Global time is assigned to data item once the item enters the system. It is a pair of logical time and the identifier of the front. The identifier is used to resolve time collisions within different fronts. It is important to notice that we do not rely on any clock synchronization between nodes. The only implication of the clock skew is the system degradation regarding latency: 1 ms of the fronts clock difference appends 1 ms to minimal latency.

Each map operation can produce multiple items from one.  An ordinal number, child id, is stored in the meta information to differentiate them. {\it ChildIds} is an array of child ids, that corresponds to all visited map operations.

The global time and child ids are enough to uniquely identify data item within a stream if all processing is done in-order. If there are any grouping repairs happened during processing, multiple items with the same global time and child ids exist in the stream. 

To match tombstone item with an invalidated item, there is {\it Trace} value stored in the meta-information. The unique random 64-bit identifier is assigned to each physical operation. The trace is a xor of all operations' ids visited by item so far. Invalid item and the corresponding tombstone go along the same path, because they have the same payload and the balancing functions are deterministic, so they have the same trace values.

Metas are compared lexicographically and this order is in line with the \FlameStream's ordering model.

Figure~\ref{logical-graph-ops-figure} shows the topology of each operation and how it affects the meta. Grouping and merge operations update trace of the data item by xoring initial trace with its physical id. Map and broadcast apply the same update, but also append child id for each output item.

\begin{figure}[ht]
  \centering
  \includegraphics[width=\linewidth]{pics/operations}
  \caption{The topology of each operation and how it affects the meta}
  \label {logical-graph-ops-figure}
\end{figure}

\label{mininal-time}

\subsection{Minimal time within stream}

To release an item from the barrier we need to ensure that there are no in-flight tombstones for that item. 

\newtheorem{minimal-time-claim}{Lemma}

\begin{minimal-time-claim}
  If data item {\it D} has global time {\it GT} greater than the global time of each in-flight element, then all tombstones for that item had already arrived at the barrier.
\end{minimal-time-claim}

\begin{proof}
  Let $D_{tomb}$ be an in-flight tombstone for {\it D}. According to the definition of the tombstone item, $D_{tomb}$ and {\it D} has the same global time {\it GT}. 
 We assumed that there are no in-flight element with the global time equal to {\it GT}. Contradiction.
  
  New tombstones for {\it D} cannot be generated because items with global time greater than {\it GT} cannot trigger repair that affects {\it D}.

  This implies that if the stream does not contain items with the global time less $ \le $ {\it GT}, then all tombstones for {\it D} had already arrived at the barrier. 
\end{proof}

Therefore, to output an item from the barrier, we should ensure that there are no items in the stream with the global time less than or equal to the global time of this item.

To track the global time of in-flight items we adopt an idea of {\it acker task} inspired by Apache Storm~\cite{apache:storm}. Acker tracks data items using a checksum hash, called {\it XOR}. When the item is sent or received by an operation, its global time and checksum are sent to the acker. This message is called {\it ack}.
 Acker groups acks by a global time and xors received checksum hashes. 
When an item is sent and later received by the next operation, xoring corresponding {\it XOR}s would yield zero.

Acks are overlapped to nullify {\it XOR} only when an item arrives at the barrier. That is, ack for receive is sent only after both processing and the ack sending for the transformed item, as illustrated in Figure~\ref{acker}. This technique guarantees that the {\it XOR} for some global time is equal to zero only if there are no in-flight elements with such global time.

\begin{figure}[ht]
  \centering
  \includegraphics[width=0.8\textwidth]{pics/acker}
  \caption{The example of tracking minimal time using acker}
  \label {acker}
\end{figure}

The minimal time within a stream is the minimal global time with non-zero {\it XOR}. On minimal time changes, acker broadcasts the {\it new minimal time notification}. Therefore, the barrier can release elements with global time {\it GT} once it received notification with time greater than {\it GT}.

To ensure that no fronts can generate item with the specific timestamp, each front periodically sends to acker special message called {\it heartbeat} indicating that front will not issue items with a timestamp lower than the reported. The value in the ack table can become zero only after the corresponding heartbeat arrives.

The proposed mechanism could be isolated by hash range. This change allows us releasing from barriers on different workers independently. This feature is also known as early key availability.



\section {Experiments}
%%%% fs-run-time-experiments   Experiments

\label {fs-experiments-section}

The performance of Flame Stream is compared with the  performance of Flink both running on 1, 2, 3, 5 and 10 computers with data set size of 1/3, 2/3, and 1 of the whole data set  (describe the data set here).

The latency and overall processing time is measured. 

More details on computers and operating environment are needed here.

Put results, preferable charts but tables are also good, here.

The results of experiments clearly show that we are in certain sense good.  (Please be more specific here.)



\section {Related work}
%%%% fs-run-time-related  Related Work

\label {fs-related-section}

{\bf Data flow:}
One specific detail of our computational model is cyclic data flow graph support. Naiad~\cite{Murray:2013:NTD:2517349.2522738} by Microsoft Research provides an implementation of this idea. Nevertheless, Naiad applies cycles only for iterative computations and allows for each operation to have its own state. 

Another similar concept of Naiad is the usage of logical timestamps to monitor progress. However, to propagate the latest timestamp the pessimistic approach of notifications broadcasting is defined. Therefore, with the assumption of infrequent out-of-order items, our optimistic behavior is more relevant.

In our model, map and group operations are used as core processing primitives. Google Dataflow~\cite{Akidau:2015:DMP:2824032.2824076} provides the same idea. The primary distinction is that Google Dataflow has different state model which is not applicable to real-world MapReduce stream processing tasks. Additionally, this model provides different window types for grouping. FlameStream grouping is aligned with fixed-sized sliding window, but it is possible to implement other kinds of windows by using cycle and grouping with window-affiliation hash.

{\bf State:}
The common approach to state management is to give a user the ability to handle a state of almost any supported operations. Such behavior is implemented, for instance, in Apache Flink~\cite{carbone2015apache}, Storm~\cite{apache:storm}, Samza~\cite{Noghabi:2017:SSS:3137765.3137770}, Naiad~\cite{Murray:2013:NTD:2517349.2522738}.
To the best of our knowledge, FlameStream is the only open-source stream processing system that:
\begin{itemize}
    \item Stateless in terms of business-logic
    \item Supports any MapReduce transformations 
\end{itemize}

{\bf Deterministic processing and handling out-of-order items:}
Research works on this topic analyze different methods, but most of them are based on buffering.

K-slack technique can be applied, if network delay is predictable \cite{Babu:2004:EKC:1016028.1016032}. The key idea of the method is the assumption that an event can be delayed for at most K time units. Such assumption can reduce the size of the buffer. However, in the real-life applications, it is very uncommon to have any reliable predictions about the network delay.

IOP and OOP architectures are popular within research works and industrial applications. IOP architecture is applied in \cite{Cranor:2003:GSD:872757.872838, Arasu:2006:CCQ:1146461.1146463}. OOP approach is introduced in \cite{Li:2008:OPN:1453856.1453890} and it is widely used in the industrial stream processing systems, for instance, Flink \cite{carbone2015apache} and Millwheel \cite{Akidau:2013:MFS:2536222.2536229}. However, these methods require buffering all input items before each order-sensitive operation.

In~\cite{Zacheilas:2017:MDS:3093742.3093921}, the mechanism to control the trade-off between determinism and low latency is proposed. However, such approach only provides for relaxing determinism properties to achieve low latency if needed.

Regarding optimistic techniques, there is less scientific and industrial activity. In \cite{Wei:2009:SSO:1559845.1559973} so-called {\it aggressive} approach is proposed. They introduced an idea of deletion messages that is very similar to our tombstone items. However, authors describe their idea in an abstract way and do not provide any techniques to apply their method for the arbitrary operations. Another optimistic strategy is detailed in \cite{Li2011}. This method is probabilistic: it guarantees the right order with some probability. Besides, it supports only the limited number of query operators.

{\bf Tracking mechanisms within stream:}
One important task that FlameStream faces is handling of the minimal global time. In Apache Storm~\cite{apache:storm} acker is used to eliminate item traces. Unlike Strom, we use acker to track the least global time of in-flight items and to detect package losses.


\section {Conclusion and future work}
%%% fs-run-time-conclu   Conclusions

\label {fs-conclusions}

We introduced the model and implementation of distributed scalable stream processing system. Our contribution consists of the following key results:

\begin{itemize}
    \item Proposed computational model is stateless in terms of business-logic. On the other hand, the model is MapReduce complete. Therefore, our system can solve a wide range of analytical and algorithmic tasks.
    \item The user of the system is able to influence the computational layout. Such approach is motivated by the fact that typically user has deeper knowledge of underlaying data distribution.
    \item The collision management is implemented in optimistic manner. Unlike prevention-based collision management, optimistic approach does not impose overhead in the case of nothing is failed.
    \item The computational process is deterministic up to the time assignment at fronts.
    \item The new way to determine a moment when data items can be released from stream without loosing exactly-once guarantee, called adaptive micro-batching, was introduced.  
    \item It was shown that our system outperforms Apache Flink on the real-life computations.   
\end{itemize}

Considering the further progress, the following features are planned to be implemented:
\begin{itemize}
    \item Fault tolerance, and, hence, at least once and exactly once guarantees
    \item Independent readiness by key
\end{itemize}











\bibliographystyle{splncs03}
\bibliography{../../bibliography/flame-stream}

\appendix
\section{\\MapReduce transformation}
%% It is just an empty TeX file.
%% Write your code here.

\label{fs-drifting}

%\subsection{MapReduce transformation}
The map stage of MapReduce can be expressed in terms of our map operation. 
The generic reduce stage can be presented as

\begin {tabbing}
1234\=1234\= \kill
{\bf for} $mapped \in values$ {\bf do}   \\
\>$accumulator$ := combine ($mapped$, $accumulator$); \\
{\bf end for} \\
{\bf return } $accumulator$;
\end {tabbing}

The {\it accumulator} is an explicit state that should be kept between subsequent iterations.
%
%\begin{algorithm}
%\caption{Generic reduce stage}
%\label{reduce}
%\begin{algorithmic}
%  \Function{reduce}{$key$, $values$}
%    \State $accumulator$ \Comment{reduce's state}
%    \ForAll{$v \in values$} 
%      \State \Call{combine}{$v$, $accumulator$}
%    \EndFor
%    \State \Return \Call{map}{$accumulator$}
%  \EndFunction
%\%\end{algorithmic}
%\end{algorithm}
%
To implement reduce stage we apply the drifting state technique and make the accumulator value a part of the stream. 
Figure~\ref{mapreduce-graph-figure} shows a generic graph for MapReduce transformation. 
Map and reduce stages are highlighted with a dashed line. 

\begin{figure}[ht]
  \centering
  \includegraphics[width=0.6\textwidth]{pics/mapreduce}
  \caption{Logical graph for MapReduce transformations}
  \label {mapreduce-graph-figure}
\end{figure}

There are four types of data items in this stream: {\it input,} {\it mapped,}   {\it accumulator,} and {\it  reduced.} 
% Mapped, accumulator and reduced items have the key-value structure of a payload. 
The operations of the stream have the following purposes:

\begin{itemize}
  \item The first map operation outputs mapped items according to map stage of MapReduce model.
  
  \item The grouping with $WindowSize=2$ groups the $accumulator$  with next $mapped$  item. 
  % The hash function is designed to return distinct values for payloads with distinct keys
  
  \item The combine map  produces new state of $accumulator$ to be sent  to grouping.
  
  %The second map implements the actual combining. It accepts inputs that have a form of: \textit{(mapped item)} or \textit{(accumulator item, mapped item)}. The first kind is transformed into some initial value. 
%  The second one is combined into the new accumulator item as specified by reduce stage of MapReduce. 
 % The tuples with structure \textit{(mapped item, accumulator item)} are filtered out
  \item The final map converts $accumulator$ into final reduce output.
\end{itemize}

Ordering rules  guarantee that each $accumulator$  item always arrives at the grouping right before next not yet combined mapped item.
%Hence, each mapped item that has not been combined yet would be grouped with the right accumulator item. 
%Additionally, when combine map accepts tuple {\it (mapped item, accumulator item)}, 
%then it means that mapped item was generated before accumulator item, and therefore, it had been already combined. 
The cycle gives the ability for new accumulator items to get back in the grouping operation. 
%The accumulator map transforms the accumulator item into the final reduced item right before sink. 
Thereby, the stream reacts to each input item by generating new reduced item, which contains the actual value of the reduce stage.

% \subsection{Example: word count}

% We illustrate the MapReduce algorithm with an example of word counting. Map stage of this algorithm transforms each input word into key-value pair where the word is a key, and the value is 1. Reduce stage sums all values into the final result for the specific key. In this case, the accumulator map is omitted, because the accumulator is the actual result of the reduce stage.

% The example of input/output items, which are generated/ transformed by the part of the logical graph, is shown in Figure~\ref {word-count-figure}. According to our graph for MapReduce transformations, the item {\it m[dog, 1]} represents mapped item with key "dog" and value 1. The item {\it a[dog, 1]} describes accumulator item with key "dog" and value 1. Figure shows how the model reacts on two consequent input items containing word "dog". The meta-information of items is omitted for simplification. More precisely, there are four stages separated by dotted lines:

% \begin{enumerate}
%     \item New mapped item with key "dog" arrives at grouping with an empty state. Grouping outputs tuple with this single item. Combine map transforms it into the first accumulator item for key "dog" and value 1.
%     \item The accumulator item arrives at grouping after it went through the cycle. It is grouped in the tuple with the mapped item that has been already in the state with key "dog". However, combine map drops this tuple, because of the order of items.
%     \item New mapped item with key "dog" arrives at grouping. It is inserted right after the accumulator in the bucket for key "dog". The grouping operation outputs tuple containing the accumulator item and new mapped item. Map operation combines reduced and mapped items into new reduced items with key "dog" and value 2.
%     \item New accumulator item arrives at grouping through the cycle, but new generated tuple is not accepted by combine map, as well as in step 2.
% \end{enumerate}

% \begin{figure}[ht]
%   \centering
%   \includegraphics[scale=0.5]{pics/wordcount}
%   \caption{The example of input/output items, which are generated/ transformed by the part of the logical graph for word counting}
%   \label {word-count-figure}
% \end{figure}


\end {document}

\endinput
you can put whatever here