%%%% fs-run-time-motivation  FlameStream motivation

Unlike batch and micro-batch processing, stream processing is inherently non-deterministic~\cite{Zaharia:2012:DSE:2342763.2342773}. Particularly, there is no guarantee that the messages will be processed in the same order and the system produces the same result between any two runs. Although such behavior is observed in most state-of-the-art stream processing systems, it has several significant pitfalls:

\begin{itemize}
    \item It is natural for the user of a software system to assume that two independent runs within the same input data produce exactly the same result. The fact that this contract can be violated is able to cause misleadings and complicates the usage of the system
    \item The lack of determinism leads to the loss of reproducibility of the results, that in turn makes the processes of testing and verification excessively complicated
    \item The ability to reproduce predictable results is extremely useful for providing consistency guarantees~\cite{Stonebraker:2005:RRS:1107499.1107504}. The absence of this property forces the usage of heavy transactional protocols to achieve exactly-once semantics~\cite{Carbone:2017:SMA:3137765.3137777, jacques2016consistent} 
\end{itemize}

Existing stream processing solutions do not provide any options for deterministic processing by design. The only way to achieve it is to manually define the order on data items and to buffer all inputs before each order-sensitive operation until it is guaranteed that there are no out-of-order items up the stream. However, this approach requires the implementation of additional logic and can provide overhead in terms of performance.  

Keeping the above in mind, we offer the following requirements for a stream processing system:
\begin{itemize}
    \item Computational model should be deterministic by design, i.e. it should produce deterministic results for any pipelines and business-logic
    \item There should be only one buffer per pipeline to reduce the overhead
    \item Performance of the proposed model should be comparable with existing state-of-the-art stream processing systems
\end{itemize}