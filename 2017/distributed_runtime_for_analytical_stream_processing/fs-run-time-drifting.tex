%% It is just an empty TeX file.
%% Write your code here.

\label{fs-drifting}

%\subsection{MapReduce transformation}
The map stage of MapReduce can be expressed in terms of our map operation. 
The generic reduce stage can be presented as

\begin {tabbing}
1234\=1234\= \kill
{\bf for} $mapped \in values$ {\bf do}   \\
\>$accumulator$ := combine ($mapped$, $accumulator$); \\
{\bf end for} \\
{\bf return } $accumulator$;
\end {tabbing}

The {\it accumulator} is an explicit state that should be kept between subsequent iterations.
%
%\begin{algorithm}
%\caption{Generic reduce stage}
%\label{reduce}
%\begin{algorithmic}
%  \Function{reduce}{$key$, $values$}
%    \State $accumulator$ \Comment{reduce's state}
%    \ForAll{$v \in values$} 
%      \State \Call{combine}{$v$, $accumulator$}
%    \EndFor
%    \State \Return \Call{map}{$accumulator$}
%  \EndFunction
%\%\end{algorithmic}
%\end{algorithm}
%
To implement reduce stage we apply the drifting state technique and make the accumulator value a part of the stream. 
Figure~\ref{mapreduce-graph-figure} shows a generic graph for MapReduce transformation. 
Map and reduce stages are highlighted with a dashed line. 

\begin{figure}[ht]
  \centering
  \includegraphics[width=0.6\textwidth]{pics/mapreduce}
  \caption{Logical graph for MapReduce transformations}
  \label {mapreduce-graph-figure}
\end{figure}

There are four types of data items in this stream: {\it input,} {\it mapped,}   {\it accumulator,} and {\it  reduced.} 
% Mapped, accumulator and reduced items have the key-value structure of a payload. 
The operations of the stream have the following purposes:

\begin{itemize}
  \item The first map operation outputs mapped items according to map stage of MapReduce model.
  
  \item The grouping with $WindowSize=2$ groups the $accumulator$  with next $mapped$  item. 
  % The hash function is designed to return distinct values for payloads with distinct keys
  
  \item The combine map  produces new state of $accumulator$ to be sent  to grouping.
  
  %The second map implements the actual combining. It accepts inputs that have a form of: \textit{(mapped item)} or \textit{(accumulator item, mapped item)}. The first kind is transformed into some initial value. 
%  The second one is combined into the new accumulator item as specified by reduce stage of MapReduce. 
 % The tuples with structure \textit{(mapped item, accumulator item)} are filtered out
  \item The final map converts $accumulator$ into final reduce output.
\end{itemize}

Ordering rules  guarantee that each $accumulator$  item always arrives at the grouping right before next not yet combined mapped item.
%Hence, each mapped item that has not been combined yet would be grouped with the right accumulator item. 
%Additionally, when combine map accepts tuple {\it (mapped item, accumulator item)}, 
%then it means that mapped item was generated before accumulator item, and therefore, it had been already combined. 
The cycle gives the ability for new accumulator items to get back in the grouping operation. 
%The accumulator map transforms the accumulator item into the final reduced item right before sink. 
Thereby, the stream reacts to each input item by generating new reduced item, which contains the actual value of the reduce stage.

% \subsection{Example: word count}

% We illustrate the MapReduce algorithm with an example of word counting. Map stage of this algorithm transforms each input word into key-value pair where the word is a key, and the value is 1. Reduce stage sums all values into the final result for the specific key. In this case, the accumulator map is omitted, because the accumulator is the actual result of the reduce stage.

% The example of input/output items, which are generated/ transformed by the part of the logical graph, is shown in Figure~\ref {word-count-figure}. According to our graph for MapReduce transformations, the item {\it m[dog, 1]} represents mapped item with key "dog" and value 1. The item {\it a[dog, 1]} describes accumulator item with key "dog" and value 1. Figure shows how the model reacts on two consequent input items containing word "dog". The meta-information of items is omitted for simplification. More precisely, there are four stages separated by dotted lines:

% \begin{enumerate}
%     \item New mapped item with key "dog" arrives at grouping with an empty state. Grouping outputs tuple with this single item. Combine map transforms it into the first accumulator item for key "dog" and value 1.
%     \item The accumulator item arrives at grouping after it went through the cycle. It is grouped in the tuple with the mapped item that has been already in the state with key "dog". However, combine map drops this tuple, because of the order of items.
%     \item New mapped item with key "dog" arrives at grouping. It is inserted right after the accumulator in the bucket for key "dog". The grouping operation outputs tuple containing the accumulator item and new mapped item. Map operation combines reduced and mapped items into new reduced items with key "dog" and value 2.
%     \item New accumulator item arrives at grouping through the cycle, but new generated tuple is not accepted by combine map, as well as in step 2.
% \end{enumerate}

% \begin{figure}[ht]
%   \centering
%   \includegraphics[scale=0.5]{pics/wordcount}
%   \caption{The example of input/output items, which are generated/ transformed by the part of the logical graph for word counting}
%   \label {word-count-figure}
% \end{figure}
