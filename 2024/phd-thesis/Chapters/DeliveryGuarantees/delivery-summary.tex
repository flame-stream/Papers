In this chapter, we introduced a formal framework for modeling delivery guarantees in distributed stream processing. We described the behavior of state-of-the-art research and industrial systems in terms of the proposed framework. It was shown that the property of determinism is tightly connected with the concept of of exactly-once. We proved that non-deterministic systems must persistently save a state of non-commutative operations before output delivery in order to achieve exactly-once. Therefore, in non-deterministic systems that guarantee exactly-once processing, output elements must wait to be released until after the next snapshot is taken.

Most of the state-of-the-art stream processing systems use one of the following approaches to achieve exactly-once delivery guarantee: 

\begin{itemize}
    \item Inherit exactly-once from batch processing using small-sized batches (micro-batching)
    \item Apply distributed transaction control protocols which guarantee that states are saved before delivery of elements affected by these states
    \item Write results of an operation to external storage on each input element
\end{itemize}

Using the proposed formal framework we demonstrated the foundations behind high latency for the exactly-once mode in these systems. In the first case, latency cannot be lower than the batching period, in the second case, latency depends on the snapshotting period, while in the third case latency is bounded below by the duration of external writes.