This thesis has made contributions to the field of stream processing, addressing the challenges of consistency in a distributed environment. As demonstrated in Chapter~\ref{thesis-chapter-literature-review}, prior work on this topic has made significant progress, with many ideas being implemented in state-of-the-art Stream Processing Engines (SPEs). In this thesis, we focused on building formal models for the existing problems and identifying areas for improvement using the proposed models. 

A brief overview of the main contributions is the following:
\begin{itemize}
    \item {\bf Formal model for delivery guarantees}: we developed a formal model for the concept of delivery guarantees and showed connections between the property of determinism and exactly-once delivery guarantee, demonstrating the potential efficiency in terms of latency of deterministic systems ensuring exactly-once.
    \item {\bf Mechanism for exactly-once based on the lighweight determinism}: we proposed {\em drifting state} technique that offers a lightweight and efficient method for ensuring determinism and exactly-once processing. This approach allows us to reduce processing latency significantly, providing an improvement over current industrial solutions.
    \item {\bf Formal model for the substream management and the lower bound estimation for network traffic}: we have formalized the substream management problem. We demonstrated that the extra traffic for a substream management solution cannot be lower than $O(K||\Pi||)$, where $|\Pi|$ is the number of computational nodes and $K$ is the number of substreams. 
    \item {\bf Substream management approach reaching the lower bound of network traffic}: our findings show that existing punctuation-based frameworks are far from optimal ($O(K|\Pi|^2)$), prompting us to develop the \tracker\ technique. This new approach meets the lower bound of network traffic overhead and is able to handle cycles and fine-grained substreams, making it suitable for advanced stream processing applications.
\end{itemize}

