This thesis has made contributions to the field of stream processing, addressing the challenges of consistency in a distributed environment. As demonstrated in Chapter~\ref{thesis-chapter-literature-review}, prior work on this topic has made significant progress, with many ideas being implemented in state-of-the-art Stream Processing Engines (SPEs). In this thesis, we focused on building formal models for the existing problems and identifying areas for improvement using the proposed models.

Firstly, we developed a formal model for the concept of delivery guarantees. We showed connections between the property of determinism and exactly-once delivery guarantee, demonstrating the potential efficiency in terms of latency of deterministic systems ensuring exactly-once. We proposed {\em drifting state} technique that offers a lightweight and efficient method for ensuring determinism and exactly-once processing. This approach allows us significantly reduce processing latency, providing an improvement over current industrial solutions.

Secondly, we have formalized the substream management problem. We demonstrated that the extra traffic for a substream management solution cannot be lower than $O(K||\Pi||)$, where $|\Pi|$ is the number of computational nodes and $K$ is the number of substreams. Our findings show that existing punctuation-based frameworks are far from optimal ($O(K|\Pi|^2)$), prompting us to develop the \tracker\ technique. This new approach meets the lower bound of network traffic overhead and is able to handle cycles and fine-grained substreams, making it suitable for advanced stream processing applications.

