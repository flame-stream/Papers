In this chapter, we designed and implemented a new substreams management technique called \tracker\ that does not require injecting service elements directly into the stream. Instead, we mark all data elements with ordered labels and use the distributed agent, which notifies operators that a substream ends. Our approach has the following features:

\begin{itemize}
     \item {\bf Cyclic dataflows support:} the method is suitable for problems that require non-linear executions: graph traversing, iterative algorithms, etc. We evaluated this feature within a real-life problem.
     \item {\bf Low overhead:} we showed that our implementation achieves the lower bound of service traffic overhead. We demonstrated that \tracker\ insignificantly affects the throughput of an SPE in practice.
     \item {\bf Fine-grained substreams support:} \tracker\ framework is suitable for substreams consisting of a few elements. This feature is achieved due to low traffic overhead and another way of notifications propagation.
     \item {\bf Scalability:} extra network traffic from operators can be distributed between multiple nodes. Experiments on synthetic dataflows indicated the practical feasibility of balancing extra traffic.
\end{itemize}

These properties of the \tracker\ framework create a possibility to apply substream management in new applications; as indicated in the experimental section, this includes smart caching of operator state and latency-conscious windowed joins.

Fault tolerance is a limitation of the solution presented in this chapter. SPE should ensure recovery of the \tracker\ agent in case of failures. We are leaving this topic for future work.
