\subsection{Consistency in stream processing}

\subsubsection{Delivery guarantees}

\paragraph{Types of delivery guarantees}\mbox{} \\

Delivery guarantees, such as at-most-once, at-least-once, and exactly-once, define how and when data is delivered and processed in the system~\cite{fragkoulis2024survey, carbone2018scalable, Akidau:2013:MFS:2536222.2536229}:
\begin{itemize}
    \item At-most-once is a delivery guarantee in system processing where an input element is processed atomically (with all its derivatives) or not processed at all. This means that the system ensures that each input element is processed only one time at maximum, but there's a possibility that some input elements may not get processed.
    \item At-least-once is a type of delivery guarantee where the system ensures that every input element is processed one or more times. This could lead to potential duplication if an input contains duplicated items. This method ensures no data is lost, but it does not prevent possible repetitions.
    \item Exactly-once is a system processing term that guarantees each input element or transaction is processed exactly one time. This means that it ensures there is no data loss and no duplicate processing. It combines both the prevention of input data losses and the avoidance of repeated delivery of results.
\end{itemize}

The type of delivery guarantee chosen for a system directly influences how that system handles failures~\cite{zhang2024survey, silvestre2021clonos}. In at-most-once delivery, if a failure occurs during processing, the system may not attempt to reprocess the input, leading to potential data loss. In at-least-once delivery, the system ensures no data is lost by reprocessing inputs in case of failures. However, this can lead to duplicate processing and results if the system doesn't track what has already been processed. In exactly-once delivery, the system ensures that every input is processed exactly one time, even in the case of failures. This is the most complex to implement as it needs to ensure atomicity between reading input data, processing, and delivering results, and it requires the system to be able to recover to a consistent state after a failure.


% Delivery guarantees, such as at-most-once, at-least-once, and exactly-once, define how and when data is delivered and processed in the system. These guarantees have a direct impact on the consistency of the system:
% \cite{carbone2018scalable}

\paragraph{State recovery and consistency}
\cite{carbone2018scalable}

\paragraph{Suitable consistency levels for various types of problems}
\cite{boden2017distributed}

\subsubsection{Completeness of results}

\paragraph{Substreams}
\cite{Tucker:2003:EPS:776752.776780}

\paragraph{Window aggregations}
\cite{verwiebe2023survey}

\paragraph{Handling out-of-order events}
\cite{traub2018scotty, wang2021consistency}

\subsubsection{Statistical consistency and anomalies}
\cite{tellis2018detecting}

\subsection{Fault tolerance mechanisms in stream processing}

\subsubsection{Global state checkpointing}

\paragraph{Micro-batching}
\cite{Zaharia:2012:DSE:2342763.2342773}

\paragraph{Asynchronous snapshots}
\cite{Carbone:2017:SMA:3137765.3137777}

\subsubsection{Critical path recovery}

\paragraph{Lineage stash}
\cite{Wang:2019:LSF:3341301.3359653}

\paragraph{Effective determinism}
\cite{Akidau:2013:MFS:2536222.2536229, silvestre2021clonos}
