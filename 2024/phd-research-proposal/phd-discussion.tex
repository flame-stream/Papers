\label{phd_discussion}

The concepts of exactly-once, at-least-once, and at-most-once are not formally defined and can be interpreted in different ways. Nevertheless, in Section~\ref{consistency_overview}, we demonstrated that delivery guarantees, particularly exactly-once, are fundamental for other types of consistency. Result completeness is meaningful only in scenarios where no elements are lost or duplicated. Transactional guarantees require exactly-once or at-most-once delivery to ensure atomicity. Additionally, a transactional system may need determinism along with the exactly-once guarantee.

In Section~\ref{phd-related-fault-tolerance}, we showed that there are multiple methods to ensure delivery guarantees in the event of failures, each with its own drawbacks. One significant disadvantage of many fault tolerance techniques is the high latency overhead for regular processing. There is a lack of analysis on which properties of stream processing systems affect processing latency under exactly-once guarantees. We see an opportunity to design more efficient techniques for exactly-once guarantees through the formalization of delivery guarantees definitions and by researching the connection between delivery guarantees and determinism.

In Section~\ref{consistency_overview}, we discussed the topic of inefficient mechanisms for monitoring the completeness of results. State-of-the-art techniques for detecting result completeness generate high network traffic, which can impact the performance of end-to-end processing. This issue also arises in snapshot-based methods for ensuring exactly-once delivery as we showed in Section~\ref{phd-related-fault-tolerance}. Therefore, an interesting research direction is to estimate the lower bound for network traffic in the context of the completeness of results monitoring and to design a method that achieves this lower bound.