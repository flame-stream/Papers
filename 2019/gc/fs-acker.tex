\documentclass{vldb}

\usepackage{graphicx}
\usepackage{url}
\usepackage{hyperref}
\usepackage{algorithm} % for algorithms
\usepackage{algpseudocode}
% \usepackage{booktabs} % For formal tables
% `\usepackage{amsthm} % For claims
\usepackage{balance}  % for  \balance command ON LAST PAGE  (only there!)
\usepackage{caption}
\usepackage{subcaption}

% \theoremstyle{remark}

\pagestyle{empty} % removes running headers

\newcommand{\PicScale}{0.5}
\newcommand {\FlameStream} {FlameStream}
\newcommand {\tracker} {trAcker}
\newcommand {\acker} {Acker}

% Include information below and uncomment for camera ready
\vldbTitle{Tracking Dependencies in Distributed Streaming Dataflows}
\vldbAuthors{Nikita Sokolov, Artem Trofimov, Igor Kuralenok, Nikita Marshalkin, and Boris Novikov}
\vldbDOI{https://doi.org/10.14778/xxxxxxx.xxxxxxx}
\vldbVolume{12}
\vldbNumber{xxx}
\vldbYear{2019}

\begin{document}

\title {Garbage Collection in Distributed Streaming Dataflows}

\numberofauthors{5}

\author{
\alignauthor
Nikita Sokolov\\
    \affaddr{ITMO University}\\
    \affaddr{Saint Petersburg, Russia}\\
    \email{faucct@gmail.com}
\alignauthor
Artem Trofimov\\
    \affaddr{Yandex}\\
    \affaddr{Saint Petersburg, Russia}\\
    \email{tomato@yandex-team.ru}
\alignauthor
Igor Kuralenok\\
    \affaddr{Yandex}\\
    \affaddr{Saint Petersburg, Russia}\\
    \email{solar@yandex-team.ru}
\and 
\alignauthor
Nikita Marshalkin\\
    \affaddr{VK}\\
    \affaddr{Saint Petersburg, Russia}\\
    \email{n.marshalkin@corp.vk.com}
\alignauthor
Boris Novikov\\
    \affaddr{National Research University Higher School of Economics}\\
    \affaddr{Saint Petersburg, Russia}\\
    \email{borisnov@acm.org}
}

\maketitle

\begin{abstract}
%The majority of state-of-the-art stream processing systems faces a problem of obtaining notifications when the specific set of input elements have already been applied to all operators in a dataflow. Such notifications are commonly used to take consistent state snapshots or to release all descendants of an input element atomically while preserving exactly-once delivery guarantee. To generate these notifications, there is a need to track dependencies between input and their descendants and to inform when all descendants have been completely processed. In this work, we propose a method for tracking dependencies between streaming elements that can be applied for both cyclic (iterative) and acyclic dataflows. We demonstrate that our technique has low latency and throughput overhead within various setups and can significantly outperform methods employed by state-of-the-art stream processing systems.

%In distributed stream processing, it is hard to determine a correspondence between input and output elements due to complex transformations, filterings, producing multiple items from a single, and so on. On the other hand, mechanisms for taking consistent state snapshots and for transactional stream processing require notifications when a system has already processed a specific set of input elements together with all their descendants. To produce these notifications, one needs to track dependencies between items in a dataflow. In this work, we propose a method for monitoring dependencies between streaming elements applicable to cyclic dataflows as well. We enrich each data item with a logical timestamp that identifies corresponding input items. We build a scalable method to watch for the computational progress and to notify when dataflow contains no elements with specific timestamps. These notifications indicate when a system completely processed all descendants of concrete input elements. We demonstrate that our technique provides low notification latency while having almost no throughput overhead on regular processing, and can outperform methods employed by state-of-the-art stream processing systems.

% In distributed stream processing, a system often needs to determine a correspondence between input and output elements, e.g. to take state snapshot affected by  or to release all descendants of an input element atomically while preserving exactly-once delivery guarantee. This task is hard due to complex transformations, filterings, producing multiple elements from a single, etc. 

In distributed stream processing, it is often required to manage in-memory state for independent keys with undefined lifetime, for example user sessions. These keys can be either internal or external. Such state includes buffers for joining streams. When this state is no longer required, it could be dumped to disk or even deleted. In this work, we propose a method for monitoring a presence of in-flight elements per key applicable to cyclic dataflows as well. We demonstrate that our method provides low notification latency and compare its memory footprint with methods employed by state-of-the-art stream processing systems.

\end{abstract}

% \keywords{Data streams, exactly-once, drifting state, optimistic OOP}

\thispagestyle{empty}

\section {Conclusion}
\label {fs-acker-conclusion}

% In this work, we formulated a problem of dependency tracking between input and output elements in streaming dataflows. We demonstrated that state-of-the-art distributed stream processing systems face this problem in state snapshotting mechanisms~\cite{Carbone:2017:SMA:3137765.3137777, apache:storm}, the materialization of time-varying relations~\cite{Begoli:2019:OSR:3299869.3314040}, and atomic delivery of all descendants of an input item~\cite{we2018adbis}.  

% To solve this problem, we proposed a mechanism that adopts ideas from the Apache Storm completion tracking mechanism called \acker. We extend each data item with a logical timestamp that denotes corresponding input items and tracks if dataflow contains elements with specific timestamps. 

% Our solution, called \tracker\ inherits from \acker\ fine-grained tracking, and cyclic dataflows support and provides the following features:
% \begin{itemize}
%     \item {\bf Notifications order preservation:} the order of notifications that system completely processed some set of input items does not contradict the order of input elements. This feature allows using \tracker\ for state snapshotting.
%     \item {\bf Dataflow locality:} different parts of a dataflow can receive independent notifications that allow a stream processing system to apply efficient asynchronous state snapshotting algorithms, e.g., one that is used in Apache Flink~\cite{Carbone:2017:SMA:3137765.3137777}. 
%     \item {\bf Scalability:} we introduced a distributed version of \tracker\ that allows a system to distribute extra network traffic between all computational units. 
%     \item {\bf Low overhead:} \tracker\ does not produce any significant performance penalty and does not affect the throughput of a distributed streaming dataflow.
% \end{itemize}

% We conducted a series of experiments and compared the proposed method with a baseline approach based on the markers mechanism used in Apache Flink. We demonstrated that both centralized and decentralized implementations of \tracker\ provide lower notification latency that does not considerably degrade with an increase of a logical graph size or a cluster size. Experiments also showed that \tracker\ has lower throughput overhead in case of fine-grained tracking.

In this work, we formalized the problem of substreams management and demonstrated the main properties of the most common punctuations approach that bases on injecting special elements (punctuations) into the stream. On the one hand, receiving such elements naturally guarantees the end of a substream since that moment, because all elements within the operator are totally ordered. On the other hand, propagating punctuations through a whole dataflow leads to the lack of cyclic dataflows support, poor scalability, and excessive extra traffic.

We designed and implemented a new substreams management technique called \tracker\ that does not require injecting service elements directly into the stream. Instead, we mark all data elements with ordered labels and designed the distributed agent, which notifies operators that a substream ends since receiving an input element with the specified label. Our approach provides the following features:

\begin{itemize}
     \item {\bf Cyclic dataflows support:} the method is suitable for problems that require non-linear executions: graph traversing, iterative algorithms, etc. We evaluated this feature within the real-life problem.
     \item {\bf Low overhead:} we showed that our implementation implies a lower amount of extra network traffic. We demonstrated that it produces low performance penalty, and insignificantly affects the throughput of SPE.
     \item {\bf Better scalability:} extra network traffic from operators can be distributed between multiple nodes. Experiments on synthetic dataflows indicated the practical feasibility of balancing extra traffic.
\end{itemize}



% both centralized and decentralized implementations of \tracker\ provide lower notification latency that does not considerably degrade with an increase of a logical graph size or a cluster size. Experiments also showed that \tracker\ has lower throughput overhead in case of fine-grained tracking.

\bibliographystyle{abbrvurl}
\bibliography{../../bibliography/flame-stream}

\end {document}

\endinput
