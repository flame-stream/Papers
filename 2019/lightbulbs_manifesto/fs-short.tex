\documentclass[sigconf]{acmart}

\usepackage{graphicx}
\usepackage{algorithm} % for algorithms
\usepackage{algpseudocode}
\usepackage{booktabs} % For formal tables
\usepackage{amsthm} % For claims

\theoremstyle{remark}

\settopmatter{printacmref=false, printccs=true, printfolios=true}
\pagestyle{empty} % removes running headers

\newcommand{\PicScale}{0.5}
\newcommand {\FlameStream} {FlameStream}
\begin{document}

% \copyrightyear{2018} 
% \acmYear{2018} 
% \setcopyright{acmcopyright}
% \acmConference[BeyondMR'18]{Algorithms and Systems for MapReduce and Beyond }{June 15, 2018}{Houston, TX, USA}
% \acmBooktitle{BeyondMR'18: Algorithms and Systems for MapReduce and Beyond , June 15, 2018, Houston, TX, USA}
% \acmPrice{15.00}
% \acmDOI{10.1145/3206333.3209273}
% \acmISBN{978-1-4503-5703-6/18/06}

\title {Poster: decentralized validation of statistical properties in distributed stream processing}

% \author{  Igor E. Kuralenok,$^1$     Artem Trofimov,$^ {1,2}$    Artem Permiakov,$^ {2}$   Andrey Pirozhkov,$^ {2}$    Semen Kutuzov,$^ {2}$     Artur,$^ {2}$    and  Boris Novikov$^ {1,2}$ }
% \affiliation{%
% \institution{$^1$JetBrains Research}
%   \city{St. Petersburg} 
%   \country{Russia}
% }
% \affiliation{%
% \institution{$^2$Saint Petersburg State University}
%   \city{St. Petersburg} 
%   \country{Russia}
% }
% \email{\string{ikuralenok, trofimov9artem\string}@gmail.com, borisnov@acm.org}

\begin{abstract}
Streaming applications often require validation of statistical properties. The satisfaction of such properties can directly influence the correctness of the final results. For example, in A/B testing task, there is a need to ensure that users are uniformly distributed among application versions. Otherwise, experiment results may lead to inaccurate business decisions. There are a lot of statistical tools that aim to solve this problem. Most of them assume that the whole data is processed on a single computational unit. However, this approach does not fit in distributed stream processing, because merging all data to a single machine limits throughput and causes extra communication cost. In this work, we investigate the decentralized method for statistical data validation of data streams: each computational unit independently verifies data that is assigned to it. We demonstrate statistical limitations and challenges that we recognized during experiments on a uniform and discrete data. We also show that the API of state-of-the-art stream processing systems is not enough to solve this problem efficiently. Possible solutions to the mentioned issues and open problems are discussed.      
\end {abstract}

% \keywords{Data streams, exactly-once, drifting state, optimistic OOP}

\maketitle

\thispagestyle{empty}

\section {Introduction}
\label {fs-short-intro}

Streaming data validation is a hot research topic~\cite{Xu:2013:MVS:2488222.2488275, frank2018semantic} with a lot of practical applications, including:

{\bf A/B testing}. To obtain reliable A/B testing results there is a need to ensure that approximately the same number of users have experienced each application version. Otherwise, the comparison between versions may be unfair due to unbalanced user groups. This issue can lead to inaccurate business decisions, an increase of churn rate, and financial losses.
    
{\bf Spam detection}. It is assumed that input data for spam detectors contain natural language texts. However, issues in data producers or in a data preparation pipeline could result in a violation of this assumption, e.g. incorrect stemming may shift term frequencies which are commonly used as features for spam classification. Without early detection, this bug can lead to many false decisions and affect service reputation.

Both mentioned problems require validation of the general properties of the whole input data, and hardly can be solved by individual stream elements filtering. A typical solution for this class of problems is a statistical hypothesis testing. In A/B testing example, we can check that the number of users per application version has a Poisson distribution with a constant parameter $\lambda$. Regarding spam detector, one can examine if word frequencies in the whole corpus of input texts fit Zipf law. 

Despite the fact that statistical hypothesis testing on streams is a well-studied topic~\cite{kifer2004detecting, lall2015data}, it becomes much more challenging if a stream is processed on different networked computers as it is implemented in state-of-the-art systems like Flink~\cite{Carbone:2017:SMA:3137765.3137777} and Spark Streaming~\cite{Zaharia:2012:DSE:2342763.2342773}. In this case, individual stream elements may be processed by independent computational units. To apply existing approaches in this setting, we can merge data on a single unit for validation. The main disadvantage of this method is that validation may become a throughput bottleneck for the whole data processing pipeline. 

In this work, we highlight an approach where data is independently validated on each node in a cluster. In our setting, a final decision about hypothesis acceptance is made based on local decisions of computational units. In the rest of this paper, we discuss the main challenges and trade-offs of the proposed approach and share preliminary results.

\section {Decentralized statistical validation}
\label {fs-short-model}

Let $X^{+\infty}=\{x_t\}_{t=1}^{+\infty}$ be an input stream. If this stream is partitioned among $n$ computational units, it can be represented as $X^{+\infty}=\bigsqcup_{i=1}^n X_i^{+\infty}$, where $X_i^{+\infty}=\{x_t^i\}_{t=1}^{+\infty}$. Let $F_0$ and $F_1$ be different families of probability distributions, and $F_1$ is not known. 

Our task is for each $T\in \mathbb{N}$ test a hypothesis that $X^T=\{x_t\}_{t=1}^{T} \sim G(z)$, where $G(z) \in F_0$. Let $y_T$ be a vector of test statistics from all computational nodes,  $y_T = (S(X_1^T),..,S(X_n^T))$ , where $S(X_i^T)$ is a statistic of a chosen test. Hence, to solve this task, we need to design a decision-making function $\varphi(y_T)$ such that $\varphi(y_T) = 0$, if $X^T \sim G(z)$  and $\varphi(y_T) = 1$ otherwise.

The proposed problem is similar to decentralized change detection approaches mentioned in~\cite{tartakovsky2008asymptotically, tran2014change}. However, these techniques consider a change from one unknown data distribution to another unknown one, while our task is to detect a change of a particular distribution family. Besides, these works do not take into account different data partition strategies peculiar to distributed streaming pipelines and important performance characteristics, e.g. latency and throughput.

\section {Experiments}
\label {fs-short-experiments}

\begin{figure*}[t!]
    \begin{subfigure}[b]{0.31\textwidth}
            \includegraphics[width=\linewidth]{pics/throughput}
            \caption{Throughput for centralized and decentralized approaches}
            \label{throughput}
    \end{subfigure}%
    \hspace{5mm}
    \begin{subfigure}[b]{0.31\textwidth}
            \includegraphics[width=\linewidth]{pics/detection_rate}
            \caption{Change detection speed for round robin and hash partitionings}
            \label{detection_rate}
    \end{subfigure}%
    \hspace{5mm}
    \begin{subfigure}[b]{0.31\textwidth}
            \includegraphics[width=\linewidth]{pics/decision_making.png}
            \caption{A comparison between decision making approaches}
            \label{decision_making}
    \end{subfigure}%
    \caption{Results of preliminary experiments}
\end{figure*}


\indent

{\bf Setup.} We conducted several experiments to evaluate if the proposed problem statement is reasonable. We used words from Wikipedia corpus as an input stream for experiments. Such stream of words can be considered as a part of word count or IDF computing pipelines. A validation task is to check that frequencies of input words fit Zipf distribution. As it was mentioned above, this validation task is useful for fraud detection.

{\bf Throughput.} We measured throughput of two approaches: centralized, where words are merged to a single node for validation, and decentralized. Experiments were conducted with Apache Flink on a cluster of Amazon EC2 small instances with 1 core CPU and 2GB RAM. Figure~\ref{throughput} shows that centralized method significantly limits throughput and scalability of the whole data flow. This result demonstrates that proposed problem statement has practical rationale.

{\bf Change detection speed.} In this experiment we measure the duration (in input events) of change detection. A change in data is simulated using input from generated corpus with lognormally distributed word frequencies. We compared two data partitioning methods. The first one is natural for word count and IDF computing task: each word sticks to a concrete node. In this case, each node processes only its own subsample of all possible words. The second approach is round robin. While it is not suitable for word count, such partitioning ensures that each word may be processed on all computational nodes. As expected, Figure~\ref{detection_rate} demonstrates that round robin provides faster detection, because simulated lognormal data is distributed among nodes in a more uniform way. This behavior means that for faster detection, it may be reasonable to reshuffle words in a round-robin manner after, e.g. IDF aggregation.

{\bf Making decisions approaches.} This experiment demonstrates comparison between three simple mechanisms for making a global decision about hypothesis acceptance based on local decisions of each node. The first method indicates that hypothesis is rejected if at least one node rejects it. The second technique make the decision based on a majority vote of all nodes decisions. Last method rejects hypothesis only if all nodes rejects it. Figure~\ref{decision_making} shows a dependency between number of nodes and number of input elements needed to detect change for round robin partitioning. As we can see, method that   

\section {Conclusion}
\label {fs-short-conclusion}

In this work we highlighted the problem of efficient testing that data 

\bibliographystyle{ACM-Reference-Format}
\bibliography{../../bibliography/flame-stream}

\end {document}

\endinput
