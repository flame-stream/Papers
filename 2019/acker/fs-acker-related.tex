\label {fs-acker-related}

% Most dependency tracking techniques employed in state-of-the-art stream processing systems are discussed in detail in Section~\ref{existing_solutions}. Naiad~\cite{Murray:2013:NTD:2517349.2522738} uses a kind of similar to \tracker\ mechanism for tracking the progress of iterative computations. Within this method, each data item in a system is assigned with an {\em epoch} and a vector of logical timestamps called {\em loop counter}. Epoch is similar to our notion of {\em global time} concept but provided by an external user. The value on the $i$th position of the loop counter indicates the number of times this element went through the $i$th {\em loop context} (cycle) in a dataflow. Special distributed agents monitor for the items and their timestamps and notify when all elements reach some iteration number or all elements from an epoch are entirely processed. There are three main differences between the mentioned technique and the \tracker. Firstly, \tracker\ relies on the global identifier of an element provided by a system itself. Secondly, the protocol used in Naiad causes the enormous number of extra network messages that quadratically depend on the number of machines even with optimizations~\cite{Murray:2013:NTD:2517349.2522738}. Thirdly, Naiad's method uses counter updates (+1/-1) instead of XORs, hence update messages do not commutate. This fact complicates the implementation (especially distributed) of this method.

% The problems of transactional processing, providing for delivery guarantees, and fault tolerance, are extensively studied in recent years~\cite{Akidau:2013:MFS:2536222.2536229, Carbone:2017:SMA:3137765.3137777, thepaper, Wang:2019:LSF:3341301.3359653}. While state-of-the-art stream processing systems still provide high overhead on regular processing due to fault tolerance protocols, transactional processing, etc., we expect that this area will be studied further. As we mentioned above, a dependency tracking mechanism is an essential part of the solutions to these problems. Hence, \tracker\ can be applied to optimize the existing techniques.