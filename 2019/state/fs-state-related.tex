 %%% fs-state-related - Related work

\label {fs-related-seciton}

Techniques for providing exactly-once~\cite{Carbone:2017:SMA:3137765.3137777, Akidau:2013:MFS:2536222.2536229, Zaharia:2012:DSE:2342763.2342773} are discussed   in details in sections~\ref{fs-intro-seciton} and~\ref{fs-eo-impl}. Many prior works in the field of stream processing do not consider exactly-once maintenance. 
For instance, Aurora~\cite{Abadi:2003:ANM:950481.950485} and Borealis~\cite{abadi2005design} do not provide any guarantees on data at all. Some other systems provide only partial consistency. Apache Storm~\cite{apache:storm} supports message tracking mechanism that prevents the loss of data. 
However, exactly-once semantics is not provided, because duplicates are still possible. Twitter Heron, that was presented as the next generation of Apache Storm~\cite{Kulkarni:2015:THS:2723372.2742788}, does not provide for exactly-once as well. 
Samza~\cite{Noghabi:2017:SSS:3137765.3137770} also implements fairly similar to Storm model and has the same consistency guarantees.

Prior works on stream processing formalization concentrate on operations specification rather than delivery guarantees. Logical foundation for specifying streaming computations is discussed in~\cite{alur2018interfaces}. Declarative algebraic notations for the streaming queries are introduced in~\cite{halle2014formalization}. Another direction in streaming formalization is designing frameworks to define operations semantics~\cite{beck2018lars}.

A comprehensive analysis of different approaches to consistency and fault tolerance in stream procesing is provided in~\cite{Wang:2019:LSF:3341301.3359653}.  The authors observe that, in order to provide consistency guarantees, execting systems have to choose between checkpoints and lineage. The checkpoints tend to be expensive during the normal execution, while lineage usually results in expensive failure recovery. In terms of the above classification our work belongs to the tatter group, as it relies on the total ordering of input items and  re-plays  all items that weren't delivered before failure. 

THe authors of \cite{Wang:2019:LSF:3341301.3359653} also describe an approach that allows non-deterministic processing during normal execution but reproduce exactly same lineage during recovery.  This results in effectively deterministic computation. In contrast with our work, the lineage is produced dynamically during the execution.  This approach still esults in significant computational overheas, but  the lineage may be saved asynchonously and thus produces ònly minor impact on the latency. 
