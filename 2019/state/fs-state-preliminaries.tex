\label{fs-preliminaries}

The concept of {\em consistency} is traditionally expressed in terms of the transactional behavior, known as ACID properties of transactions in the database management systems. These properties ensure that the (database) state is consistent to the degree required by the specified isolation level, providing consistency of processing. However, database systems are rarely used standalone: they interact with data producers (clients) and data consumers (clients or other applications). In such setting, the notion of consistency includes {\em delivery guarantees} as well. {\em At least once}  ensures that input data are not lost, {\em at most once} eliminates duplicate processing, and {\em exactly once} combines both ensuring the absence of input data losses and repeated delivery of results. An implementation of exactly once processing of transactions based on persistent queues is described in~\cite{DBLP:books/mk/WeikumV2002}.

Distributed processing systems are typically shared-nothing and do not require concurrency control. Hence, isolation levels are meaningless for them. In this case, consistency is rather related to system behavior in case of failure. Classical consistent state snapshotting protocols for distributed systems were proposed in~\cite{Chandy:1985:DSD:214451.214456}. However, as typical database systems, modern distributed processing systems interact with data producers and data consumers. Therefore, the notion of consistency in distributed processing systems should also cover delivery guarantees.

Delivery guarantees are typically not considered in batch processing systems, because they always ensure atomicity between reading input data, processing, and delivery of results. In other words, it means that each record within a batch is processed exactly once. This is achieved by consecutive processing of batches, persistent storage for all intermediate results, and re-processing of all suspected failures. A disadvantage of this approach is the impossibility to deliver correct results before the whole data is completely processed.

In contrast, in stream processing systems, consistency guarantees on data are commonly declared in terms of delivery only. A pitfall here is that terms like {\em exactly once} and {\em at least once} hide the fact that system state must also be consistent with input and output in order to achieve correct results. Without regular notations, such definitions can lead to an improper perception that state does not play an important role in consistency enforcement. While the state snapshotting protocol for distributed systems was adopted for streaming systems~\cite{2015arXiv150608603C}, to the best of our knowledge, there are no formal definitions of delivery guarantees and consistent results. In the following section, we propose a formal framework that allows us to define delivery guarantees with aware of the system state, data producers, and data consumers. Using this framework we demonstrate in details how delivery guarantees are implemented in state-of-the-art stream processing systems. We also reveal an approach to exactly once implementation with low performance overhead. 