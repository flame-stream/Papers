\documentclass[sigconf]{acmart}

\usepackage{graphicx}
\usepackage{algorithm} % for algorithms
\usepackage{algpseudocode}
\usepackage{booktabs} % For formal tables
\usepackage{amsthm} % For claims
\usepackage{bbm} % indicator function

\theoremstyle{remark}

\settopmatter{printacmref=false, printccs=true, printfolios=true}
\pagestyle{empty} % removes running headers

\newcommand{\PicScale}{0.5}
\newcommand {\FlameStream} {FlameStream}
\begin{document}

\copyrightyear{2019} 
\acmYear{2019} 
\setcopyright{acmcopyright}
% \acmConference[BeyondMR'18]{Algorithms and Systems for MapReduce and Beyond }{June 15, 2018}{Houston, TX, USA}
% \acmBooktitle{BeyondMR'18: Algorithms and Systems for MapReduce and Beyond , June 15, 2018, Houston, TX, USA}
% \acmPrice{15.00}
% \acmDOI{10.1145/3206333.3209273}
% \acmISBN{978-1-4503-5703-6/18/06}

\title {Distributed Classification of Multi-Labelled Text Streams}

\author{Artem Trofimov,$^ {1}$    Mikhail Shavkunov,$^2$    Sergey Reznick,$^3$     Nikita Sokolov,$^{4}$   Mikhail Yutman,$^2$ \\   Igor E. Kuralenok,$^1$    and  Boris Novikov$^ {2}$ }
\affiliation{%
\institution{$^1$JetBrains Research}
}
\affiliation{%
\institution{$^2$National Research University Higher School of Economics}
}
\affiliation{%
\institution{$^3$ NA}
}
\affiliation{%
\institution{$^4$ ITMO University}
  \city{St. Petersburg} 
  \country{Russia}
}
\email{\string{trofimov9artem, ikuralenok\string}@gmail.com, borisnov@acm.org}

\begin{abstract}

%Задача классификации текстов является хорошо изученной, однако когда данные являются потоковыми начинаются трудности. При такого рода данных, основная сложность является в обеспечении малой задержки между получением очередного текста и его разметкой. Задача обработки текстов на потоковых данных известна, однако существующие решения либо batch processing либо не являются масштабируемыми, также эти решения не обладают гибкостью к постоянно изменяющимся потоку данных. В нашей работе мы представляем систему, где нам удалось создать решение, обеспечивающее минимальную задержку при классификации текстов, при этом, проходить дообучение в реальном времени, сохраняя эту задержку. Мы провели эксперименты, показывающие успешность наших результатов.

Large-scale classification of text streams is an emerging problem that is hard to solve using state-of-the-art data processing systems. Among the main challenges are requirements for: low latency, scalability, online model updating, results predictability, consistency, concept drift handling etc. In this work we propose a framework on top of~\FlameStream\ processing engine that is able to overcome the mentioned problems. We demonstrate the feasibility of the proposed solution with the series of experiments on a news dataset.

% Комментарии БА: нет постановки задачи из-за этого все едет.
% Как такого request-а нет -- при написании слова реквест кажется, что как будто пользователь его делает, но это не так.
% Prior research on processing streaming data has been worked only on batch based solutions or non-scalable ones. However, due to the nature of data is critical to have a scalable solution, moreover, obtaining an answer for a [request] essentially to have low latency. In this paper, we propose text classification on streaming data, where this aspects were considered. Our system is able to continue to fit on the fly without compromising the latency, but increasing answer accuracy instead.

\end {abstract}

% \keywords{Data streams, exactly-once, drifting state, optimistic OOP}

\maketitle

\thispagestyle{empty}

\section {Introduction}
\label {fs-short-intro}

Classification of large text streams is hard, but important task for researchers and practitioners. It has a wide range of applications including detection of emerging news and current user interests, suspicious traffic analysis, spam detection, etc. Popular open-source libraries like sklearn~\cite{sklearn_api} provide a rich set of tools, but they mostly aim at handling static datasets. The lack of scalability across multiple computational units is another limitation of these solutions. There are plenty of works which adapt batch processing systems for text classification~\cite{semberecki2016distributed, svyatkovskiy2016large, baltas2016apache}. Their advantages are fault tolerance, high throughput, and scalability. On the other hand, these systems do not provide low latency that is a strong requirement for most streaming applications.

An immediate idea is to employ a distributed stream processing engine such as Flink~\cite{Carbone:2017:SMA:3137765.3137777} or Storm~\cite{apache:storm}. However, unlike batch engines, stream processing systems have several peculiarities: 

\begin{itemize}
    \item In a general case, failure and recovery are not transparent for a user. The guarantees on data in case of failures are defined in terms of delivery guarantees: {\em at least once} and {\em exactly once}. The choice of a guarantee may affect the correctness of text classification.
    \item Most of streaming systems are inherently non-deterministic. It means that different runs on the same data may produce different results. This feature can influence the classification process as well.
\end{itemize}

In this work, we investigate the applicability of state-of-the-art stream processing systems to the text classification and demonstrate the challenges that a developer can experience. In particular, we discuss how the delivery guarantees and non-deterministic pipelines may affect the results. Possible solutions to the mentioned issues are proposed. 

\section {Processing framework}

Our task is to create a service, which analyzes user messages in real-time, also during the process the service is intended to train on a dataset of incoming labelled texts. The final solution should have a scalablity and a low latency.

\subsection{Distributed stream processing}

In order to solve real life tasks with high-velocity data, the stream processing systems can be used as a solution. The examples of such systems are Apache Flink [?], Google's MillWheel [?], Spark Streaming[?] or Apache Storm[?]. In these systems every piece of data enters and exists the system one by one. Unlike batch-processing systems, this is done without any bufferization during the processing, which provides a low latency for the elements. After entering, the elements are being transformed from one state to another by a map operation. For a scalablity, these systems are running on clusters of computers. To maximize the performance, calculations are evenly distributed among all computers. This method is called sharding technique. Usually each element in streaming systems has a balancing function for determining the shard, where further map transformations will be done with this element. That is, after each transormation every element can be sent to another machine. 

Stream processing systems have several issues to deal with. Most of them have a lack of determinism, which means that the result of computing is not the same between independent launches. Another issue is connected with exactly-once delivery guarantee. This guarantee provides processing each element by exactly one time. Another cases are processing it at most one time and at least one time. The exactly-once guarantee is desirable, because valuable data should not be lost and, on the other hand, multiple processing of the same data causes latency.

In this work, our computations based on FlameStream \cite{kuralenok2018flamestream} distributed model, which provides the following advantages:

\textbf{Determinism.} This makes whole classifying process more predictable as it was discussed in \cite{stonebraker20058} Rule 4. In addition, the concept ensures an opportunity for tests that validate the whole pipeline. FlameStream determinism is considered as lightweight.

\textbf{Exactly-once.} FlameStream process data in exactly-once manner by default, which the same data processing on the Spark systems will be at better performance. The exactly-once guarantee ensures a low latency with almost no overhead.

These advantages allows us to create a classifier with better performance and due to the determinism and exactly-once delivery guarantee provide consistent results.

\subsection{Data flow}

\subsubsection{Computational pipeline}

Similar to other stream processing systems, FlameStream set calculations scheme by a computational graph. This graph is  commonly referred to as logical graph. Logical graph maps into actual physical for processing computations. The graph provides a scheme of data flow and can be presented like this:

\begin{figure}[htbp]
  \centering
  \includegraphics[scale=0.48]{pics/logical-graph}
  \caption{The logical pipeline}
  \label {logical graph}
\end{figure}

The Figure 1 illustrates the principle pipeline. An oriented edge indicates the flow and kind of the data. Every vertex in the graph represents a processing unit such as a single computer or a cluster. The initial text document splits into two computations: the first is calculates term frequencies and the second calculates inverse document frequencies. TF features is computed on the same machine, where the document is arrived, and passes further into TF-IDF vertex. On the other hand, IDF features computations is balanced among all shards. The process of balancing the workload is same as in FlameStream. The range $[0, 2^{32} - 1]$ is divided into equal parts by the number of the shards. Every single shard is responsible for its own range. Hash of each word defines what shard will be used for further IDF computation. In case of words with high frequency such as conjunctions or prepositions, we add to them the salt to the end thereby distributing all the words between the shards evenly. Results from IDF vertex in TF-IDF vertex aggregate with TF features and text classifier receives TF-IDF features of the document. The classifier outputs labelled document and this document returns to the user.

\subsubsection{Dealing with concept drift}

Concept drift is a phenomenon of changing users' interests from time to time, which usually depends on recent events, and results in shifting the distribution of text classes to particular ones. Essentially, this may affect correctness of the pipeline, more specifically, the computing of the IDF features. To overcome the issue, we use windowed IDF calculation: concrete IDF values will be provided based on input within a timed window. For instance, the window can be set to a day or a week. This scheme allows to deal with the sudden changes of the topics.

\subsubsection{Partial fit}

\begin{figure}[htbp]
  \centering
  \includegraphics[scale=0.375]{pics/physical-graph}
  \caption{The physical pipeline}
  \label {physical graph}
\end{figure}

The Figure 2 shows the physical graph of the pipeline for real calculations. Input text can be received by any machine. After that, during IDF computing the corresponding inputs reshuffle for balancing workload and final TF-IDF features aggregate in TF-IDF stage. In final stage text classifier obtains a label for the text and returns to the user.

In addition, the two figures above provide a scheme for a partial fitting. Some of the text documents is labelled and such documents accumulate in the Partial Fit vertex. Additional training is triggered by a special element, which is submitted similarly to the input. This process is analogous to punctuation processing \cite{tucker2003exploiting}. The conditions, when the partial fitting starts, can be chosen arbitrarily, for example, train on each 10000 documents. During this process, the existing classifier's weights are being updated as it is described below. After that, new weights is distributed among the system and the classification continues. 

\subsection{ML model}

The classifier's model can be chosen independently from other computations. In our case, we use Multinomial Logistic Regression. The initial classifier parameters such as weights provided by a pretrain process, which can be executed using sklearn library. We vectorise texts using TfidfTransformer class and the model fits the documents by SGDClassifier class with l1 regularization. The regularization provides us weights as sparse matrix.

The following fitting process can be described in terms of optimization of a cost function. This function in our case is written below:

\begin{center}

$$ J(W) = -\frac{1}{m} \sum \limits_{i = 1}^{m} \sum \limits_{j = 1}^{k} \mathbbm{1}_{\{y^{(i)} == j\}} \cdot \log \frac{\exp\left({W_{j}^Tx^{(i)}}\right) }{\sum \limits_{l = 1}^{k}  \exp\left({W_{l}^Tx^{(i)}}\right) }$$ 
 $$ +  \lambda_1 ||W||_1 + \lambda_2 ||W - W_{prev}||_2 $$

\end{center} 

The number of points in new dataset is denoted as $m$. The point with index $i$ showed as $x^{(i)}$. The number of classes is $k$. New weights are designated as $W$ and previous weights are $W_{prev}$.

The formula provides the goal of the training. The first component is standard softmax function for multiple classes. The second component keeps the l1 regularization of the weights. To use previous history of the classifier weights we apply l2 regularization as the third component. Fitting new points and the consideration of the previous weights ensure better accuracy of the classifier.

We interested in finding such $W$ that minimizes $J(W)$. Taking derivatives, one can show that the gradient for each class component is:

\begin{center}

$$ \nabla_{W_j} \; J(W) = -\frac{1}{m} \sum \limits_{i = 1}^{m} \left[ x^{(i)} \left( \mathbbm{1}_{\{y^{(i)} == j\} } - \frac{\exp\left({W_{j}^Tx^{(i)}}\right)}{\sum \limits_{l = 1}^{k}  \exp\left({W_{l}^Tx^{(i)}}\right)} \right) \right] $$
$$ - \; \lambda_1 sign(W) - \frac{\lambda_2}{2} \left(W - W_{prev} \right), j = [1..k] $$

\end{center} 

This formula is applied to each component during one step of gradient descent. There are a few steps of the gradient is executed and new models' parameters is distributed among shards for further classification.

\section {Experiments}
\label {fs-experiments}

To prove the feasibility of the proposed framework we conducted a series of experiments. We show the efficiency and scalability of the distributed streaming dataflow on top of~\FlameStream\ processing system. Latency and throughput are used as performance metrics. We also demonstrate an achieved accuracy using a simple machine learning model. As a dataset, we used an open corpus of news articles from Russian media resource lenta.ru~\cite{lentaru}. This dataset contains about 700 000 documents, which are labeled by one of 90 different topics. In the experiments, we generated a stream consisted of articles from the dataset sorted by the time of publishing.

\begin{figure}[htbp]
  \centering
  \includegraphics[scale=0.1]{pics/classifier_latencies}
  \caption{Classifier latencies}
  \label {latencies}
\end{figure}

\subsection{Data flow evaluation}

For evaluation, we deployed FlameStream on clusters, containing 2, 4 and 8 Amazon EC2 small instances with 2 GB of RAM and 1 core CPU. Exactly once guarantee was enabled. We measured throughput that is possible to achieve and the corresponding latency for prediction pipeline. We took into consideration only the performance of streaming pipeline without persistent queue. The results are shown in Figures~\ref{latencies} and~\ref{throughput}. As we can see, there is a linear trend in throughput, which proves the scalability of the framework. On the other hand, one can observe, that latency increases moderately and keeps under 25 ms for a median and under 100 ms for 99th percentile.

\begin{figure}[htbp]
  \centering
  \includegraphics[scale=0.25]{pics/classifier_throughput}
  \caption{Classifier throughput}
  \label {throughput}
\end{figure}

\subsection{Classifier evaluation}

In order to be efficiently embedded in the proposed data flow, several properties of the machine learning model are desirable:
\begin{itemize}
     \item small size of the model for storing and updating it in reasonable time and space,
     \item a variability of the model through time.
\end{itemize}

Linear models proved their applicability over time. For this reason, we use Multinomial Logistic Regression as a training method. Model parameters (weights) are denoted as $W$. The training process is the maximization of the following formula in terms of $W$:

$$ logP(W | X) = \frac{1}{|X|} \sum \limits_{(x, y) \in X} \log \frac{e^{{W_y^T \cdot \; x}}}{\sum \limits_{l = 1}^{k}  e^{{W_{l}^T \cdot \; x}}} - \lambda_1 ||W||_1 - \lambda_2 ||W - W_{prev}||_2 $$ 

$X$ denoted as a training dataset and the number of classes is $k$. For changing the model over time, we use weights that computed in the previous step -- $W_{prev}$. At the first step, $W_{prev}$ can be provided by a pre-train process.

The first component is the standard softmax function for multiple classes. The second component keeps the l1 regularization of the weights, and provides sparsity, hence, the model has a small size -- about 1 Mb, which can be stored and updated with low cost. We apply l2 regularization as the third component for using previous weights of the classifier.

We compared two approaches to the performance demonstration of the proposed machine learning model. The first one is training on the complete dataset with applying the formula, where $\lambda_2 = 0$. The approach does not use the history of the weights, therefore we denote it as a {\em static training}. The second one consists in dividing training dataset into relatively small batches and consequent applying MLR with $l2$ regularization to each batch. The latter case demonstrates the behavior of our streaming classification approach. The results are shown in Table~\ref{accuracy}. Streaming approach performance is equal to or better than the first one. It indicates that our framework is able to handle the streaming nature of the news dataset including concept drift. The more formal rationale behind this fact is out of the scope of this paper and relates to our future work.

\begin{table}[htbp]
\begin{tabular}{lc}
Method             & Accuracy \% \\
Static training    & 0.667?       \\
Streaming training & 0.671?         
\end{tabular}
\caption{Accuracy comparison between static and stream training}
\label{accuracy}
\vspace{-7mm}
\end{table}

\section{Related Work}
\label{fs-related}

While the classification of text streams is a well-studied problem~\cite{zhang2008one, tampakas2005}, it is challenging to solve this task at scale. There are plenty of projects that apply batch or micro-batch processing systems to distributed text classification~\cite{semberecki2016distributed, 8029336, baltas2016apache, svyatkovskiy2016large}. However, as it was mentioned above, these methods are not suitable for streaming classification due to low latency requirements. 

General problems regarding machine learning at scale are formulated by the TFX project team~\cite{Baylor:2017:TTP:3097983.3098021}. Among them are continuous training, reliability, reproducibility, etc. It is shown that these problems are hard even if an environment is reliable and fault tolerant. In this work, we argue that these goals are even harder to achieve using state-of-the-art distributed stream processing systems.

Despite the fact that machine learning on top of distributed stream processing is a hot topic~\cite{qiu2016survey}, previous works in this field do not consider issues related to the delivery guarantees and distributed environment~\cite{khumoyun2016real}. A SAMOA framework~\cite{morales2015samoa} provides implementations of several popular algorithms which can be executed on Flink, Storm or Samza but does not take responsibility for reproducible results and correctness in case of failures. Therefore, SAMOA delegates the work on enforcing reproducibility and fault tolerance on a developer. Spark MLlib~\cite{meng2016mllib} is another popular library for adaptation of machine learning pipelines to scalable batching or micro-batching data flows. While Spark Streaming can produce reproducible and reliable results, it is not able to provide latency less than a second~\cite{karimov2018benchmarking, S7530084}.

The problem of non-deterministic and non-reproducible data flows within distributed stream processing is also have been studied in recent years.  As we demonstrated above, the key problem regarding determinism enforcement is races in a data flow. Popular methods to handle such races is {\em in-order} and {\em out-of-order} processing approaches~\cite{Li:2008:OPN:1453856.1453890}. Both these techniques require buffering before each order-sensitive operations until there is a guarantee that all elements are properly ordered. A mechanism that allows a user to control the trade-off between determinism and latency is proposed in~\cite{Doulkeridis:2014:SLA:2628707.2628782}. However, this technique provides for low latency only with a {\em some level} of determinism. An optimistic approach to handle out-of-order elements approach is introduced in~\cite{we2018seim}. Basically, this method mitigates a need for buffering before each operation and reduces the latency of the whole data flow. An adaptation of this approach for machine learning pipelines relates to our future work.

% Techniques for achieving exactly once the delivery guarantee is discussed in plenty of works~\cite{Carbone:2017:SMA:3137765.3137777, Akidau:2013:MFS:2536222.2536229, Zaharia:2012:DSE:2342763.2342773}. However, most of them have a high performance overhead. A promising technique to obtain exactly once was proposed in~\cite{we2018beyondmr}. The key idea behind it is .

\section {Conclusion}
\label {fs-short-conclusion}

In this work, we investigated the suitability of distributed stream processing engines to the problem of text streams classification with the following requirements:

\begin{itemize}
    \item Unbiased by distributed environment: node failures or races do not affect the ultimate result distribution.
    \item Reproducible: if input elements are stored in persistent storage, the same predictions are obtained on each new run.
\end{itemize}

We discussed several pitfalls with a straightforward approach and highlight several limitations regarding the choice of a processing engine and the structure of data flow:

\begin{itemize}
    \item Exactly once is a strong requirement.
    \item Embedding of time-consuming train process in the prediction pipeline leads to significant latency spikes.
\end{itemize}

As future work, we plan to implement a text classification framework that satisfies the proposed requirements and provides for low latency.


\bibliographystyle{ACM-Reference-Format}
\bibliography{../../bibliography/flame-stream}

\end {document}

\endinput
