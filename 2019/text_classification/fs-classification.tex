\documentclass[sigconf]{acmart}

\usepackage{graphicx}
\usepackage{algorithm} % for algorithms
\usepackage{algpseudocode}
\usepackage{booktabs} % For formal tables
\usepackage{amsthm} % For claims

\theoremstyle{remark}

\settopmatter{printacmref=false, printccs=true, printfolios=true}
\pagestyle{empty} % removes running headers

\newcommand{\PicScale}{0.5}
\newcommand {\FlameStream} {FlameStream}
\begin{document}

\copyrightyear{2019} 
\acmYear{2019} 
\setcopyright{acmcopyright}
% \acmConference[BeyondMR'18]{Algorithms and Systems for MapReduce and Beyond }{June 15, 2018}{Houston, TX, USA}
% \acmBooktitle{BeyondMR'18: Algorithms and Systems for MapReduce and Beyond , June 15, 2018, Houston, TX, USA}
% \acmPrice{15.00}
% \acmDOI{10.1145/3206333.3209273}
% \acmISBN{978-1-4503-5703-6/18/06}

\title {Distributed Classification of Multi-Labelled Text Streams}

% \author{  Igor E. Kuralenok,$^1$     Artem Trofimov,$^ {1,2}$    Nikita Marshalkin,$^ {1,2}$   and  Boris Novikov$^ {1,2}$ }
% \affiliation{%
% \institution{$^1$JetBrains Research}
%   \city{St. Petersburg} 
%   \country{Russia}
% }
% \affiliation{%
% \institution{$^2$Saint Petersburg State University}
%   \city{St. Petersburg} 
%   \country{Russia}
% }
% \email{\string{ikuralenok, trofimov9artem, marnikitta\string}@gmail.com, borisnov@acm.org}

\begin{abstract}

%Задача классификации текстов является хорошо изученной, однако когда данные являются потоковыми начинаются трудности. При такого рода данных, основная сложность является в обеспечении малой задержки между получением очередного текста и его разметкой. Задача обработки текстов на потоковых данных известна, однако существующие решения либо batch processing либо не являются масштабируемыми, также эти решения не обладают гибкостью к постоянно изменяющимся потоку данных. В нашей работе мы представляем систему, где нам удалось создать решение, обеспечивающее минимальную задержку при классификации текстов, при этом, проходить дообучение в реальном времени, сохраняя эту задержку. Мы провели эксперименты, показывающие успешность наших результатов.

Multi-labelled text classification problem is well-studied one, and even can be solved on a large amount of static data using batch processing systems. However, classifying of high-velocity text streams is a challenging task. The main difficulty here is to provide a low latency between obtaining text and its labeling. There are solutions of processing texts on streams, but they are non-scalable and mostly pay attention to problems connected with a nature of streaming data, e.g. concept shift. In this work, we introduce a computational scheme, which considers the aspects of previous works, but provide low latency and scalability, while being able to train in real-time. Our experiment results demonstrate the benefits of our approach.

% Комментарии БА: нет постановки задачи из-за этого все едет.
% Как такого request-а нет -- при написании слова реквест кажется, что как будто пользователь его делает, но это не так.
% Prior research on processing streaming data has been worked only on batch based solutions or non-scalable ones. However, due to the nature of data is critical to have a scalable solution, moreover, obtaining an answer for a [request] essentially to have low latency. In this paper, we propose text classification on streaming data, where this aspects were considered. Our system is able to continue to fit on the fly without compromising the latency, but increasing answer accuracy instead.

\end {abstract}

% \keywords{Data streams, exactly-once, drifting state, optimistic OOP}

\maketitle

\thispagestyle{empty}

\section {Introduction}
\label {fs-short-intro}

Classification of large text streams is hard, but important task for researchers and practitioners. It has a wide range of applications including detection of emerging news and current user interests, suspicious traffic analysis, spam detection, etc. Popular open-source libraries like sklearn~\cite{sklearn_api} provide a rich set of tools, but they mostly aim at handling static datasets. The lack of scalability across multiple computational units is another limitation of these solutions. There are plenty of works which adapt batch processing systems for text classification~\cite{semberecki2016distributed, svyatkovskiy2016large, baltas2016apache}. Their advantages are fault tolerance, high throughput, and scalability. On the other hand, these systems do not provide low latency that is a strong requirement for most streaming applications.

An immediate idea is to employ a distributed stream processing engine such as Flink~\cite{Carbone:2017:SMA:3137765.3137777} or Storm~\cite{apache:storm}. However, unlike batch engines, stream processing systems have several peculiarities: 

\begin{itemize}
    \item In a general case, failure and recovery are not transparent for a user. The guarantees on data in case of failures are defined in terms of delivery guarantees: {\em at least once} and {\em exactly once}. The choice of a guarantee may affect the correctness of text classification.
    \item Most of streaming systems are inherently non-deterministic. It means that different runs on the same data may produce different results. This feature can influence the classification process as well.
\end{itemize}

In this work, we investigate the applicability of state-of-the-art stream processing systems to the text classification and demonstrate the challenges that a developer can experience. In particular, we discuss how the delivery guarantees and non-deterministic pipelines may affect the results. Possible solutions to the mentioned issues are proposed. 

\section {Experiments}
\label {fs-experiments}

To prove the feasibility of the proposed framework we conducted a series of experiments. We show the efficiency and scalability of the distributed streaming dataflow on top of~\FlameStream\ processing system. Latency and throughput are used as performance metrics. We also demonstrate an achieved accuracy using a simple machine learning model. As a dataset, we used an open corpus of news articles from Russian media resource lenta.ru~\cite{lentaru}. This dataset contains about 700 000 documents, which are labeled by one of 90 different topics. In the experiments, we generated a stream consisted of articles from the dataset sorted by the time of publishing.

\begin{figure}[htbp]
  \centering
  \includegraphics[scale=0.1]{pics/classifier_latencies}
  \caption{Classifier latencies}
  \label {latencies}
\end{figure}

\subsection{Data flow evaluation}

For evaluation, we deployed FlameStream on clusters, containing 2, 4 and 8 Amazon EC2 small instances with 2 GB of RAM and 1 core CPU. Exactly once guarantee was enabled. We measured throughput that is possible to achieve and the corresponding latency for prediction pipeline. We took into consideration only the performance of streaming pipeline without persistent queue. The results are shown in Figures~\ref{latencies} and~\ref{throughput}. As we can see, there is a linear trend in throughput, which proves the scalability of the framework. On the other hand, one can observe, that latency increases moderately and keeps under 25 ms for a median and under 100 ms for 99th percentile.

\begin{figure}[htbp]
  \centering
  \includegraphics[scale=0.25]{pics/classifier_throughput}
  \caption{Classifier throughput}
  \label {throughput}
\end{figure}

\subsection{Classifier evaluation}

In order to be efficiently embedded in the proposed data flow, several properties of the machine learning model are desirable:
\begin{itemize}
     \item small size of the model for storing and updating it in reasonable time and space,
     \item a variability of the model through time.
\end{itemize}

Linear models proved their applicability over time. For this reason, we use Multinomial Logistic Regression as a training method. Model parameters (weights) are denoted as $W$. The training process is the maximization of the following formula in terms of $W$:

$$ logP(W | X) = \frac{1}{|X|} \sum \limits_{(x, y) \in X} \log \frac{e^{{W_y^T \cdot \; x}}}{\sum \limits_{l = 1}^{k}  e^{{W_{l}^T \cdot \; x}}} - \lambda_1 ||W||_1 - \lambda_2 ||W - W_{prev}||_2 $$ 

$X$ denoted as a training dataset and the number of classes is $k$. For changing the model over time, we use weights that computed in the previous step -- $W_{prev}$. At the first step, $W_{prev}$ can be provided by a pre-train process.

The first component is the standard softmax function for multiple classes. The second component keeps the l1 regularization of the weights, and provides sparsity, hence, the model has a small size -- about 1 Mb, which can be stored and updated with low cost. We apply l2 regularization as the third component for using previous weights of the classifier.

We compared two approaches to the performance demonstration of the proposed machine learning model. The first one is training on the complete dataset with applying the formula, where $\lambda_2 = 0$. The approach does not use the history of the weights, therefore we denote it as a {\em static training}. The second one consists in dividing training dataset into relatively small batches and consequent applying MLR with $l2$ regularization to each batch. The latter case demonstrates the behavior of our streaming classification approach. The results are shown in Table~\ref{accuracy}. Streaming approach performance is equal to or better than the first one. It indicates that our framework is able to handle the streaming nature of the news dataset including concept drift. The more formal rationale behind this fact is out of the scope of this paper and relates to our future work.

\begin{table}[htbp]
\begin{tabular}{lc}
Method             & Accuracy \% \\
Static training    & 0.667?       \\
Streaming training & 0.671?         
\end{tabular}
\caption{Accuracy comparison between static and stream training}
\label{accuracy}
\vspace{-7mm}
\end{table}

\section {Conclusion}
\label {fs-short-conclusion}

In this work, we investigated the suitability of distributed stream processing engines to the problem of text streams classification with the following requirements:

\begin{itemize}
    \item Unbiased by distributed environment: node failures or races do not affect the ultimate result distribution.
    \item Reproducible: if input elements are stored in persistent storage, the same predictions are obtained on each new run.
\end{itemize}

We discussed several pitfalls with a straightforward approach and highlight several limitations regarding the choice of a processing engine and the structure of data flow:

\begin{itemize}
    \item Exactly once is a strong requirement.
    \item Embedding of time-consuming train process in the prediction pipeline leads to significant latency spikes.
\end{itemize}

As future work, we plan to implement a text classification framework that satisfies the proposed requirements and provides for low latency.


\bibliographystyle{ACM-Reference-Format}
\bibliography{../../bibliography/flame-stream}

\end {document}

\endinput
