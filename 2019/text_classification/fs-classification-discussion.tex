\label {fs-discussion}

In this section, we discuss and demonstrate the main pitfalls which arise with the na\"ive data flow. The purpose of our evaluation is not to compare various machine learning models, but to investigate the applicability of stream processing systems as a {\em tool} for building text classification pipelines. We concentrate on a question on how distributed stream processing features may affect reproducibility and reliability of classification results.

For experiments, we used an implementation of the proposed data flow on top of Apache Flink. The experiments were performed on a single 4 core CPU 8GB RAM machine with 2 Flink workers. Such setting is chosen to show that issues can be faced even in a deployment with a limited asynchrony. As a dataset, we used an open corpus of news articles from Russian media resource lenta.ru~\cite{lentaru}. This dataset contains documents, which are labeled by one of 90 different topics such as {\em sport}, {\em politics}, {\em science}, etc. In the experiments, we generated a stream consisted of articles from the dataset.

We used multinomial logistic regression model as a classifier. We considered the multi-label classification problem: each news article can be labeled by multiple topics. For instance, text about novel research in sports food may be denoted as 70\% about sport, 20\% about science and 10\% about food. 

\subsection{Reproducibility}

Users of popular open-source machine learning libraries like sklearn used to obtain results which are unbiased by an execution environment. Migration to batch processing systems like Hadoop or Spark usually do not cause many extra issues, because these engines mostly hide effects of asynchronous processing from a user and provide deterministic results. On the contrary, most distributed stream processing engines are non-deterministic due to processing model aiming to low latency. Therefore, the main challenge regarding reproducibility of streaming machine learning pipelines is to achieve predictable results, while keeping low processing latency. 

\subsubsection{Online training}
An issue regarding the na\"ive pipeline is that the training process may be time-consuming. If training and prediction processes run consecutively, there will be significant latency spikes, e.g. if a training process lasts for several minutes, then spikes may be 10 000 times greater than the latency for prediction. However, without synchronization, there will be no reproducible correspondence between texts and applied model. It is almost impossible to achieve the same results within a new run on the same data because the training time becomes a hidden parameter that influences output. For instance, assume that we make two runs. On the first run model update consumes 70 seconds, but on the second run 75 seconds due to extra CPU load. If training and predicting are not synchronized, more unlabeled input elements are processed by an outdated model in the second case, so the distribution of news topics may be different between these two runs. We propose two solutions for the issues in question:

\begin{itemize}
    \item Use online learning algorithms. In this case, model updating is smooth and its synchronization with training does not cause latency spikes. We discuss this approach in details in the next section.
    \item Consider model parameters as special input elements that are stored with other input elements in a persistent queue, e.g. using Kafka~\cite{kreps2011kafka}. To reproduce results, there is just a need to replay elements from this queue.
\end{itemize}

\subsubsection{Races in the data flow}
The issue is that there is a race between documents in the data flow before IDF update. Hence, IDF features of the words in articles may vary from run to run. For example, let us consider two documents stream: the first one contains word {\em cat}, while the second consists of {\em cat} and {\em dog}. If the first document is processed before the second, IDF for the word {\em cat} within TF-IDF features of the second document will be 1, while otherwise, it will be 0. This issue is more sophisticated than the previous one, but can also make results irreproducible. 

To show how this behavior affects text classification results we made 10 runs on a stream consisted of 10 000 news articles. We compared the most probable 1,2,3,4,5 obtained labels for the same documents between runs. Our comparison was order-sensitive: if top 2 labels for the document on the first run is [sport (50\%), science (20\%)], but on the second run [science (50\%), sport (20\%)], then we denoted these results as varied. 

Table~\ref{race_table} demonstrates the results of the experiment. As we can see, approximately 56 out of 10 000 articles obtained distinct top 1 label. With the growth of considering the top, the percent of varied results significantly increases: 1270 articles achieved different top 5 labels on the average. These results indicate that the issue may influence the classification results and makes them hardly reproducible between independent runs. The solution to this problem is to determine the order of input documents and preserve this order before IDF computation. An implementation of this fix is specific for a concrete streaming engine.

\begin{table}[htbp]
\caption{Effects of races in the data flow}
\begin{threeparttable}
\begin{tabular}{lcl}
Top labels for comparison    & \% of varied results & std    \\
\hline
1   &   0.56    &   0.06    \\
2   &   2.38    &   0.14    \\
3   &   5.27    &   0.22    \\
4   &   9.27    &   0.35    \\
5   &   13.7    &   0.53    \\
\end{tabular}
\end{threeparttable}
\label{race_table}
\end{table}

\subsection{Fault tolerance}

As we demonstrated above, if machine learning pipeline is run on multiple computational units, there can be issues with reproducibility. Unfortunately, it is not the only challenge: computational nodes and the network may fail and potentially cause shifted or even incorrect results. Hence, in large-scale production deployments it is important to ensure that nodes and network failures do not {\em invisibly} influence outcome.

In stream processing systems guarantees on data in case of failure are typically described in terms of {\em delivery guarantees}: at least once and exactly once. If a system provides exactly once, it is guaranteed that a streaming element is applied to data flow operators and released exactly one time. With at least once, it is ensured that an element is not lost, but it can influence operator states and be delivered to end-user multiple times. At least once may be preferred, because it typically has lower performance overhead. However, in the following experiments, we demonstrate that at least once guarantee may not be acceptable in some cases.  




