\documentclass[sigconf]{acmart}

\usepackage{graphicx}
\usepackage{algorithm} % for algorithms
\usepackage{algpseudocode}
\usepackage{booktabs} % For formal tables
\usepackage{amsthm} % For claims
\usepackage{bbm} % indicator function
\usepackage{acmcopyright}

% listings
\usepackage{xcolor,listings}
\usepackage{textcomp}
\lstset{upquote=true}

% plots
\usepackage{pgfplots}

% table
\usepackage[flushleft]{threeparttable} % http://ctan.org/pkg/threeparttable
\usepackage{booktabs,caption}

\theoremstyle{remark}

\settopmatter{printacmref=false, printccs=true, printfolios=true}
\pagestyle{empty} % removes running headers

\newcommand{\PicScale}{0.5}
\newcommand {\FlameStream} {FlameStream}
\begin{document}

% \copyrightyear{2019}
% \acmYear{2019}
% \setcopyright{rightsretained}
% \acmConference[DEBS '19]{DEBS '19: The 13th ACM International Conference on Distributed and Event-based Systems}{June 24--28, 2019}{Darmstadt, Germany}
% \acmBooktitle{DEBS '19: The 13th ACM International Conference on Distributed and Event-based Systems (DEBS '19), June 24--28, 2019, Darmstadt, Germany}\acmDOI{10.1145/3328905.3332514}
% \acmISBN{978-1-4503-6794-3/19/06}

\title{Calco: a contract-based approach to \\declaratively specify distributed dataflows}

% \author{Alexander Chernokoz, Darya Sharkova, Artem Trofimov, Nikita Sokolov, Ekaterina Gorshkova, and Boris Novikov}
% \affiliation{%
% \institution{$^5$ ITMO University}
%   \city{St. Petersburg}
%   \country{Russia}
% }
% \affiliation{%
% \institution{$^1$Higher School of Economics}
% }
% \author{Artem Trofimov,$^ {1,2}$    Mikhail Shavkunov,$^3$    Sergey Reznick,$^4$     Nikita Sokolov,$^{5}$   Mikhail Yutman,$^3$ \\   Igor E. Kuralenok,$^1$    and  Boris Novikov$^ {3}$ }
% \affiliation{%
% \institution{$^1$JetBrains Research}
% }
% \affiliation{%
% \institution{$^2$Saint Petersburg State University}
% }
% \affiliation{%
% \institution{$^3$National Research University Higher School of Economics}
% }
% \affiliation{%
% \institution{$^4$ Kofax}
% }
% \affiliation{%
% \institution{$^5$ ITMO University}
%   \city{St. Petersburg}
%   \country{Russia}
% }
% \email{\string{trofimov9artem, mv.shavkunov, sergey.reznick, faucct, myutman, ikuralenok\string}@gmail.com, borisnov@acm.org}

% План статьи:

%  1 Аннотация + Интро - 2 колонки
%  a Распределенная обработка и ее виды. Есть MapReduce и батчи, есть потоки.
%  b Исторически, пайплайны распределенной обработки задавали графами исполнения (переход от Hadoop к Spark). Графы исполнения популярны, т.к. их можно задать кодом и версионировать как код, у них есть переиспользуемые единицы (вершины), их можно визуализировать. Однако, графы состоят из произвольных пользовательских операций, над которыми нет никакой заданной алгебры. Следовательно, графы можно оптимизировать очень ограниченно - если мы перестроим граф, то у нас нет возможности понять, остался ли он корректным.
%  c Еще один способ задания вычислений SQL - подход, основанный на реляционной алгебре, в котором через запрос задается результат, который необходимо получить. Пользуясь правилами реляционной алгебры можно получать гарантированно эквиваллентные графы исполнения из одного запроса и из них выбирать оптимальный. Однако, у SQL есть проблемы - через него нельзя выразить произвольное MapReduce преобразование, а user-defined функции не поддаются оптимизации, т.к. на них не распространяются правила реляционной алгебры.
%  d Наш подход - основан на задании алгебры над пользовательскими операциями через контракты. Таким образом, мы можем переставлять произвольные операции в соответсвии с правилами, которые задали пользователи. В частности, через наш подход выражается SQL - с соответствующими правилами в контрактах (пример?).
%  e Большое отличие распределенных от баз данных - обычно запросы выполняются долго и меняются редко, а данные меняются часто. Наш подход позволяет делать совместную оптимизацию запросов, заданных SQL и произвольными пользовательскими операциями через сведение к контрактам. Таким образом, открывается возможность переиспользовать уже запущенные операции, используя их в различных задачах.
%  2 Пример совместной оптимизации - 1 колонка
%  3 Описание контрактов - 1.5 колонки
%  4 Пример совместной оптимизации на контрактах - 1.5 колонки
%  5 Ссылки, вывод и related work + discussion & fw - 1.5 колонки

\begin{abstract}

    There are two general ways to define computations: imperative and declarative.
    The first one is more transparent for programmers, while the second one is more suitable for complex optimizations.
    One declarative definition can have multiple implementations, which makes it possible to choose the most optimal one.
    Accordingly, distributed dataflow can be specified in two ways: defining concrete execution graph and SQL.
    A concrete graph does not capture high-level information about the problem it solves, so a distributed data processing system cannot permute operations for performance purposes.
    Hence, graph optimization is generally a programmer concern, but real-world graphs can be big enough, that makes it hard to optimize them manually.
    Moreover, most of the necessary information for the graph cost evaluation is available only in runtime.
    SQL is popular for data analytics, and its optimization is well-researched.
    However, it is inconvenient or sometimes impossible to use SQL for general data management tasks, such as ETL, machine learning pipelines, etc.

    In this work, we introduce a novel approach to specify distributed dataflows.
    Our method is based on declarative specifications of user-defined operations that we call contracts.
    Such specifications allow us to automatically generate execution graphs with the needed semantics to choose the most optimal one.
    An arbitrary operation or a dataflow part can be described by contracts, so this approach combines the transparency and flexibility of the imperative approach and optimization possibilities of the declarative one.
    We implement a prototype and demonstrate automatic graphs generation on a real-world dataflow (TODO).
    We also outline the challenges that we face regarding the optimization problem.

\end{abstract}

% \begin{CCSXML}
% \begin{CCSXML}
% \begin{CCSXML}
% <ccs2012>
% <concept>
% <concept_id>10002951.10002952.10002953.10010820.10003208</concept_id>
% <concept_desc>Information systems~Data streams</concept_desc>
% <concept_significance>500</concept_significance>
% </concept>
% <concept>
% <concept_id>10002951.10003317.10003347.10003356</concept_id>
% <concept_desc>Information systems~Clustering and classification</concept_desc>
% <concept_significance>500</concept_significance>
% </concept>
% <concept>
% <concept_id>10002951.10003227.10003351.10003446</concept_id>
% <concept_desc>Information systems~Data stream mining</concept_desc>
% <concept_significance>300</concept_significance>
% </concept>
% </ccs2012>
% \end{CCSXML}

% \ccsdesc[500]{Information systems~Data streams}
% \ccsdesc[500]{Information systems~Clustering and classification}
% \ccsdesc[300]{Information systems~Data stream mining}

% \keywords{Data streams, text classification, reproducibility, exactly once}

\maketitle

\thispagestyle{empty}

\section{Introduction}
TODO cross domain optimization

TODO SQL is just not enough

TODO intersecting queries optimization (cross-query optimization)

TODO runtime information is needed for optimization, so it should be done automatically

TODO too many concrete graphs for the simple examples to code them manually

TODO


\section{Running example}
To consider possibilities and limitations of the our approach later we introduce an example machine learning streaming problem.

Let's suppose that we have an application that consists of the frontend and backend sides.
They both log user's queries and send logs to the log service.
For simplicity, let's assume that every user query produces one log entity from the frontend and one log entity from the backend, and they have equal and unique query id.

The log entities have the following structure (ts stands for the timestamp):

\begin{tabular}{|l|llll|}
    \hline
    \textbf{frontend} & version & queryId & userId & ts \\
    \hline
\end{tabular}

\vspace{0.1em}

\begin{tabular}{|l|lllll|}
    \hline
    \textbf{backend} & id & queryId & userId & ts & payload \\
    \hline
\end{tabular}

We want to calculate statistics about the time that logged users with new frontend client versions wait for the application answer per user session.
To evaluate when the session ends we will use machine learning model that requires frontend features and users features.

One of the possible solutions is represented on the image TODO. % TODO image ref
Obviously, this solution is not optimal.
Filters are applied to streams too late and number of partitions potentially can be reduced.
To be able to fix it safely we need to know about the requirements and impact of the each operation.

\begin{itemize}
    \item \textbf{joinByQueryId}: Requires queryId field in the elements of both streams.
    \item \textbf{addUsersFeatures}: Requires partition by userId. Users should not be filtered before to get valid statistics. Adds userFeatures field to elements.
    \item \textbf{addFrontFeatures}: Requires only new front versions, adds frontFeatures field to elements in the stream.
    \item \textbf{filterNewFronts}: Leaves only elements with new front versions.
    \item \textbf{modelInference}: Requires front and users features. Sets trigger that signals when to emit data from aggregation.
    \item \textbf{filterAuthorizedUsers}: Requires users features. Filters users that are authorized.
    \item \textbf{stats}: Requires authorized users from new front versions. Also needs session trigger to be set. Sends accumulated statistics to the dashboard.
\end{itemize}

% TODO SQL proof
We see here complex pipeline with stateful operations and streaming triggers management.
It is not possible to describe itin SQL because of it.
Moreover, we see non-trivial operation requirements here, so concrete execution graph specifying will not provide space for optimizations.

% TODO too much lets
Let's consider another pipeline that satisfies requirements: TODO. % TODO image ref
We can see that filters here are maximally close to data sources.
And after permutation of stateful operations we have one less data partition that is rather expensive operation.

In section TODO % TODO number of section
we specify this task using contract-based approach and see, how it helps in optimizations.

% TODO insert properly
% TODO ops naming
\begin{figure}
    \label{fig:running-example-suboptimal}
    \label{fig:running-example-optimal}
    \includegraphics[width=\linewidth]{images/debs-calco-example}
\end{figure}


\section{Calco overview}
As we discussed earlier, we need information about semantics of the each operation to be able to permute them safely.
Also we want to use custom operations in the graph, because defining concrete set of operations with known permutation rules is not enough flexible approach.

Thus we propose to specify semantics of custom operations manually.
It is similar idea to the axiomatic semantics of the programming languages (TODO ref), based on the Hoare triples.
Hoare triple consists of the first predicate that should be satisfied for the program context before the statement execution, of statement, which semantics we describe, and of the second predicate that should be satisfied for the program context after the statement execution.
We propose to annotate each custom operation with the input contract, that the input data stream should satisfy, and with the output contract that describes, how this operation changes the input data stream.
So we can specify the computation as a set of annotated nodes, using this representation we can generate all concrete graphs with such nodes, which contracts are satisfied (hence, all operations work properly).
Such novel specification approach we call CGraph.

Let's consider CGraph and contracts in details.

\subsection{CGraph}

Let us define an environment as a set of nodes, annotated with input and output contracts (InCont and OutCont).
This set consists of the data sources, annotated with OutCont, one arity transformations (with one InCont and OutCont) and two arity transformations (with two InCont and OutCont).
Here we limit possible arity of operations for simplicity.

\begin{lstlisting}[language=Haskell]
type CSource = OutCont
type CTfm1 = (InCont, OutCont)
type CTfm2 = (InCont, InCont, OutCont)

data Env = Env
  { sources :: Map NodeName CSource
  , tfms1   :: Map NodeName CTfm1
  , tfms2   :: Map NodeName CTfm2 }
\end{lstlisting}

Graph gets data from the data sources, transforms it in transformation nodes.
But some of the transformation nodes do particular side effects that form result of the graph execution (writes data to the some storage, displays some statistics on the dashboard, etc.).
Such nodes should exist in all concrete execution graphs that correspond to the given CGraph.
Let's call the set of the nodes with the such side effects as a graph semantics.
It will let us to have useless nodes in environment (that are not necessary for graph to have the needed semantics) and reduce the total number of concrete graphs to enumerate.

\begin{lstlisting}[language=Haskell]
type Semantics = Set NodeNames
\end{lstlisting}

And now we are ready to define CGraph.
It is simply a pair of environment and semantics.

\begin{lstlisting}[language=Haskell]
type CGraph = (Env, Semantics)
\end{lstlisting}

\subsection{Input contracts}

Let's suppose that each element in the data stream is a mapping from attribute names to data (like a row in relational database).
So the data stream can be described by two sorts of information: attributes that every element has and the properties of the data.
We can define stream state as follows:

\begin{lstlisting}[language=Haskell]
type Attr = String
type Prop = String

data State = State
  { attrs :: Set Attr
  , props :: Set Prop }

empty :: State
empty = State { attrs = Set.empty
              , props = Set.empty }

union :: State -> State -> State
union s1 s2 = State
  { attrs s1 `Set.union` attrs s2
  , props s1 `Set.union` props s2 }
\end{lstlisting}

Input contract can be defined as tuple of three sets:
\begin{enumerate}
    \item set of attributes that are required in the input stream,
    \item set of properties that are required in the input stream,
    \item set of properties that are prohibited in the input stream.
\end{enumerate}

\begin{lstlisting}[language=Haskell]
data InCont = InCont
  { attrsI  :: Set Attr
  , propsI  :: Set Prop
  , propsI' :: Set Prop }
\end{lstlisting}

Operation's input stream state can be matched with the input contract to check if it satisfies that contract.

\begin{lstlisting}[language=Haskell]
match :: State -> InCont -> Bool
match s c =
     attrsI c `Set.isSubsetOf` attrs s
  && propsI c `Set.isSubsetOf` props s
  && propsI' c `Set.disjoint`  props s
\end{lstlisting}

\subsection{Output contracts}

Output contracts can be defined as a tuple of the three sets and one boolean:
\begin{enumerate}
    \item set of attributes to be added,
    \item boolean that is true if an operation is a projection (it removes all attributes that exist in the incoming stream),
    \item set of properties to be added,
    \item set of properties to be deleted.
\end{enumerate}

\begin{lstlisting}[language=Haskell]
data OutCont = OutCont
  { attrsO  :: Set Attr
  , isProj  :: Bool
  , propsO  :: Set Prop
  , propsO' :: Set Prop }
\end{lstlisting}

To get the output data stream state of the node, input data stream state should be updated with the output contract of the operation.

\begin{lstlisting}[language=Haskell]
update :: State -> OutCont -> State
update s c = State
  { attrs = Set.union
      (attrsO c)
      (if isProj c then Set.empty
                   else attrs s)
  , props = Set.union
      (propsO c)
      (props s `Set.difference` propsO' c) }
\end{lstlisting}

To get the output data stream state of the data source, empty state should be updated with the source's output contract.

\begin{lstlisting}[language=Haskell]
sourceOut :: OutCont -> State
sourceOut = update State.empty
\end{lstlisting}

To get the output data stream state of the one-arity transformation, incoming data stream state should be updated with the transformation's output contract.

\begin{lstlisting}[language=Haskell]
tfm1Out :: State -> OutCont -> State
tfm1Out = update
\end{lstlisting}

To get the output data stream state of the two-arity transformation, union of the incoming data streams states should be updated with the transformation's output contract.

\begin{lstlisting}[language=Haskell]
tfm2Out :: State -> State
        -> OutCont -> State
tfm2Out s1 s2 = update (s1 `State.union` s2)
\end{lstlisting}

\subsection{Work scheme}

All concrete graphs that correspond to the given CGraph can be generated.
So having the cost function we can choose the most optimal one and run it.
But having the execution statistics from the runtime, cost function can be much more precise.
Thus we should continuously recompute the concrete graphs costs using the updating statistics.
When the running graph becomes enough less optimal then another graph to compensate the reconfiguration cost, running graph should be dynamically reconfigured.

\subsection{Prototype}

We have implemented a prototype in Haskell.
It consists of:
\begin{enumerate}
    \item contracts definition and CGraph definition,
    \item function that checks if a concrete graph corresponds to the given CGraph,
    \item function that generates all concrete graph that correspond to the given CGraph (runs rather fast because of early enumeration branch truncation),
    \item set of basic operations that emulate such set of the Apache Beam framework,
    \item evaluation function that runs the concrete graph.
\end{enumerate}

TODO ref https://github.com/flame-stream/halco


\section{Example evaluation}
Semantics for the running example consists of the "fraudMetric" and "participants" nodes.
The environment has three sources, two one-arity operations, and two two-arity operations.

We discussed earlier an abstract definition of the contracts using the type classes and a particular implementation of them. 
However, using this implementation we are not able to describe the required input data state of the fraud metric, which works on the different alternatives of the input data.
So we need to create a new implementation of input contract.
We just write an instance of the InCont for the list of the InContImpl with the needed match function:

\begin{lstlisting}[language=Haskell]
newtype ListImpl = ListImpl [InContImpl]

instance InCont State ListImpl where
  match :: State -> ListImpl -> Bool
  match s (ListImpl is) = any (match s) is
\end{lstlisting}

For example let us consider contracts of the bid-auction join and the fraud metric.
Join requires "bid.auction" and "auction.id" attributes and sets "joinedBA" property:
\begin{lstlisting}[language=Haskell]
( InContImpl (Set.singleton "bid.auction")
             Set.empty Set.empty
, InContImpl (Set.singleton "auction.id")
             Set.empty Set.empty
, OutContImpl Set.empty
              (Set.singleton "joinedBA") )
\end{lstlisting}

Fraud metric has one input contract with four alternatives (list of four InContImpl):
only bid attributes
or joined bid and auction attributes
or joined bid and person attributes
or all attributes.

We implemented a graphs generation prototype in Haskell.
Our prototype generates all possible 6 execution graphs corresponding to the CGraph of the running example.



TODO SQL comparison


\section{Discussion and future work}
In this section we discuss successes and concerns of the proposed contract-based approach.

\subsection{Contracts}

As we saw, semantics of the custom operations can be naturally described by contracts.
However it may be inconvenient to write all contracts as is.
For example, if some operations need incoming streams not to be filtered, adding new filter property makes programmer to add it to all this operations manually.
There are some ideas that should make defining CGraph more handy, here are some of them:
\begin{itemize}
    \item properties should be able to be grouped to use them together (prohibit all filter properties as one, for example),
    \item some kind of properties may be prohibited by default,
    \item interactive environment that hints, which attributes and properties are not yet satisfied for the nodes that form the graph semantics.
\end{itemize}

\subsection{Work scheme and optimization}

Before having runtime statistics, a reasonable concrete graph should be chosen.
To do that we need an information about the output stream cardinalities of nodes.
Seems like this information should be manually provided by the programmer.

There are no production-ready frameworks for distributed data processing that gather needed statistics for accurate concrete graph cost evaluation.
Also none of them can dynamically reconfigure graph.

Finally, it is needed to precisely compute cost function using the runtime statistics and also estimate the cost of the graph dynamic reconfiguration.

\subsection{General implementation}

We plan to develop libraries to be able to specify computations using CGraph on Python and Java.


\section{Related work}
We review works which propose declarative approaches to define distributed computations and use it to make optimizations.
Most of such works are dedicated to the specific computational areas.

There are

For example: machine learning algorithms and graph queries processing.



Works dedicated to distributed machine learning algorithms implementation provide possibilities to effectively compute matrix operation TODO references.

TODO machine learning \\
TODO graph queries processing \\
TODO paper references \\
TODO


\section{Conclusion}
We presented a novel approach to specify distributed dataflows.
It is based on the contracts, preconditions and postconditions of the graph operations.
We considered a small machine leaning streaming pipeline example, discussed ways of its optimization and saw that our approach defines the most optimal execution graph too.
% TODO We demonstrated the prototype?
Also we outlined optimization challenges that we faced.
Finally, we showed ideas for the future work enhancements.

We proposed to specify dataflow as CGraph (contracted graph).
CGraph is a pair of environment and semantics, where environment maps nodes to their contracts, and semantics is a set of nodes that produce dataflow target side-effects.
We provide an algorithm that generates all concrete execution graphs that satisfy contracts and semantics.

We discussed the example that SQL implementation seems to be unnatural.
And the only possible way to specify it is execution graph, that represents rather imperative approach then declarative, hence do not provide enough space for optimizations.
We specified the example problem by CGraph and pointed out that the most optimal graph exists in the set of concrete execution graphs, generated from CGraph.
% TODO Way of reasoning about problems by contracts in declarative way

We pointed out the optimization problems that we face.
% TODO Preliminary cost evaluation
% TODO Runtime statistics gathering
% TODO Cost evaluation by statistics analysis
% TODO Actor entity that decides, whether to restructure execution graph
% TODO How to restructure execution graph in runtime

In future TODO
% TODO General implementation
% TODO Better contracts interface (trello)
% TODO Smart advices


\bibliographystyle{ACM-Reference-Format}
\bibliography{../../bibliography/flame-stream}

\end{document}

\endinput
