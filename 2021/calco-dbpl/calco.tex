\documentclass[sigconf]{acmart}

\usepackage{graphicx}
\usepackage{algorithm} % for algorithms
\usepackage{algpseudocode}
\usepackage{booktabs} % For formal tables
\usepackage{amsthm} % For claims
\usepackage{bbm} % indicator function
\usepackage{acmcopyright}

% listings
\usepackage{xcolor,listings}
\usepackage{textcomp}
\lstset{
    upquote=true,
    basicstyle=\small
}

% plots
\usepackage{pgfplots}

% table
\usepackage[flushleft]{threeparttable} % http://ctan.org/pkg/threeparttable
\usepackage{booktabs,caption}

\theoremstyle{remark}

\settopmatter{printacmref=false, printccs=true, printfolios=true}
\pagestyle{empty} % removes running headers

\newcommand{\PicScale}{0.5}
\newcommand {\FlameStream} {FlameStream}
\begin{document}

% \copyrightyear{2019}
% \acmYear{2019}
% \setcopyright{rightsretained}
% \acmConference[DEBS '19]{DEBS '19: The 13th ACM International Conference on Distributed and Event-based Systems}{June 24--28, 2019}{Darmstadt, Germany}
% \acmBooktitle{DEBS '19: The 13th ACM International Conference on Distributed and Event-based Systems (DEBS '19), June 24--28, 2019, Darmstadt, Germany}\acmDOI{10.1145/3328905.3332514}
% \acmISBN{978-1-4503-6794-3/19/06}

\title{Calco: Specifying Distributed Dataflows with Contracts }

\author{Andrey Stoyan}
\affiliation{%
  \institution{Yandex}
  \city{Saint Petersburg}
  \country{Russia}}
\email{yukio@yandex-team.ru}

\author{Artem Trofimov}
\affiliation{%
  \institution{Yandex}
  \city{Saint Petersburg}
  \country{Russia}}
\email{tomato@yandex-team.ru}

\author{Nikita Sokolov}
\affiliation{%
  \institution{Yandex}
  \city{Saint Petersburg}
  \country{Russia}}
\email{faucct@yandex-team.com}

\author{Boris Novikov}
\affiliation{%
  \institution{HSE University}
  \city{Saint Petersburg}
  \country{Russia}}
\email{borisnov@acm.org}

\author{Igor Kuralenok}
\affiliation{%
  \institution{Yandex}
  \city{Saint Petersburg}
  \country{Russia}}
\email{solar@yandex-team.ru}

\begin{abstract}
    Distributed dataflows can be specified in two ways: by defining an execution graph or specifying the desired result, e.g., using SQL.
    An execution graph consists of arbitrary operations, so it cannot be automatically transformed for optimization.
    SQL is popular for data analytics, and its optimization is well-researched.
    However, it is sometimes inconvenient or impossible to use SQL for general data management tasks, such as ETL, machine learning pipelines, etc.
    In this work, we introduce a novel approach to specify distributed dataflows.
    Our method is based on declarative specifications of user-defined operations called contracts.
    Such specifications allow us to build a set of equivalent graphs, which form a space for optimization.
    We implement a graphs generation prototype and outline the challenges regarding the optimization problem.
\end{abstract}

% \begin{CCSXML}
% \begin{CCSXML}
% \begin{CCSXML}
% <ccs2012>
% <concept>
% <concept_id>10002951.10002952.10002953.10010820.10003208</concept_id>
% <concept_desc>Information systems~Data streams</concept_desc>
% <concept_significance>500</concept_significance>
% </concept>
% <concept>
% <concept_id>10002951.10003317.10003347.10003356</concept_id>
% <concept_desc>Information systems~Clustering and classification</concept_desc>
% <concept_significance>500</concept_significance>
% </concept>
% <concept>
% <concept_id>10002951.10003227.10003351.10003446</concept_id>
% <concept_desc>Information systems~Data stream mining</concept_desc>
% <concept_significance>300</concept_significance>
% </concept>
% </ccs2012>
% \end{CCSXML}

% \ccsdesc[500]{Information systems~Data streams}
% \ccsdesc[500]{Information systems~Clustering and classification}
% \ccsdesc[300]{Information systems~Data stream mining}

% \keywords{Data streams, text classification, reproducibility, exactly once}

\maketitle

\thispagestyle{empty}

\section{Introduction}
TODO cross domain optimization

TODO SQL is just not enough

TODO intersecting queries optimization (cross-query optimization)

TODO runtime information is needed for optimization, so it should be done automatically

TODO too many concrete graphs for the simple examples to code them manually

TODO


\section{Running example}
To consider possibilities and limitations of the our approach later we introduce an example machine learning streaming problem.

Let's suppose that we have an application that consists of the frontend and backend sides.
They both log user's queries and send logs to the log service.
For simplicity, let's assume that every user query produces one log entity from the frontend and one log entity from the backend, and they have equal and unique query id.

The log entities have the following structure (ts stands for the timestamp):

\begin{tabular}{|l|llll|}
    \hline
    \textbf{frontend} & version & queryId & userId & ts \\
    \hline
\end{tabular}

\vspace{0.1em}

\begin{tabular}{|l|lllll|}
    \hline
    \textbf{backend} & id & queryId & userId & ts & payload \\
    \hline
\end{tabular}

We want to calculate statistics about the time that logged users with new frontend client versions wait for the application answer per user session.
To evaluate when the session ends we will use machine learning model that requires frontend features and users features.

One of the possible solutions is represented on the image TODO. % TODO image ref
Obviously, this solution is not optimal.
Filters are applied to streams too late and number of partitions potentially can be reduced.
To be able to fix it safely we need to know about the requirements and impact of the each operation.

\begin{itemize}
    \item \textbf{joinByQueryId}: Requires queryId field in the elements of both streams.
    \item \textbf{addUsersFeatures}: Requires partition by userId. Users should not be filtered before to get valid statistics. Adds userFeatures field to elements.
    \item \textbf{addFrontFeatures}: Requires only new front versions, adds frontFeatures field to elements in the stream.
    \item \textbf{filterNewFronts}: Leaves only elements with new front versions.
    \item \textbf{modelInference}: Requires front and users features. Sets trigger that signals when to emit data from aggregation.
    \item \textbf{filterAuthorizedUsers}: Requires users features. Filters users that are authorized.
    \item \textbf{stats}: Requires authorized users from new front versions. Also needs session trigger to be set. Sends accumulated statistics to the dashboard.
\end{itemize}

% TODO SQL proof
We see here complex pipeline with stateful operations and streaming triggers management.
It is not possible to describe itin SQL because of it.
Moreover, we see non-trivial operation requirements here, so concrete execution graph specifying will not provide space for optimizations.

% TODO too much lets
Let's consider another pipeline that satisfies requirements: TODO. % TODO image ref
We can see that filters here are maximally close to data sources.
And after permutation of stateful operations we have one less data partition that is rather expensive operation.

In section TODO % TODO number of section
we specify this task using contract-based approach and see, how it helps in optimizations.

% TODO insert properly
% TODO ops naming
\begin{figure}
    \label{fig:running-example-suboptimal}
    \label{fig:running-example-optimal}
    \includegraphics[width=\linewidth]{images/debs-calco-example}
\end{figure}


\section{Calco overview}
As we discussed earlier, we need information about semantics of the each operation to be able to permute them safely.
Also we want to use custom operations in the graph, because defining concrete set of operations with known permutation rules is not enough flexible approach.

Thus we propose to specify semantics of custom operations manually.
It is similar idea to the axiomatic semantics of the programming languages (TODO ref), based on the Hoare triples.
Hoare triple consists of the first predicate that should be satisfied for the program context before the statement execution, of statement, which semantics we describe, and of the second predicate that should be satisfied for the program context after the statement execution.
We propose to annotate each custom operation with the input contract, that the input data stream should satisfy, and with the output contract that describes, how this operation changes the input data stream.
So we can specify the computation as a set of annotated nodes, using this representation we can generate all concrete graphs with such nodes, which contracts are satisfied (hence, all operations work properly).
Such novel specification approach we call CGraph.

Let's consider CGraph and contracts in details.

\subsection{CGraph}

Let us define an environment as a set of nodes, annotated with input and output contracts (InCont and OutCont).
This set consists of the data sources, annotated with OutCont, one arity transformations (with one InCont and OutCont) and two arity transformations (with two InCont and OutCont).
Here we limit possible arity of operations for simplicity.

\begin{lstlisting}[language=Haskell]
type CSource = OutCont
type CTfm1 = (InCont, OutCont)
type CTfm2 = (InCont, InCont, OutCont)

data Env = Env
  { sources :: Map NodeName CSource
  , tfms1   :: Map NodeName CTfm1
  , tfms2   :: Map NodeName CTfm2 }
\end{lstlisting}

Graph gets data from the data sources, transforms it in transformation nodes.
But some of the transformation nodes do particular side effects that form result of the graph execution (writes data to the some storage, displays some statistics on the dashboard, etc.).
Such nodes should exist in all concrete execution graphs that correspond to the given CGraph.
Let's call the set of the nodes with the such side effects as a graph semantics.
It will let us to have useless nodes in environment (that are not necessary for graph to have the needed semantics) and reduce the total number of concrete graphs to enumerate.

\begin{lstlisting}[language=Haskell]
type Semantics = Set NodeNames
\end{lstlisting}

And now we are ready to define CGraph.
It is simply a pair of environment and semantics.

\begin{lstlisting}[language=Haskell]
type CGraph = (Env, Semantics)
\end{lstlisting}

\subsection{Input contracts}

Let's suppose that each element in the data stream is a mapping from attribute names to data (like a row in relational database).
So the data stream can be described by two sorts of information: attributes that every element has and the properties of the data.
We can define stream state as follows:

\begin{lstlisting}[language=Haskell]
type Attr = String
type Prop = String

data State = State
  { attrs :: Set Attr
  , props :: Set Prop }

empty :: State
empty = State { attrs = Set.empty
              , props = Set.empty }

union :: State -> State -> State
union s1 s2 = State
  { attrs s1 `Set.union` attrs s2
  , props s1 `Set.union` props s2 }
\end{lstlisting}

Input contract can be defined as tuple of three sets:
\begin{enumerate}
    \item set of attributes that are required in the input stream,
    \item set of properties that are required in the input stream,
    \item set of properties that are prohibited in the input stream.
\end{enumerate}

\begin{lstlisting}[language=Haskell]
data InCont = InCont
  { attrsI  :: Set Attr
  , propsI  :: Set Prop
  , propsI' :: Set Prop }
\end{lstlisting}

Operation's input stream state can be matched with the input contract to check if it satisfies that contract.

\begin{lstlisting}[language=Haskell]
match :: State -> InCont -> Bool
match s c =
     attrsI c `Set.isSubsetOf` attrs s
  && propsI c `Set.isSubsetOf` props s
  && propsI' c `Set.disjoint`  props s
\end{lstlisting}

\subsection{Output contracts}

Output contracts can be defined as a tuple of the three sets and one boolean:
\begin{enumerate}
    \item set of attributes to be added,
    \item boolean that is true if an operation is a projection (it removes all attributes that exist in the incoming stream),
    \item set of properties to be added,
    \item set of properties to be deleted.
\end{enumerate}

\begin{lstlisting}[language=Haskell]
data OutCont = OutCont
  { attrsO  :: Set Attr
  , isProj  :: Bool
  , propsO  :: Set Prop
  , propsO' :: Set Prop }
\end{lstlisting}

To get the output data stream state of the node, input data stream state should be updated with the output contract of the operation.

\begin{lstlisting}[language=Haskell]
update :: State -> OutCont -> State
update s c = State
  { attrs = Set.union
      (attrsO c)
      (if isProj c then Set.empty
                   else attrs s)
  , props = Set.union
      (propsO c)
      (props s `Set.difference` propsO' c) }
\end{lstlisting}

To get the output data stream state of the data source, empty state should be updated with the source's output contract.

\begin{lstlisting}[language=Haskell]
sourceOut :: OutCont -> State
sourceOut = update State.empty
\end{lstlisting}

To get the output data stream state of the one-arity transformation, incoming data stream state should be updated with the transformation's output contract.

\begin{lstlisting}[language=Haskell]
tfm1Out :: State -> OutCont -> State
tfm1Out = update
\end{lstlisting}

To get the output data stream state of the two-arity transformation, union of the incoming data streams states should be updated with the transformation's output contract.

\begin{lstlisting}[language=Haskell]
tfm2Out :: State -> State
        -> OutCont -> State
tfm2Out s1 s2 = update (s1 `State.union` s2)
\end{lstlisting}

\subsection{Work scheme}

All concrete graphs that correspond to the given CGraph can be generated.
So having the cost function we can choose the most optimal one and run it.
But having the execution statistics from the runtime, cost function can be much more precise.
Thus we should continuously recompute the concrete graphs costs using the updating statistics.
When the running graph becomes enough less optimal then another graph to compensate the reconfiguration cost, running graph should be dynamically reconfigured.

\subsection{Prototype}

We have implemented a prototype in Haskell.
It consists of:
\begin{enumerate}
    \item contracts definition and CGraph definition,
    \item function that checks if a concrete graph corresponds to the given CGraph,
    \item function that generates all concrete graph that correspond to the given CGraph (runs rather fast because of early enumeration branch truncation),
    \item set of basic operations that emulate such set of the Apache Beam framework,
    \item evaluation function that runs the concrete graph.
\end{enumerate}

TODO ref https://github.com/flame-stream/halco


\section{Graphs generation}
We need to generate all execution graphs that correspond to the given CGraph.
To do it we build one big directed acyclic graph that includes all needed execution graphs and then extract them.

We represent graph as adjacency list, implemented as a map:

\begin{lstlisting}[language=Haskell]
data Node =
    Source NodeName
  | Op1 NodeName NodeId
  | Op2 NodeName NodeId NodeId

newtype Graph = Graph (Map NodeId Node)
\end{lstlisting}

Big graph we will build layer by layer.
The first one consists of the all environment sources.
To get next we consider all available sources and operations and if some sources satisfy input contracts of the some operation, we add this operation to the big graph and to the sources list.
It is enough to generate number of layers one more than number of operations.

During generation we use some tricks to make it faster.
Firstly, for each source we keep a set of its predecessor operations to filter the following operations (each execution graph cannot have node copies).
Secondly, we keep a set of the generated nodes to not to add the same nodes later with different id.

For each node from the Semantics set we find its ids in the big graph.
Then we calculate a cartesian product of the such sets of ids.
Each element of the cartesian product is a set of ids that correspond to the each node of the Semantics set.
For each element we extract corresponding execution graph from the big graph.
And finally we filter graphs that does not have repeating nodes.

TODO link 
% https://github.com/flame-stream/halco/blob/c7bba22e6ca4e9b8d28d14dcc76e261673d19d09/src/Halco/GraphGen/Fast.hs


\section{Example Evaluation}
Semantics for the running example consists of the "fraudMetric" and "participants" nodes.
The environment has three sources, two one-arity operations, and two two-arity operations.

We discussed earlier an abstract definition of the contracts using the type classes and a particular implementation of them. 
However, using this implementation we are not able to describe the required input data state of the fraud metric, which works on the different alternatives of the input data.
So we need to create a new implementation of input contract.
We just write an instance of the InCont for the list of the InContImpl with the needed match function:

\begin{lstlisting}[language=Haskell]
newtype ListImpl = ListImpl [InContImpl]

instance InCont State ListImpl where
  match :: State -> ListImpl -> Bool
  match s (ListImpl is) = any (match s) is
\end{lstlisting}

For example let us consider contracts of the bid-auction join and the fraud metric.
Join requires "bid.auction" and "auction.id" attributes and sets "joinedBA" property:
\begin{lstlisting}[language=Haskell]
( InContImpl (Set.singleton "bid.auction")
             Set.empty Set.empty
, InContImpl (Set.singleton "auction.id")
             Set.empty Set.empty
, OutContImpl Set.empty
              (Set.singleton "joinedBA") )
\end{lstlisting}

Fraud metric has one input contract with four alternatives (list of four InContImpl):
only bid attributes
or joined bid and auction attributes
or joined bid and person attributes
or all attributes.

We implemented a graphs generation prototype in Haskell.
Our prototype generates all possible 6 execution graphs corresponding to the CGraph of the running example.



TODO SQL comparison


\section{Challenges}
As we mentioned earlier, this work focuses on the problem of equivalent graph generation. The following challenges remain in the adaptation of our technique for end-to-end distributed dataflows optimization:

\textbf{Multicriteria optimization.}
Our framework requires a complex cost model because a user may desire to include some business metrics in the optimization process, e.g., the quality of results or acceptable performance, as we demonstrated in the running example. The planner based on such cost model should use multicriteria optimization that is limitedly studied~\cite{yarygina2014optimizing}.

\textbf{Runtime reconfiguration.}
In batch processing, some systems support dynamic reconfiguration to a new execution graph, e.g., Spark Catalyst~\cite{armbrust2015spark}. Although of efforts on the topic~\cite{grulich2020grizzly, 10.14778/3329772.3329777}, stream processing systems have a lack of such mechanism for global optimization, e.g., it is unclear how to estimate the cost of reconfiguration in comparison with the potential outcome from the more optimal graph.

\textbf{Improving contracts expressiveness and interoperability.}
Currently, contracts are not convenient enough to use in practice, e.g., if some operations need incoming data not to be filtered, adding new filter property makes a user add it to the prohibited properties set of input contracts of all such operations manually. We also aim to build an automatic contracts generation from popular declarative languages such as SQL.


\section{Related work}
We review works which propose declarative approaches to define distributed computations and use it to make optimizations.
Most of such works are dedicated to the specific computational areas.

There are

For example: machine learning algorithms and graph queries processing.



Works dedicated to distributed machine learning algorithms implementation provide possibilities to effectively compute matrix operation TODO references.

TODO machine learning \\
TODO graph queries processing \\
TODO paper references \\
TODO


\section{Conclusion}
We presented a novel approach to specify distributed dataflows.
It is based on the contracts, preconditions and postconditions of the graph operations.
We considered a small machine leaning streaming pipeline example, discussed ways of its optimization and saw that our approach defines the most optimal execution graph too.
% TODO We demonstrated the prototype?
Also we outlined optimization challenges that we faced.
Finally, we showed ideas for the future work enhancements.

We proposed to specify dataflow as CGraph (contracted graph).
CGraph is a pair of environment and semantics, where environment maps nodes to their contracts, and semantics is a set of nodes that produce dataflow target side-effects.
We provide an algorithm that generates all concrete execution graphs that satisfy contracts and semantics.

We discussed the example that SQL implementation seems to be unnatural.
And the only possible way to specify it is execution graph, that represents rather imperative approach then declarative, hence do not provide enough space for optimizations.
We specified the example problem by CGraph and pointed out that the most optimal graph exists in the set of concrete execution graphs, generated from CGraph.
% TODO Way of reasoning about problems by contracts in declarative way

We pointed out the optimization problems that we face.
% TODO Preliminary cost evaluation
% TODO Runtime statistics gathering
% TODO Cost evaluation by statistics analysis
% TODO Actor entity that decides, whether to restructure execution graph
% TODO How to restructure execution graph in runtime

In future TODO
% TODO General implementation
% TODO Better contracts interface (trello)
% TODO Smart advices


\bibliographystyle{ACM-Reference-Format}
\bibliography{bibliography/flame-stream}

\end{document}

\endinput
