\documentclass[sigconf]{acmart}

\usepackage{graphicx}
\usepackage{algorithm} % for algorithms
\usepackage{algpseudocode}
\usepackage{booktabs} % For formal tables
\usepackage{amsthm} % For claims
\usepackage{bbm} % indicator function
\usepackage{acmcopyright}

% listings
\usepackage{xcolor,listings}
\usepackage{textcomp}
\lstset{upquote=true}

% plots
\usepackage{pgfplots}

% table
\usepackage[flushleft]{threeparttable} % http://ctan.org/pkg/threeparttable
\usepackage{booktabs,caption}

\theoremstyle{remark}

\settopmatter{printacmref=false, printccs=true, printfolios=true}
\pagestyle{empty} % removes running headers

\newcommand{\PicScale}{0.5}
\newcommand {\FlameStream} {FlameStream}
\begin{document}

% \copyrightyear{2019}
% \acmYear{2019}
% \setcopyright{rightsretained}
% \acmConference[DEBS '19]{DEBS '19: The 13th ACM International Conference on Distributed and Event-based Systems}{June 24--28, 2019}{Darmstadt, Germany}
% \acmBooktitle{DEBS '19: The 13th ACM International Conference on Distributed and Event-based Systems (DEBS '19), June 24--28, 2019, Darmstadt, Germany}\acmDOI{10.1145/3328905.3332514}
% \acmISBN{978-1-4503-6794-3/19/06}

\title{Calco: a contract-based approach to specify distributed dataflows}

% \author{Alexander Chernokoz, Darya Sharkova, Artem Trofimov, Nikita Sokolov, Ekaterina Gorshkova, and Boris Novikov}
% \affiliation{%
% \institution{$^5$ ITMO University}
%   \city{St. Petersburg}
%   \country{Russia}
% }
% \affiliation{%
% \institution{$^1$Higher School of Economics}
% }
% \author{Artem Trofimov,$^ {1,2}$    Mikhail Shavkunov,$^3$    Sergey Reznick,$^4$     Nikita Sokolov,$^{5}$   Mikhail Yutman,$^3$ \\   Igor E. Kuralenok,$^1$    and  Boris Novikov$^ {3}$ }
% \affiliation{%
% \institution{$^1$JetBrains Research}
% }
% \affiliation{%
% \institution{$^2$Saint Petersburg State University}
% }
% \affiliation{%
% \institution{$^3$National Research University Higher School of Economics}
% }
% \affiliation{%
% \institution{$^4$ Kofax}
% }
% \affiliation{%
% \institution{$^5$ ITMO University}
%   \city{St. Petersburg}
%   \country{Russia}
% }
% \email{\string{trofimov9artem, mv.shavkunov, sergey.reznick, faucct, myutman, ikuralenok\string}@gmail.com, borisnov@acm.org}

\begin{abstract}

    There are two general ways to define computations: imperative and declarative.
    The first one is more transparent for programmers, while the second one is more suitable for complex optimizations.
    One declarative definition can have multiple implementations so that system automatically chooses the most optimal one.
    Accordingly, distributed dataflow can be specified in two ways: defining concrete execution graph and SQL.
    A concrete graph does not capture high-level information about the problem it solves, so a distributed processing system cannot permute operations for performance purposes.
    Hence, graph optimization is generally a programmer concern, but real-world graphs can consist of many nodes that make it hard to optimize them manually.
    Moreover, most of the necessary information for the graph cost evaluation is available only in runtime.
    SQL is popular for data analytics, and its optimization is well-researched.
    However, it is inconvenient or sometimes impossible to use SQL for general data management tasks, such as ETL, machine learning pipelines, etc.

    In this work, we introduce a novel approach to specify distributed dataflows.
    Our method is based on declarative specifications of user-defined operations that we call contracts.
    Such specifications allow us to automatically generate execution graphs with the needed semantics to choose the most optimal one.
    An arbitrary operation or a dataflow part can be described by contracts, so this approach combines the transparency of the imperative approach and optimization possibilities of the declarative one.
    We implement a prototype and demonstrate automatic graphs generation on a real-world dataflow.
    We also outline the challenges that we face regarding the optimization problem.

\end{abstract}

% \begin{CCSXML}
% \begin{CCSXML}
% \begin{CCSXML}
% <ccs2012>
% <concept>
% <concept_id>10002951.10002952.10002953.10010820.10003208</concept_id>
% <concept_desc>Information systems~Data streams</concept_desc>
% <concept_significance>500</concept_significance>
% </concept>
% <concept>
% <concept_id>10002951.10003317.10003347.10003356</concept_id>
% <concept_desc>Information systems~Clustering and classification</concept_desc>
% <concept_significance>500</concept_significance>
% </concept>
% <concept>
% <concept_id>10002951.10003227.10003351.10003446</concept_id>
% <concept_desc>Information systems~Data stream mining</concept_desc>
% <concept_significance>300</concept_significance>
% </concept>
% </ccs2012>
% \end{CCSXML}

% \ccsdesc[500]{Information systems~Data streams}
% \ccsdesc[500]{Information systems~Clustering and classification}
% \ccsdesc[300]{Information systems~Data stream mining}

% \keywords{Data streams, text classification, reproducibility, exactly once}

\maketitle

\thispagestyle{empty}

\section{Introduction}
There are two common options to define dataflows in state-of-the-art distributed stream processing systems.
The first option is defining a logical {\em execution graph}.
An execution graph is a directed graph, where nodes represent operations and edges denote data flows.
This mechanism is robust and suitable for complex dataflows (TODO synonym) but has limited optimization abilities.
A processing system can apply only local physical optimizations because it cannot ensure that the restructured graph is equivalent to the original one.

Another way is declarative: the user defines the result that he aims to obtain, and the execution graph is generated automatically by the processing system.
The declarative approach is commonly implemented using SQL.
SQL is based on relational algebra that includes a set of operations and query transformation rules.
This way, a system can obtain multiple equivalent execution graphs and choose optimal one using a {\em cost model}.
Unfortunately, SQL is not rich enough to express some user-defined operations, e.g., complex machine learning pipelines~\cite{PROOF} (TODO).

In this work, we present a declarative framework called {\em Calco} to specify distributed dataflows.
Our framework is bases on the ideas of the contract programming~\cite{REF} (TODO) and aim to solve the following problems:

{\bf Custom dataflows optimization}: user can annotate a custom operation with {\em contracts} which captures operation properties and allows the system to apply global optimization, e.g., to permute such operations.

{\bf Cross-domain optimization}: SQL can be automatically translated into Calco contracts and optimized together with the custom user-defined operations.

Optimization problem can be separated into two tasks: equivalent graphs generation and developing a cost model.
In this work we focus on the first one.


\section{Running example}
TODO SQL cross query optimization
TODO \\


\section{Calco overview}
\subsection{Contracts interface}

Graph operations can be user-defined, so we cannot deduce their characteristics (e.g., commutativity) automatically to transform the graph safely.
So we need to ask the user to specify the characteristics manually.
Such specifications we call contracts.

There are two kinds of contracts: input and output.
Input contract specifies requirements that input data should satisfy, output contract defines how an operation changes the input data.
We call information about the attributes and properties of data elements a {\em data scheme}.
Input contract is a data type with the {\em match} function defined: it takes an input data scheme, input contract, and returns boolean if an input data scheme satisfies an input contract~\footnote{All notions and examples are demonstrated in Haskell}:

\begin{lstlisting}[language=Haskell]
class InCont s i where
  match :: s -> i -> Bool
\end{lstlisting}

Graph nodes can be of three types: data sources, one-arity operations, and two-arity operations.
So there are three definitions of output contracts.
The output contract of the data source should have a function to get an output data scheme from it:

\begin{lstlisting}[language=Haskell]
class OutCont s o where
  toScheme :: o -> s
\end{lstlisting}

Output contract of the one-arity operation defines, how the operation changes the input data, hence there should be a function {\em update1} that takes an input data scheme, an output contract and returns an output data scheme:

\begin{lstlisting}[language=Haskell]
class OutCont1 s o where
  update1 :: s -> o -> s
\end{lstlisting}

Output contract of the two-arity operation defines how the node transforms the data of two sources:

\begin{lstlisting}[language=Haskell]
class OutCont2 s o where
  update2 :: (s, s) -> o -> s
\end{lstlisting}

\subsection{Contracts implementation}

Input and output data can be described using two sorts of information: attributes that every data element has and properties of these attributes.
Properties are denoted just as strings, like {\em "reliablePerson"} or {\em "popularItem"}.

\begin{lstlisting}[language=Haskell]
type Attr = String
type Prop = String

data Scheme = Scheme
  { attrs :: Set Attr
  , props :: Set Prop }

empty :: Scheme
empty = Scheme { attrs = Set.empty
               , props = Set.empty }

union :: Scheme -> Scheme -> Scheme
union s1 s2 = Scheme
  { attrs = attrs s1 `Set.union` attrs s2
  , props = props s1 `Set.union` props s2 }
\end{lstlisting}

Input contract consists of three sets: 
attributes and properties that should exist in the input data scheme,
and the properties that should not exist in the input data scheme.

\begin{lstlisting}[language=Haskell]
data InContImpl = InContImpl
  { attrsI  :: Set Attr
  , propsI  :: Set Prop
  , propsI' :: Set Prop }

instance InCont Scheme InContImpl where
  match :: Scheme -> InContImpl -> Bool
  match s i =
       attrsI i `Set.isSubsetOf` attrs s
    && propsI i `Set.isSubsetOf` props s
    && propsI' i `Set.disjoint`  props s
\end{lstlisting}

Output contract consists of two sets: 
the attributes that operation adds to each data element and the new properties of the data elements (e.g., filter operation adds property "filteredByCategory").

\begin{lstlisting}[language=Haskell]
data OutContImpl = OutContImpl
  { attrsO :: Set Attr
  , propsO :: Set Prop }

instance OutCont Scheme OutContImpl where
  toScheme :: OutContImpl -> Scheme
  toScheme = update1 Scheme.empty

instance OutCont1 Scheme OutContImpl where
  update1 :: Scheme -> OutContImpl -> Scheme
  update1 s o = Scheme
    { attrs = if delete o then attrsO o
              else attrs s `Set.union` attrsO o
    , props = props s `Set.union` propsO o }

instance OutCont2 Scheme OutContImpl where
  update2 :: 
    (Scheme, Scheme) -> OutContImpl -> Scheme
  update2 (s1, s2) = 
    update1 (s1 `Scheme.union` s2)
\end{lstlisting}

\subsection{CGraph}

{\em Environment} is a set of {\em nodes}, annotated with {\em input} and {\em output contracts}.
Nodes can be data sources, one-arity operations, and two-arity operations.

\begin{lstlisting}[language=Haskell]
data Env i o o1 o2 = Env
  { sources :: Map NodeName o
  , ops1 :: Map NodeName (i, o1)
  , ops2 :: Map NodeName (i, i, o2) }
\end{lstlisting}

Some nodes produce particular side effects that form the result of the running graph (writes data to the storage, displays some statistics on the dashboard, etc.).
Such nodes should exist in all generated execution graphs.
We call a set of nodes that produce such side effects a {\em graph semantics}.

\begin{lstlisting}[language=Haskell]
type Semantics = Set NodeName
\end{lstlisting}

CGraph is simply a pair of environment and semantics:

\begin{lstlisting}[language=Haskell]
type CGraph i o o1 o2 = 
  (Env i o o1 o2, Semantics)
\end{lstlisting}

Graphs with satisfied contracts that include all semantics nodes form the desired set of equivalent graphs. 
The implementation of the execution graphs generation algorithm is rather effective despite the brute force approach because graphs that do not satisfy contracts can be ejected. 
The algorithm is exponential in the size of the semantics set, which is often small enough.
The algorithm implementation details are discussed in the next section.


\section{Example evaluation}
We discussed earlier an abstract definition of the contracts using the type classes and the particular implementation.
However, using this implementation, we cannot describe the required input data scheme of the fraud metric, which works on the different alternatives of the input data.
So we need to create a new implementation of the input contract.
We can write an instance of the InCont for the list of the InContImpl with the needed match function:

\begin{lstlisting}[language=Haskell]
newtype ListImpl = ListImpl [InContImpl]

instance InCont Scheme ListImpl where
  match :: Scheme -> ListImpl -> Bool
  match s (ListImpl is) = any (match s) is
\end{lstlisting}

Fraud metric has one input contract with four alternatives (list of four InContImpl):
only bid attributes
or joined bid and auction attributes
or joined bid and person attributes
or all attributes.

For example, let us also consider contracts of the bid-auction join and the fraud metric.
Join requires "bid.auction" and "auction.id" attributes and sets "joinedBA" property:
\begin{lstlisting}[language=Haskell]
( InContImpl (Set.singleton "bid.auction")
             Set.empty Set.empty
, InContImpl (Set.singleton "auction.id")
             Set.empty Set.empty
, OutContImpl Set.empty False
              (Set.singleton "joinedBA") )
\end{lstlisting}

Semantics for the running example consists of the "fraudMetric" and "participants" nodes.
The environment has three sources, two one-arity operations, and two two-arity operations.

We implemented a graphs generation prototype in Haskell~\footnote{https://github.com/flame-stream/halco}.
Our prototype generates all possible 6 execution graphs corresponding to the CGraph of the running example. Figure~\ref{fig:gen} demonstrates a graph generation process.
Purple graph nodes correspond to the fraud metric semantics node, orange~--- to the participants semantics node.
Graph nodes with red frames represent an extracted graph.

In the first step, the algorithm adds all sources to the universal graph.
In the next steps, it adds all possible nodes such that their input contracts are satisfied by the nodes from the previous step. (TODO)

Note that a semantics operation set with their input contracts looks pretty similar to the declarative dataflows specification approach, e.g., SQL, because we indicate only the desired result.
However, as opposed to the SQL, other user-defined operations annotated with contracts can also be used in equivalent graphs generation process, e.g., can be safely reordered.


\section{Discussion and future work}
In this section we discuss advantages and concerns of the proposed contract-based approach.

\subsection{Contracts}

As we saw, semantics of the custom operations can be naturally described by contracts.
However it may be inconvenient to write all contracts as is.
For example, if some operations need incoming streams not to be filtered, adding new filter property make programmer to add it to this operations manually.
There are some ideas that should make defining CGraph more handy, here are some of them:
\begin{itemize}
    \item properties should be able to be grouped to use them together (prohibit all filter properties as one, for example),
    \item some kind of properties may be prohibited by default,
    \item interactive environment that hints, which attributes and properties are not yet satisfied for the semantics nodes.
\end{itemize}

\subsection{Work scheme and optimization}

Before having runtime statistics reasonable concrete graph should be chosen.
To do that information is needed, how nodes change dataflow cardinality.
Seems like this information should be manually provided by the programmer.

There are no production-ready frameworks for distributed data processing that gather needed statistics for accurate concrete graph cost evaluation.
Also none of them can dynamically reconfigure graph.

Finally, it is needed to precisely compute cost function using the runtime statistics and also estimate the cost of the graph dynamic reconfiguration.

\subsection{General implemenation}

We plan to develop some libraries to be able to specify computations using CGraph on Python and Java.


\section{Related work}
Previous works on distributed dataflows global optimization primarily study optimization of SQL-like approaches~\cite{chang2014hawq, armbrust2015spark, sethi2019presto}. An extension of SQL for machine learning that supports linear algebra operators is presented in~\cite{schule2019mlearn}. Another similar framework is discussed in~\cite{schelter2016samsara}.

Several works go beyond relational algebra. Galax~\cite{re2006complete} applies nested-relational algebra for XML processing. SASE project~\cite{gyllstrom2006sase} introduces a custom algebra for finding temporal patterns across data items. The approach presented in~\cite{hueske2012opening} aims to find safe reordering based on the analysis of the read-set and write-set of user-defined operators.

Many previous works focus on an optimal mapping of the logical execution graph to the computational resources~\cite{grulich2020grizzly, davidson2013optimizing, bosagh2016matrix}. A more complex method that firstly translates queries into an intermediate representation (IR) and then applies physical optimization is presented in~\cite{kroll2019arc}. These techniques support a wide range of physical optimizations: operator redundancy elimination, operator separation, and fusion~\cite{hirzel2014catalog}. However, the optimizations based on operator reordering are limited within such methods due to the lack of algebraic properties defined on the operations.

\section{Conclusion}
% We have observed existing ways to specify computations in distributed data processing frameworks: concrete graph definition and SQL.
% Also we have outlined disadvantages and limitations of these approaches with explanations.

% We have presented a novel approach to specify distributed dataflows.
% It is based on the contracts, preconditions, and postconditions of the custom graph operations (their semantics).
% Such specification has been called CGraph.
% CGraph is a pair of environment and semantics, where the environment maps nodes to their contracts, and semantics is a set of nodes that produce dataflow target side-effects.
% This representation allows us to generate all possible graphs with the needed semantics using defined operations.
% So having a cost evaluation function we can choose the most optimal graph.

% TODO example

% We have discussed challenges about making CGraph specification API comfortable enough to solve real-world tasks and maintain these solutions.
% Finally, we have studied problems that we will encounter in future work regarding the optimization: cost evaluation and the adaptivity loop implementation.

TODO


\bibliographystyle{ACM-Reference-Format}
\bibliography{../../bibliography/flame-stream}

\end{document}

\endinput
