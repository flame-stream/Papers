Previous works on distributed dataflows global optimization primarily study optimization of SQL-like approaches~\cite{chang2014hawq, armbrust2015spark, sethi2019presto}. An extension of SQL for machine learning that supports linear algebra operators is presented in~\cite{schule2019mlearn}. Another similar framework is discussed in~\cite{schelter2016samsara}.

Several works go beyond relational algebra. Galax~\cite{re2006complete} applies nested-relational algebra for XML processing. SASE project~\cite{gyllstrom2006sase} introduces a custom algebra for finding temporal patterns across data items. The approach presented in~\cite{hueske2012opening} aims to find safe reordering based on the analysis of the read-set and write-set of user-defined operators.

Many previous works focus on an optimal mapping of the logical execution graph to the computational resources~\cite{grulich2020grizzly, davidson2013optimizing, bosagh2016matrix}. A more complex method that firstly translates queries into an intermediate representation (IR) and then applies physical optimization is presented in~\cite{kroll2019arc}. These techniques support a wide range of physical optimizations: operator redundancy elimination, operator separation, and fusion~\cite{hirzel2014catalog}. However, the optimizations based on operator reordering are limited within such methods due to the lack of algebraic properties defined on the operations.