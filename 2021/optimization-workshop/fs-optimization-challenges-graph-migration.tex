% well i have nothing to say about this one
Streaming data is ever-changing: new data continues arriving indefinitely, and the statistics for the next window elements might differ significantly from the statistics for previous windows. Even if the optimal query execution plan was selected based upon statistics accumulated for previous windows, the new data statistics might be different enough to render the previous plan no longer optimal. In order to adapt to the changes in data, it's necessary to identify the moment in time in which the previous plan is no longer optimal for the current or upcoming data and to migrate the execution to a new graph. % this is terribly grammatically wrong 
We identify the following challenges in migrating the execution graph in runtime:
\begin{itemize}
    \item \textbf{Technical}: the mechanics of graph migration, particularly for stateful operations such as joins and aggregations, in runtime have been researched \cite{zhu2004dynamic}; % TODO references
    however, most current popular SPEs do not provide such functionality. In order to support graph migration in stream processing frameworks such as Apache Beam, one such strategy is the parallel track strategy described in \cite{zhu2004dynamic}: a second graph can start the query execution along the first one, and the first one can terminate execution once the current window has been processed, so that all the subsequent windows are only processed by the new execution graph.
    % something something about state and migration. i don't remember
    \item \textbf{Research}: identifying the point in time at which it's feasible and beneficial to perform the graph migration is an open problem. First of all, it's necessary to be able to estimate the costs of graph migration at the current point in time. Secondly, we need to establish what qualifies as substantial enough change in statistics to warrant graph migration. % I don't know what we propose in order to do this.
\end{itemize}