Streaming data is ever-changing: new data continues to arrive indefinitely, and the statistics for the following window elements might differ significantly from the statistics for previous windows. 

Even if the optimal query execution plan was selected based upon statistics accumulated for previous windows, the new data statistics might be different enough to render the previous plan no longer optimal. Therefore, to adapt to the changes in data, it is necessary to identify the moment in time in which the previous plan is no longer optimal for the current or upcoming data and to migrate the execution to a new graph. 

We identify the following challenges in migrating the execution graph in runtime:
\begin{itemize}
 \item \textbf{Research}: Identifying the point in time at which it is feasible and beneficial to perform the graph migration is an open problem. First of all, it is necessary to estimate the costs of graph migration at the current point in time. Secondly, we need to establish what qualifies as substantial enough change in statistics to warrant graph migration. 

 \item \textbf{Engineering}: 
    The mechanics of graph migration, particularly for stateful operations such as joins and aggregations, in runtime have been researched~\cite{zhu2004dynamic}.
 However, most current popular SPEs do not provide such functionality. 
   \end{itemize}
One such strategy is the parallel track strategy described in~\cite{zhu2004dynamic}. 
The second graph can start the query execution and the first one, while the first one stops execution when the current window terminates. So all the subsequent windows are only processed by the new execution graph.

