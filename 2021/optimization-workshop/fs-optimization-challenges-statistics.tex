Cost-based optimization requires statistical information on data in order to calculate cost function values for each plan. However, upon starting a streaming query execution, no information about the data, such as its arrival rate, is available. 
Therefore, to properly apply cost-based optimization to streaming SQL queries, it is necessary to collect data statistics over the course of query execution. 
Moreover, since we possess no definitive knowledge about the arriving data, we need to predict statistics for each next window based on statistics for previous windows to utilize an optimal plan for the upcoming windows. To this end, we identify two challenges in using statistics for streaming query optimization:

\begin{itemize}
    \item \textbf{Engineering}:
    Statistical information on stream elements, such as their arrival rate, needs to be collected during execution at runtime without seriously affecting the performance of a distributed SPE. % which can present certain challenges since there is no centralized point at which to collect statistics, so we would need to aggregate it somehow, which can affect the performance of the system itself
    \item \textbf{Research}
We need techniques to predict statistics for upcoming windows based on statistics collected for previous windows.    
We expect that previous window statistics would present a decent baseline. 
However, this assumption requires further investigation. 
\end{itemize}

Popular SPEs and frameworks for defining streaming workflows do not offer any statistics fetching or predicting. 
For example, the Apache Beam framework passes constant values to the query planner instead of any actual data statistics to
 Apache Calcite, a dynamic data management framework that implements its SQL processing functionality.

