To illustrate the problem of the cross-domain optimization, let us consider an example that consists of the SQL query based on the NEXMark~\cite{TODO} benchmark model and the execution graph.

The SQL query joins all tables and calculates the number of participants for each auction:
\begin{lstlisting}[language=SQL]
SELECT auction.id, COUNT(person)
FROM bid
INNER JOIN person ON bidder = person.id
INNER JOIN auction ON auction = auction.id
GROUP BY auction.id
\end{lstlisting}

An execution graph computes metric that helps to detect frauds.
This metric can be computed only using the bids with the less accuracy or using also an information about the bidder and auction with better accuracy and lower latency.
So in the first case the graph consists of two nodes: bids source and frauds metric computation.
In the second case: graph joins all three tables and computes the fraud metric.

To get optimal execution plan joins can be permuted.
Also result of joins can be reused if business needs the fraud metric to be computed accurately.
