% We propose to define a distributed dataflow by specifying a set of equivalent execution graphs using a representation called {\em CGraph}.
% An optimal graph can be chosen from this set using a cost-model.

% Since graph operations are user-defined, we need specify their properties manually to define what is the graph equivalency.
% Such specifications we call contracts.
% Input contract specifies requirements that input data flow should satisfy, output contract defines how graph node changes the input data flow.
% Contracts can be defined in different ways, in this paper we describe the trivial definition because of its simplicity.

Graph operations can be user-defined, so we need to add more information about an operation to be aware of the graph equivalency.
Such specifications we call contracts.
Input contract specifies requirements that input data should satisfy, output contract defines how an operation changes the input data.

Input and output data can be described using two sorts of information: attributes that every data element has and properties of such attributes.
Input contract consists of three sets: the required attributes, the required properties, and the prohibited properties.
Output contract consists of two sets: the attributes and the new properties.
Properties are denoted just as strings, like {\em "reliablePerson"} or {\em "popularItem"}.
We say that contracts are satisfied if output contracts match the input contracts within all connected operations in the execution graph.

{\em Environment} is a set of {\em nodes}, annotated with {\em input} and {\em output contracts} (InCont and OutCont).
Nodes can be data sources, one-arity operations, and two-arity operations.

\begin{lstlisting}[language=Haskell]
data Env = Env
  { sources :: Map NodeName OutCont
  , ops1 :: Map NodeName (InCont, OutCont)
  , ops2 :: Map NodeName
                (InCont, InCont, OutCont) }
\end{lstlisting}

Some nodes produce particular side effects that form the result of the running graph (writes data to the storage, displays some statistics on the dashboard, etc.).
Such nodes should exist in all generated execution graphs.
We call the nodes that produce such side effects a {\em graph semantics}.

\begin{lstlisting}[language=Haskell]
type Semantics = Set NodeName
\end{lstlisting}

CGraph is simply a pair of environment and semantics.

\begin{lstlisting}[language=Haskell]
type CGraph = (Env, Semantics)
\end{lstlisting}

Graphs with satisfied contracts that include all semantics nodes form the desired set of equivalent graphs. The implementation of the execution graphs generation algorithm is rather effective despite the brute force approach because graphs that do not satisfy contracts can be ejected. Hence, the complexity is exponential but depends only on the size of the semantics set, which is often small enough.

% So using the cost-model we can choose optimal one and run it.
% But having the runtime execution statistics, cost-model can be much more precise.
% Thus we should continuously recompute the execution graphs costs using the updating statistics.
% When the running graph becomes enough less optimal then another graph to compensate the reconfiguration cost, running graph should be dynamically reconfigured.

% \subsection{CGraph}

% Let us define an environment as a set of nodes, annotated with input and output contracts (InCont and OutCont).
% This set consists of the data sources, annotated with OutCont, one arity operations (with one InCont and OutCont) and two arity operations (with two InCont and OutCont).
% Here we limit possible arity of operations for simplicity.

% \begin{lstlisting}[language=Haskell]
% type CSource = OutCont
% type CTfm1 = (InCont, OutCont)
% type CTfm2 = (InCont, InCont, OutCont)

% data Env = Env
%   { sources :: Map NodeName CSource
%   , ops1    :: Map NodeName COp1
%   , ops2    :: Map NodeName COp2 }
% \end{lstlisting}

% Graph gets data from the data sources, transforms it by operations.
% But some of the operation nodes do particular side effects that form result of the graph execution (writes data to the some storage, displays some statistics on the dashboard, etc.).
% Such nodes should exist in all concrete execution graphs that correspond to the given CGraph.
% Let's call the set of the nodes with the such side effects as a graph semantics.
% It will let us to have useless nodes in environment (that are not necessary for graph to have the needed semantics) and reduce the total number of concrete graphs to enumerate.

% \begin{lstlisting}[language=Haskell]
% type Semantics = Set NodeNames
% \end{lstlisting}

% And now we are ready to define CGraph.
% It is simply a pair of environment and semantics.

% \begin{lstlisting}[language=Haskell]
% type CGraph = (Env, Semantics)
% \end{lstlisting}

% \subsection{Data stream state}

% We define here a trivial model of contracts, because the pragmatic model is a bit more complex. (TODO)

% Let's suppose that each element in the data stream is a mapping from attribute names to data (like a row in relational database).
% So the data stream can be described by two sorts of information: attributes that every element has and the properties of the data.
% We can define data stream state as follows:

% \begin{lstlisting}[language=Haskell]
% type Attr = String
% type Prop = String

% data State = State
%   { attrs :: Set Attr
%   , props :: Set Prop }

% empty :: State
% empty = State { attrs = Set.empty
%               , props = Set.empty }

% union :: State -> State -> State
% union s1 s2 = State
%   { attrs s1 `Set.union` attrs s2
%   , props s1 `Set.union` props s2 }
% \end{lstlisting}

% \subsection{Input contracts}

% Input contract can be defined as tuple of three sets:
% \begin{enumerate}
%     \item set of attributes that are required in the input stream,
%     \item set of properties that are required in the input stream,
%     \item set of properties that are prohibited in the input stream.
% \end{enumerate}

% \begin{lstlisting}[language=Haskell]
% data InCont = InCont
%   { attrsI  :: Set Attr
%   , propsI  :: Set Prop
%   , propsI' :: Set Prop }
% \end{lstlisting}

% Operation's input stream state can be matched with the input contract to check if it satisfies that contract.

% \begin{lstlisting}[language=Haskell]
% match :: State -> InCont -> Bool
% match s c =
%      attrsI c `Set.isSubsetOf` attrs s
%   && propsI c `Set.isSubsetOf` props s
%   && propsI' c `Set.disjoint`  props s
% \end{lstlisting}

% \subsection{Output contracts}

% Output contracts can be defined as a tuple of the three sets and one boolean:
% \begin{enumerate}
%     \item set of attributes to be added,
%     \item boolean that is true if an operation is a projection (it removes all attributes that exist in the incoming stream), (TODO can we call this as projection?)
%     \item set of properties to be added,
%     \item set of properties to be deleted.
% \end{enumerate}

% \begin{lstlisting}[language=Haskell]
% data OutCont = OutCont
%   { attrsO  :: Set Attr
%   , delete  :: Bool
%   , propsO  :: Set Prop
%   , propsO' :: Set Prop }
% \end{lstlisting}

% To get the output data stream state of the node, input data stream state should be updated with the output contract of the operation.

% \begin{lstlisting}[language=Haskell]
% update :: State -> OutCont -> State
% update s c = State
%   { attrs = Set.union
%       (attrsO c)
%       (if isProj c then Set.empty
%                    else attrs s)
%   , props = Set.union
%       (propsO c)
%       (props s `Set.difference` propsO' c) }
% \end{lstlisting}

% To get the output data stream state of the data source, empty state should be updated with the source's output contract.

% \begin{lstlisting}[language=Haskell]
% sourceOut :: OutCont -> State
% sourceOut = update State.empty
% \end{lstlisting}

% To get the output data stream state of the one-arity transformation, incoming data stream state should be updated with the transformation's output contract.

% \begin{lstlisting}[language=Haskell]
% tfm1Out :: State -> OutCont -> State
% tfm1Out = update
% \end{lstlisting}

% To get the output data stream state of the two-arity transformation, union of the incoming data streams states should be updated with the transformation's output contract.

% \begin{lstlisting}[language=Haskell]
% tfm2Out :: State -> State
%         -> OutCont -> State
% tfm2Out s1 s2 = update (s1 `State.union` s2)
% \end{lstlisting}

% \subsection{Work scheme}

% All concrete graphs that correspond to the given CGraph can be generated.
% So having the cost function we can choose the most optimal one and run it.
% But having the execution statistics from the runtime, cost function can be much more precise.
% Thus we should continuously recompute the concrete graphs costs using the updating statistics.
% When the running graph becomes enough less optimal then another graph to compensate the reconfiguration cost, running graph should be dynamically reconfigured.

% \subsection{Prototype}

% We have implemented a prototype in Haskell.
% It consists of:
% \begin{enumerate}
%     \item contracts definition and CGraph definition,
%     \item function that checks if a concrete graph corresponds to the given CGraph,
%     \item function that generates all concrete graph that correspond to the given CGraph (runs rather fast because of early enumeration branch truncation),
%     \item set of basic operations that emulate such set of the Apache Beam framework,
%     \item evaluation function that runs the concrete graph.
% \end{enumerate}

% TODO ref https://github.com/flame-stream/halco
