\documentclass[12pt]{article}
\usepackage{fontspec}
\usepackage{polyglossia}
\setdefaultlanguage{russian}
\setmainfont[Mapping=tex-text]{CMU Serif}

\begin{document}
%% Весь этот текст можно удалить
%% ====== от сих =====
% {\LARGE Привет, пользователь \LaTeX!}
% \bigskip

% Если тебе не нужно ничего, кроме нескольких абзацев текста, то этот шаблон как раз для тебя.
% {\fontspec{Liberation Serif} Разные шрифты}, автоматические переносы, и простые математические
% формулы --- всё имеется.
% \Large
% $$
% (a + b)^n = \sum_{k=0}^n{n\choose k}a^{n-k}b^k
% $$

% \bigskip
% \centering{Подумаешь, бином Ньютона.}
%% ===== и до сих =====

\textbf{Общая схема тракера}. Для заданного предиката $p(x)$/$p(X)$ ищем время $t$, такое, что

$$\forall \tau < t, x \in Cl^{-1}_D(a_\tau) \cap W_\tau : \neg p(x)$$

или

$$\forall \tau < t, X \subseteq Cl^{-1}_D(a_\tau) \cap W_\tau : \neg p(X)$$


\textbf{Пример 1:} время для снепшота стейта (если разделить множество $W_\tau$ по частям графа, например, $W^{l1}_\tau$, $W^{l2}_\tau$ и т.п. то получится совсем точный аналог пунктуаций для ациклических графов)

$$\forall \tau < t, X \subseteq Cl^{-1}_D(a_\tau) \cap W_\tau : X = \emptyset$$

\textbf{Пример 2:} время для чистки стейта (считаем, что есть предикат для элемента, который говорит, содержит ли элемент нужный ключ - contains(x, key))

$$\forall \tau < t, x \in Cl^{-1}_D(a_\tau) \cap W_\tau : \neg contains(x,key)$$

\textbf{Примечание 1:} если предикат объявлен на одном элементе потока, то его можно проверять в каждой операции независимо, а в тракер посылать только id предиката, которому удовлетворяет элемент. Если на множестве - то необходимо посылать данные, необходимые для вычисления этого предиката. Пример со снепшотом является как раз случаем, когда мы посылаем данные - время.

\textbf{Примечание 2:} при таком подходе нужно либо рассматривать стейт как обычный элемент, у которого есть время $\tau$, либо блокировать вход на время чистки/снепшота. Иначе, есть вероятность гонок.

\textbf{Зависимые и независимые переходы}. Правила перехода $D$ независимы относительно предиката $p(x)$, если $\forall \tau_1, \tau_2 : \tau_1 < \tau_2$ выполянется:\\

$(\exists \tau_1,x \in Cl^{-1}_D(a_{\tau_1}) \cap W_{\tau_1} : p(x)) \land (\forall \tau > \tau_1 : a_\tau = \emptyset) \Rightarrow \\ \forall \tau > \tau_1, x \in Cl^{-1}_D(a_{\tau}) \cap W_{\tau} : \neg p(x)$

\textbf{Примеры зависимых переходов}. Переход в группировке - зависимый. Наверняка, тут есть больше примеров :)

\textbf{Регистрация предикатов in-flight}. При регистрации in-flight, нас интересуют два времени $t_s$ и $t_f$

$$\forall \tau < t_f, \tau > t_s, x \in Cl^{-1}_D(a_\tau) \cap W_\tau : \neg p(x)$$

\end{document}
