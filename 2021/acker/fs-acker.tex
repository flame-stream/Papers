\documentclass{vldb}

\usepackage[utf8]{inputenc}

\usepackage{graphicx}
\usepackage{url}
\usepackage{multirow}
\usepackage{array}
\usepackage{hyperref}
\usepackage{algorithm} % for algorithms
% \usepackage{algorithmicx}
% \usepackage{algorithm2e} % for algorithms
\usepackage{algpseudocode}
% \usepackage{booktabs} % For formal tables
\algdef{SE}[SUBALG]{Indent}{EndIndent}{}{\algorithmicend\ }%
\algtext*{Indent}
\algtext*{EndIndent}

\usepackage{balance}  % for  \balance command ON LAST PAGE  (only there!)
\usepackage{caption}
\usepackage{subcaption}
\usepackage{tikz}
\usepackage[flushleft]{threeparttable}
\usetikzlibrary{backgrounds, calc, positioning, fit, decorations.pathreplacing}
% \theoremstyle{remark}

\pagestyle{empty} % removes running headers

\newcommand{\PicScale}{0.5}
\newcommand {\FlameStream} {FlameStream}
\newcommand {\tracker} {Tracker}
\newcommand {\acker} {Acker}

\newtheorem{lemma}{Lemma}

% Include information below and uncomment for camera ready
\vldbTitle{Substreams Management in Distributed Streaming Dataflows}
\vldbAuthors{Artem Trofimov, Nikita Sokolov, Igor Kuralenok, Nikita Marshalkin, and Boris Novikov}
\vldbDOI{https://doi.org/10.14778/xxxxxxx.xxxxxxx}
\vldbVolume{12}
\vldbNumber{xxx}
\vldbYear{2019}

\begin{document}

\title {Substream Management in Distributed Streaming Dataflows}

\numberofauthors{5}

\author{
\alignauthor
Artem Trofimov\\
    \affaddr{Yandex}\\
    \affaddr{Saint Petersburg, Russia}\\
    \email{tomato@yandex-team.ru}
\alignauthor
Nikita Sokolov\\
    \affaddr{Yandex}\\
    \affaddr{Saint Petersburg, Russia}\\
    \email{faucct@yandex-team.ru}
\alignauthor
Igor Kuralenok\\
    \affaddr{Yandex}\\
    \affaddr{Saint Petersburg, Russia}\\
    \email{solar@yandex-team.ru}
\and 
\alignauthor
Nikita Marshalkin\\
    \affaddr{VK}\\
    \affaddr{Saint Petersburg, Russia}\\
    \email{n.marshalkin@corp.vk.com}
\alignauthor
Boris Novikov\\
    \affaddr{HSE University}\\
    \affaddr{Saint Petersburg, Russia}\\
    \email{borisnov@acm.org}
}

\maketitle

\begin{abstract}
Stream Processing Engines (SPEs) handle a potentially infinite sequence of data elements. This sequence often can be viewed as a mixture of finite substreams with distinct properties: event times ranges, payload values, etc. Some operators ought to know since then there will be no longer elements with specified properties, i.e., when the substream ends. For instance, stateful streaming operators should clear outdated state to avoid running out of memory. Time window operators should release output after all elements within the specified time range arrived. Most state-of-the-art SPEs use {\em punctuations} to manage substreams. The punctuations approach is powerful but has limitations: it does not support cyclic dataflows, poorly scalable due to frequent broadcasts, and is inefficient for small substreams because of high network traffic overhead. In this work, we present a new substream management framework called \tracker\ that supports cyclic dataflows and tiny substreams (even substreams containing a single element). This method bases on the idea that a special agent within an SPE monitors for substreams and notifies SPE nodes about substreams lifespan events. We demonstrate that our implementation can provide the same semantics of notifications as punctuations but implies lower network traffic overhead. Experiments show that our technique is scalable, efficiently handles real-world cyclic dataflows, and outperforms punctuations for small substreams.
\end{abstract}

% \keywords{Data streams, exactly-once, drifting state, optimistic OOP}

\thispagestyle{empty}

\section {Introduction}
\label {fs-acker-intro}

The processing of a data stream without insights into the properties of its data elements can be challenging. For example, it may be unclear when a system can prune outdated keyed state~\cite{Tucker:2003:EPS:776752.776780}, release windowed aggregations~\cite{Begoli:2019:OSR:3299869.3314040}, or create a state snapshot for an epoch~\cite{Carbone:2017:SMA:3137765.3137777}.

Each of these scenarios is a particular case of a problem of monitoring substreams emergence and termination that we call a {\em substream management problem}. A substream is a part of the stream such that all its elements satisfy some predicate. 
For example, in the case of state pruning, the predicate is {\em [a data element key equals to $K$]}, for time window aggregations, the predicate is {\em [a data element has a timestamp less than $T$]}, and for state snapshotting it is {\em [a data element belongs to the epoch $E$]}.

In this paper, we focus only on two signals: substream start and its termination. Tracking a start of a substream is a straightforward task: the first event of a substream will naturally trigger its start. On the contrary, generating a substream termination event is a challenging task, and various properties may be required by practical problems:
\begin{itemize}
    \item Deterministic windowed join\footnote{given the same sequences of input tuples, the same output tuples will be produced} requires an order of termination signals to respect the order of input elements (termination events from data producers)~\cite{najdataei2019stretch, gulisano2016scalejoin}.
    \item An epoch is a substream that an SPE should process atomically. A termination event for an epoch should arrive before any elements of the next epoch~\cite{2015arXiv150608603C}.
    \item State pruning problem does not require any specific properties from termination events. However, late termination event receiving may cause sub-optimal memory utilization.
\end{itemize}

A popular substream management method is the punctuations framework~\cite{tucker2003exploiting}. The main idea behind this framework is to divide the stream by injecting special elements called {\em punctuations} that define substreams ``borders''. These special elements are propagated via the same network channels as data elements. While the punctuation approach is robust and easy to implement, it has several limitations. 

Punctuations are not applicable for cyclic dataflows in a general case because elements belonging to a substream can remain in transit within a cycle for an uncertain time~\cite{carbone2018scalable}. The technique proposed in~\cite{Carbone:2017:SMA:3137765.3137777} mitigates this issue for the state snapshotting problem. The main idea of this technique is to include in a snapshot all in-transit elements (possibly from previous epochs) within a cycle and then resend them on rollback. However, it provides a solution for a specific problem that does not allow a system to determine a substream termination for cyclic dataflows using punctuations.

The high network overhead forms another limitation. Network traffic complexity for this method is $O(K|\Pi|^2)$, where $|\Pi|$ is the number of processes and $K$ is the number of substreams because each process should propagate punctuations to all output channels. This complexity boundary covers the worst case when all processes are interconnected. However, SPEs prefer to distribute the work among nodes evenly to ensure scalability~\cite{carbone2015apache, Kulkarni:2015:THS:2723372.2742788, Akidau:2013:MFS:2536222.2536229}. This load balancing implies that each process effectively occupies channels to all other processes. The worst-case complexity boundary is tight for scenarios when an execution graph contains at least one operator that repartitions data.

Substreams can be {\em fine-grained}: for example, each user session defines a substream. If there are a lot of small substreams, an inefficient substream management system can degrade the latency~\cite{DBLP:journals/pvldb/BegoliACHKKMS21} and the throughput of an SPE~\cite{Li:2008:OPN:1453856.1453890} or affect the performance of state checkpointing~\cite{zhang2021research}.

\begin{figure}[t]
  \centering
  \includegraphics[width=0.20\textwidth]{pics/tracker-scheme.pdf}
  \caption{\tracker\ framework: tracking agent aggregates information about substreams and produces NEOSS}
  \label{tracker_scheme}
\end{figure}

In this work we formalize the substream management problem and show that the network traffic overhead of the punctuations framework is far from the optimal. We also formally define properties of a substream management technique required by various problems such as state snapshotting to ensure that a newly proposed method satisfy them. 

We introduce a new substream management framework called \tracker. Figure~\ref{tracker_scheme} shows the high-level scheme of our method. 
Within this framework, we use a dedicated agent that receives information about substreams from the entire SPE and sends back {\em end-of-substream notifications} (NEOSS). 
NEOSS messages are propagated through this agent without broadcasting between processes, reducing the amount of extra traffic. Such propagation method is suitable for cyclic dataflows because there is no need to forward service traffic through the cycles.

Basic comparison between the \tracker\ framework and its alternatives is shown in Table~\ref{solutions-overview-table}. Regarding network traffic, $|\Pi|$ is the number of computational nodes and $K$ is the number of substreams. We can outline that the punctuations framework is the only substream management mechanism that supports arbitrary predicates for substreams, so we use it as a baseline approach in the experiments. The commonalities and differences between the \tracker\ framework and alternative solutions are detailed in Section~\ref{fs-acker-related}.

\begin{table}[t]
    \caption{An overview of substream management techniques}
    \label{solutions-overview-table}
    \begin{threeparttable}
        \centering
        \begin{tabular}{|>{\bfseries}c|c|c|c|c|c|} 
          \hline
          Method & Arbitrary predicates & Cycles & Traffic  \\ \hline \hline
          Punctuations & + & - & $O(K|\Pi|^2)$ \\ \hline
          MillWheel* & - & N/A & N/A \\ \hline
          Naiad* & - & + & $O(K|\Pi|^2)$ \\ \hline
          Acker & - & + & $O(K|\Pi|)$ \\ \hline
          \tracker\ & + & + & $O(K|\Pi|)$ \\ \hline
        \end{tabular}
        *progress tracker
    \end{threeparttable}
\end{table}

In summary, our contributions are as follows:
\begin{enumerate}
    \item We provide a formal model of substream management. This model allows us to compare the properties of various substream management systems.
    \item We present a novel substream management technique that achieves a lower bound of network traffic overhead.
    \item We demonstrate \tracker\ performance in comparison to a state-of-the-art approach on diverse workloads.
\end{enumerate}

The rest of the paper is organized as follows: Section~\ref{fs-acker-preliminaries} formalizes the substream management problem and indicates its main properties. In Section~\ref{fs-acker-tracker}, we introduce a general design of the \tracker\ framework and demonstrate the properties of this substream management solution. Section~\ref{fs-acker-impl} summarizes the implementation of \tracker\. In Section~\ref{fs-experiments}, we show that the proposed technique is scalable and can outperform alternatives employed in state-of-the-art stream processing engines. The relevant prior research is outlined in Section~\ref{fs-acker-related}. Finally, we discuss our conclusions in Section~\ref{fs-acker-conclusion}.

\section{Substreams management}
\label{fs-acker-preliminaries}

First, in this section, we formalize the notion of a stream processing engine, based on Chandy-Lamport definition of a distributed system and build specifics needed for stream processing on top of this model. Then we introduce substream lifespan management, based on events from the proposed model. Fnally we study what properties of the introduced    technique are needed for practical applications.

\subsection{Processing model}

\begin{figure}[htbp]
  \centering
  \includegraphics[width=0.50\textwidth]{pics/process-scheme.pdf}
  \caption{Structure of the SPE process}
  \label{fig:spe_process}
\end{figure}

Typically, distributed stream processing engines are shared-nothing runtimes that continuously ingest input elements, transform them according to a logical dataflow graph, and deliver output elements. The logical dataflow graph consists of user-defined operators. Operators can be stateless or stateful: an output element may depend on the current state and the corresponding input element. A logical graph is mapped to a physical, distributed graph upon deployment. Commonly, a single logical operator can be deployed on multiple computational nodes. Further, we denote physical instances of logical operators as {\em processes}.

A deployed physical graph is a distributed system and could be described in terms of Chandy-Lamport model~\cite{lamport}. In this model authors introduce \textit{events} that allows to observe a state of the system. Each event is presented in tuple of 5 elements $e = (p, s, s', c, M)$, where $p$ is one of the deployed processes, $s$ and $s'$ are state of the process before and after processing, $c$ one of network FIFO channels that connect processes with each other, and $M$ is a message generated during processing. The generated event $M$ come to a channel state $C$ until it will be received by destination process. Processes and channels forms a physical graph of the system $G=\{\Pi,\mathcal{E}\}$.

In a stream processing engine we need to specify a process $p$ to reflect a specifics of this type of processing. Besides user-defined state each SPE process has a special storage for messages. We call this storage \textit{sendbox} $B_p$. This storage is needed for many reasons as will be discussed later in the paper. Here we justify the existence of \textit{sendbox} by necessity to send multiple events from one user-defined model to multiple recipients which contradicts with the original model. To manage the sendbox SPEs have a special agent. In our model we extend a role of this agent to inject a substream management there. Find a schematic representation of the model in Fig.~\ref{fig:spe_process}.

With the specified process we need to introduce special cases of system events:
\begin{itemize}
    \item Communication events: $recv$, $send$, these events are executed by sendbox controller
    \item Processing events: user-defined procedures are isolated here
    \item System events: consistency, substreams and other events in the system that provide its properties
\end{itemize}

Communication events are straightforward: $<recv, M> = (sendbox\_controller_p, B_p, B_p' = B_p\cup \{M\}, c_qp, M)$, $<send, M> = (sendbox\_controller_p, B_p, B_p' = B_p\setminus\{M\}, c_{p, dst(M)}, M)$. Note that we need to be able to get destination process directly from the message and in general message contains its source and destination, this allows us to abstract away from the physical channels which are unknown to the user-defined procedure that emits the event. As a practical case of this abstraction is a sharding scheme for some key: user-defined procedure emits event for some key and a system is responsible to find a proper physical channel to deliver this message.

Processing events are triggered by a process and allows to execute user-defined logic with the events provided by sendbox controller. Please note, that these events does not communicate with other processes and are isolated in the state of the SPE process. $<proc, M> = (func_p, s\setminus M, s \cup \{func_p(M)\}, null, null>$.

System events set depends on the system and will be defined latter in the description of particular implementation.

Note that all events within the same process $p$ are totally ordered by a local causal order relation $<_p$: $e^{0}_p,e^{1}_p,...,e^{i}_p,...$. Atomicity of the events and their sequential nature allows a system to build guaranties for processing and state management.

\subsection{Predicates on system events}
As stated in the introduction we want to allow a user or a system itself to define predicates for events in the system, track their values for all messages in the system ($S = C \cup \cup_p B_P$) . This will give opportunity to define and track quantifier expressions, that could be basis of guarantee mechanisms.

\subsubsection{Soft bound}

Many applications that apply substream management systems do not require any special properties of termination events. In this case, we denote the guarantee provided by such events as {\em soft bound}, because termination events indicate only the fact that the substream ended some time ago, and other input elements could be processed after that. More formally, we define the soft bound guarantee of the termination event (end-of-substream) $\langle eoss_{soft}, pred(m)\rangle$ as follows:

\begin{align*}
\forall e^{'} = \langle proc,m\rangle, e^{'} >_p \langle eoss_{soft}, pred(m)\rangle : \neg pred(m)
\end{align*}

Figure~\ref{general_guarantees} illustrates this notion. Terms $a,b,c,d...$ denote ordered processing events of a process $p$. The substream ends after event $c$. Note that there are several other events between the end-of-substream and $c$. This is the property of a {\em soft bound} guarantee: if $\langle eoss, pred(m)\rangle$ occurs, all subsequent elements do not satisfy the predicate, but it is not necessarily the exact substream ``border''.

\begin{figure}[htbp]
  \centering
  \includegraphics[width=0.50\textwidth]{pics/general-guarantee.pdf}
  \caption{Substream management: soft bound}
  \label{general_guarantees}
\end{figure}

\subsubsection{Firm bound}

The guarantee that any new event will not satisfy the predicate is sufficient for many real-life problems, e.g., SPE can initiate process state pruning on such events. However, some problems require a {\em firm bound}: guarantee that the substream ends {\em exactly} after the termination event. 

For example, epoch-based snapshotting protocol~\cite{2015arXiv150608603C, jacques2016consistent} bases on a notion of {\em epoch}. Epoch is a special substream that should be atomically processed. Therefore, SPE requires the termination event for an epoch that occurs right after the last processing event for this epoch. Otherwise, the snapshot can be inconsistent, because it captures elements from various epochs. To support such scenarios, the end-of-substream event $\langle eoss_{firm}, pred(m)\rangle$ should satsify the following conditions:

\begin{align*}
&1. \forall e^{'} = \langle proc,m\rangle, e^{'} >_p \langle eoss_{firm}, pred(m)\rangle : \neg pred(m) \\
&\boldsymbol{2. \forall e^{*} = \langle proc,m\rangle, e^{*} <_p \langle eoss_{firm}, pred(m)\rangle : pred(m)} \\
\end{align*}

The first condition is the same as for the soft bound guarantee. The second one allows a process to determine the exact processing event when the substream terminates: all elements after the termination event will not satisfy the predicate, but all previous elements have been satisfied. 

Figure~\ref{strict_guarantees} illustrates the notion of the firm bound. As in the previous example, terms $a,b,c,d...$ denote ordered processing events of a process $p$. However, in this case, event $\langle eoss_{firm}, pred(m)\rangle$ occurs right after the substream terminates.

\begin{figure}[htbp]
  \centering
  \includegraphics[width=0.50\textwidth]{pics/strict-guarantee.pdf}
  \caption{Substream management: firm bound}
  \label{strict_guarantees}
\end{figure}

\subsubsection{Consistent termination events order}
Some specific applications, including the mentioned earlier epoch-based snapshotting method and techniques for enforcing deterministic processing~\cite{we2018adbis} require an order of termination events to be synchronized with the order of substreams endings (events from data sources). For example, if termination events are reordered, then snapshots for consecutive epochs can be inconsistent. Another example is deterministic join that also requires the defined order of termination events~\cite{gulisano2016scalejoin}.

\begin{figure}[htbp]
  \centering
  \includegraphics[width=0.50\textwidth]{pics/notifications-reordering.pdf}
  \caption{An example of termination events reordering}
  \label{notifications_reordering}
\end{figure}

Termination events reordering in case of the soft bound guarantee is illustrated in Figure~\ref{notifications_reordering}. Terms $a,b,c,d...$ denote ordered processing events of a process $p$. Although the substream containing events $a,b$ terminates earlier, the end-of-substream event for this substream occurs after the termination event for the substream containing events $d,e$. 

Let $e^{*}_1$ and $e^{*}_2$ be the last elements of substreams defined by predicates $pred_1(m)$ and $pred_2(m)$. Termination events $\langle eoss, pred_1(m)\rangle$ and $\langle eoss, pred_2(m)\rangle$ are {\em consistently ordered} iff:

\begin{align*}
e^{*}_1 >_p e^{*}_2 \Longrightarrow \langle eoss, pred_1(m)\rangle >_p \langle eoss, pred_2(m)\rangle
\end{align*}

\subsection{Punctuations framework}

\subsubsection{Framework overview}

The main idea behind the punctuations framework is to divide the stream into substreams by injection of special elements that bear predicate $punct$. Punctuations are injected directly into a system as ordinary data elements by SPE or by external data producers. The injector promises that all further produced records do not satisfy the predicate. Hence, the punctuation itself defines the ``border'' of a substream.

Figure~\ref{punctuations_scheme} illustrates the punctuations framework. Green elements indicate elements that belong to some substream, while red elements do not. As we can see, punctuations play the role of delimiter between the substream elements and all further items.

\begin{figure}[htbp]
  \centering
  \includegraphics[width=0.50\textwidth]{pics/punctuations-scheme.pdf}
  \caption{Punctuations framework: an example}
  \label{punctuations_scheme}
\end{figure}

Processes within SPE do not apply user-defined operators to punctuations. Instead, each process propagates punctuation to all outgoing channels when it receives corresponding punctuations from all input channels. If a process receives punctuations from all inputs, it is guaranteed that it will not receive elements that satisfy the predicate further due to FIFO network channels. Hence, the formal condition of soft bound termination event $\langle eoss_{soft}, pred(m)\rangle$ is the following:

\begin{align*}
& \langle eoss_{soft}, pred(m)\rangle \Longleftrightarrow \\ 
& \forall q \in I_p, \exists punct_{qp} \in B_p, \forall m\in B : \neg pred(m)
\end{align*}

To satisfy the firm bound guarantee, one needs to hold elements in the sendbox until all punctuations have arrived from all input channels. In~\cite{Carbone:2017:SMA:3137765.3137777} such behavior is called {\em watermark (punctuation) alignment}. More formally, the sendbox procedure should ensure the following order of elements processing to achieve firm bound:

\begin{align*}
& \exists q \in I_p, e = \langle recv,m_{qp} \rangle >_p e^{'} = \langle recv,punct_{qp}\rangle \Longrightarrow \\ 
& <proc, m_{qp}> >_p \langle eoss_{soft}, pred(m)\rangle
\end{align*}

If this condition is satisfied, then $\langle eoss_{firm}, pred(m)\rangle$ = $\langle eoss_{soft}, pred(m)\rangle$ for the punctuations. The punctuations framework provides consistent termination events order by design because punctuations are naturally ordered with ordinary data elements within the processes.

\subsection{Discussion}

\subsubsection{Limitations of punctuations}

\label{fs-acker-punctuations-limitations}

While the punctuations approach is robust and easy-to-implement, it has several limitations. In the punctuations framework, the information about the ending of a substream is propagated using ordinary data elements via the data flow network channels. It implies that punctuations are not applicable for cyclic dataflows because a process that receives elements from a cyclic channel will never receive punctuations from this channel~\cite{carbone2018scalable}.

The high network overhead forms another limitation. This method's amount of service traffic is $O(K||\Pi||^2)$, where $||\Pi||$ is the number of processes and $K$ is the number of substreams. As we can see, this estimation is far from the lower bound ($O(||\Pi||)$). It is quadratic in the number of processes, as each process should propagate punctuations to all output channels. 

Substreams can be {\em fine-grained}: for example, each processing key can spawn a substream within a state pruning problem. If there are a lot of small substreams, an inefficient substream management system can reduce the throughput of an SPE itself~\cite{Li:2008:OPN:1453856.1453890} or affect the performance of state checkpointing~\cite{zhang2021research}. As we demonstrate further, the punctuation technique adds significant performance overhead on regular processing for small substreams (frequent punctuations injection).

\subsubsection{Optimal traffic overhead}

A vital performance property of a substream management system is the amount of extra network traffic. Let $||\Pi||$ be a number of processes, and $K$ be a number of substreams. 

\begin{lemma}
The network overhead induced by a substream management system cannot be lower than $O(K||\Pi||)$. 
\end{lemma}
\begin{proof}
When a substream management system detects the end of a substream, it should inform processes about that. In a general case, e.g., for a state snapshotting problem, it should inform all (stateful) processes. Hence, at least one network message (termination notification) must be sent to each process for each substream.
\end{proof}

Despite this lemma's simplicity, we will further use it to figure out how far is some substream management system from this bound. We can also claim a solution as {\em optimal} if its extra traffic estimation is equal to the lower bound from the lemma. In the next sections, we demonstrate that it is possible to design a substream management system that achieves optimal network traffic overhead.


\section{Tracker framework}
\label{fs-acker-preliminaries}

\subsection{Overview}

\tracker\ framework bases on another idea of the propagation the fact that the substream ends. Instead of injecting special elements directly into the dataflow, we design a special agent (process) that:

\begin{enumerate}
    \item Receives signals that a substream terminated from data producers.
    \item Watches for in-flight elements and if they belong to some substream.
    \item Notifies dataflow processes when the substream ends {\em for them}, i.e., when they do not receive any elements which satisfy some predicate.
\end{enumerate}

\begin{figure}[htbp]
  \centering
  \includegraphics[width=0.50\textwidth]{pics/tracker-scheme.pdf}
  \caption{\tracker\ framework: an example}
  \label{tracker_scheme}
\end{figure}

The general scheme of the \tracker\ mechanism is shown in Figure~\ref{tracker_scheme}. Special (possibly distributed) {\em tracking agent} receives signals from data sources, fetches information about in-flight elements, and then decides to send notifications about substream end. Before diving into implementation details, we should answer the following questions regarding \tracker\ framework, as explained in the next subsection.

{\bf Q1 How to organize monitoring of in-flight elements?} To notify processes that a substream ends, the tracking agent should receive the corresponding signal from data producers and ensure no in-flight elements belong to the substream. 

{\bf Q2 How to reveal the exact moment when the substream ends?} Unlike punctuations, \tracker\ notifications are completely async with dataflow elements because they go through another network channel. Hence, dataflow items and notifications are not ordered, making it hard to determine an exact event that finishes a substream.

{\bf Q3 What are the functional and performance properties of \tracker?} \tracker\ framework is designed to eliminate the restrictions of punctuations framework. We should demonstrate that it is suitable for cyclic dataflows as well as can provide lower network overhead.

\subsection{Discussion}

\subsubsection{Answering Q1: How to organize monitoring of in-flight elements?}
Assume that each process sends to tracking agent the following information about each sending or receiving event $e = <action,m>$:
\begin{enumerate}
    \item $action$: send or receive.
    \item $pred(m)$: the result of applying a substream predicate.
    \item process identifier.
\end{enumerate}

Note that the result of applying predicates can be a vector if we are tracking multiple substreams. Process identifier is an optional field that can be used to send notifications independently for various processes or parts of the physical graph. We will discuss it in detail in the next section.

Using this information along with signals from data producers, the tracking agent can detect when there is a guarantee that a substream ends and send a corresponding notification $e^{n} = <send,pred(m)>$. Therefore, function $S(E_{proc})$ for \tracker\ framework is the following:

\begin{align*}
& S_{tracker}(E_{proc}) = \exists e \in E_{proc} : e = \langle recv,pred(m)\rangle_{tracker,p}
\end{align*}

Note that the correctness of the general guarantee for \tracker\ depends on the implementation of the tracking agent detailed in the next section. The formal proofs that function $S_{tracker}(E_{proc})$ along with the \tracker\ implementation satisfy the general guarantee are in the appendix~\ref{appendix:tracker-proof}.

\subsubsection{Answering Q2: How to reveal the exact moment when the substream ends?}

\subsubsection{Answering Q3: What are the functional and performance properties of \tracker?}
% To determine an exact event that finishes a substream, one can define an order between notifications and dataflow items.

\section{Tracker Implementation}
\label {fs-acker-impl}

\subsection{Centralized \tracker\ }

\subsection{Decentralized \tracker\ }


\section {Experiments}
\label {fs-acker-experiments}

We have tested the \tracker\ performance on a simple directed path graph of various length, which was shuffling processed elements between machines in each vertex. The \tracker\ of various configurations was compared with  tracking using watermarks and no tracking at all. Elements were tracked in windows of 1, 10 and 100.
Virtual machines used in experiments had single CPUs and 4 GB of RAM per machine. One machine called a Bench Stand was used to input data into the dataflow at a fixed rate while measuring the speed of it being processed via receiving output and notifications for data being processed. \tracker\ was running on machines excluded from the dataflow: a single one for centralized configuration and two for distributed configuration.

\subsection{Network traffic}

Network traffic was measured in number of separate service messages sent over the network. Local Acker was sending messages in batches.

% https://gist.github.com/faucct/032aaf6240db361d30a184b1d7bf3c8e

\subsection{Notification latency}

Notification latency was measured as a time between moments of Bench Stand receiving last elements in tracking windows and notifications for that window.

% https://gist.github.com/faucct/032aaf6240db361d30a184b1d7bf3c8e

\subsection{Scalability}

In those experiments we are reproducing a case in which the centralized \tracker\ was not holding the load, while the distributed \tracker\ was working. While using 100 machines running our dataflow we have failed to reproduce it. Still, we have been able to simulate it by increasing the number of Ack messages sent in a single batch 9 times.

% Надо подумать, какие графики тут нужны и подобрать, какую конфигурацию мы симулируем этим способом.

\subsection{Overhead on throughput}

In those experiments we show that tracking with \tracker\ in contrary to watermarks does not have a large overhead on throughput.

% Я забыл доснять эти эксперименты.

\subsection{State snapshotting}

In those experiments we are comparing granular tracking using centralized \tracker\ and watermarks. Processed elements are divided into snapshot windows. Pipeline vertices only process elements from a current snapshot window and buffer ones from a next snapshot window until they receive a notification that all elements from a current snapshot have been processed. When this happens vertices imitate snapshotting with a fixed duration sleep and continue to process elements from next snapshot window. We have measured a number of buffered elements and total time they have spent in buffer varying the snapshot duration.

% https://gist.github.com/faucct/6097d9d08197cb979b71721b16f8b6a3/

\subsection{Count iterations?}



\section{Related Work}
\label {fs-acker-related}

Most dependency tracking techniques employed in state-of-the-art stream processing systems are discussed in detail in Section~\ref{existing_solutions}. In this section we hightlight the differences between 

% Naiad~\cite{Murray:2013:NTD:2517349.2522738} uses a kind of similar to \tracker\ mechanism for tracking the progress of iterative computations. Within this method, each data item in a system is assigned with an {\em epoch} and a vector of logical timestamps called {\em loop counter}. Epoch is similar to our notion of {\em global time} concept but provided by an external user. The value on the $i$th position of the loop counter indicates the number of times this element went through the $i$th {\em loop context} (cycle) in a dataflow. Special distributed agents monitor for the items and their timestamps and notify when all elements reach some iteration number or all elements from an epoch are entirely processed. There are three main differences between the mentioned technique and the \tracker. Firstly, \tracker\ relies on the global identifier of an element provided by a system itself. Secondly, the protocol used in Naiad causes the enormous number of extra network messages that quadratically depend on the number of machines even with optimizations~\cite{Murray:2013:NTD:2517349.2522738}. Thirdly, Naiad's method uses counter updates (+1/-1) instead of XORs, hence update messages do not commutate. This fact complicates the implementation (especially distributed) of this method.

% The problems of transactional processing, providing for delivery guarantees, and fault tolerance, are extensively studied in recent years~\cite{Akidau:2013:MFS:2536222.2536229, Carbone:2017:SMA:3137765.3137777, thepaper, Wang:2019:LSF:3341301.3359653}. While state-of-the-art stream processing systems still provide high overhead on regular processing due to fault tolerance protocols, transactional processing, etc., we expect that this area will be studied further. As we mentioned above, a dependency tracking mechanism is an essential part of the solutions to these problems. Hence, \tracker\ can be applied to optimize the existing techniques.

\section {Conclusion}
\label {fs-acker-conclusion}

In this work, we formulated and formalized a problem of dependency tracking between input and output elements in streaming dataflows. We demonstrated that state-of-the-art distributed stream processing systems face this problem in state snapshotting mechanisms~\cite{Carbone:2017:SMA:3137765.3137777, apache:storm}, the materialization of time-varying relations~\cite{Begoli:2019:OSR:3299869.3314040}, and atomic delivery of all descendants of an input item~\cite{we2018adbis}.  

To solve this problem, we proposed a mechanism that adopts ideas from the Apache Storm completion tracking mechanism called \acker. We extend each data item with a logical timestamp that denotes corresponding input items and track if data flow contains elements with specific timestamps. Our solution, called \tracker, provides the following features:
\begin{itemize}
    \item {\bf Fine-grained tracking:} \tracker\ efficiently watches and provides notifications that system completely processed some set of input items even for individual input elements.
    \item {\bf Cyclic graphs support:} proposed mechanism works for cyclic execution graphs, and that makes it suitable for iterative dataflows as well. 
    \item {\bf Scalability:} we introduced a decentralized version of \tracker\ that allows a system to distribute extra network traffic between all computational units. 
    \item {\bf Low overhead:} \tracker\ does not produce any significant performance penalty and does not affect the throughput of a distributed streaming dataflow.
\end{itemize}

We conducted a series of experiments and compared the proposed method with a baseline approach based on the checkpointing mechanism used in Apache Flink. We demonstrated that both centralized and decentralized implementations of \tracker\ provide lower notification latency that does not considerably degrade with an increase of a logical graph size or a cluster size. Experiments also showed that \tracker\ has lower throughput overhead in case of fine-grained tracking.

\appendix 
\section {Punctuations} \label{appendix:punctuations-proof}
\section {\tracker} \label{appendix:tracker-proof}

\bibliographystyle{abbrvurl}
% \bibliography{bibliography/flame-stream}
\bibliography{../../bibliography/flame-stream}

\end {document}

\endinput
