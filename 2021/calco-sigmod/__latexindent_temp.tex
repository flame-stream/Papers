To be able to automatically generate execution graphs we need to know requirements and impact of each operation of the graph.
There are two ways to achieve it.
The first one is to provide user a finite set of operations which semantics is defined by the graph specification library.
It is the way that SQL does.
But as we saw earlier this approach is not flexible enough.
The second one is to manually provide specification for a user-defined operations.
We present operation semantics annotations that we call contracts to be able to provide such specifications for user-defined operations.
Such graph definition based on contracts we call CGraph.

TODO

\subsection{CGraph}

Let us define an environment as a set of nodes, annotated with contracts (that we will define later).
Contracts have two types: input contracts (InCont) and output contracts (OutCont).
The first one define requirements that input data elements should satisfy.
The second one define guaranties that operation sets for the output.
Set consists of the data sources, annotated with OutCont, one arity transformations (with one InCont and OutCont) and two arity transformations (with two InCont and OutCont).

\begin{lstlisting}[language=Haskell]
type CSource = OutCont
type CTfm1 = (InCont, OutCont)
type CTfm2 = (InCont, InCont, OutCont)

data Env = Env
  { sources :: Map NodeName CSource
  , tfms1   :: Map NodeName CTfm1
  , tfms2   :: Map NodeName CTfm2
  }
\end{lstlisting}

Graph gets data from the data sources, transforms it in transformation nodes.
But some of the transformation nodes do particular side effects that form result of the graph execution (writes data to the some storage, display on the dashboard, etc.).
Let's call the set of the nodes with the such side effects as graph semantics.
It will let us to have useless nodes in environment and reduce the total number of graphs to enumerate.

\begin{lstlisting}[language=Haskell]
type Semantics = Set NodeNames
\end{lstlisting}

And now we are ready to define CGraph.
It is the pair of environment and semantics.

\begin{lstlisting}[language=Haskell]
type CGraph = (Env, Semantics)
\end{lstlisting}

CTInfo --- information available before runtime time (construct time) (filters and joins).
RTInfo --- information extracted from runtime statistics.
So the work pipeline can be the following:

\begin{lstlisting}[language=Haskell]
cgraph :: CGraph
ctinfo :: CTInfo

genGraphs :: CGraph -> [Graph]

graphs :: [Graph]
graphs = genGraphs cgraph

cost :: CTInfo -> Maybe RTInfo
     -> Graph -> Integer

chooseGraph :: (Graph -> Integer) -> [Graph] -> Graph


eval :: Graph -> Runtime Graph
cost :: Graph
     -> Maybe RTInfo -> CTInfo ->
     -> Integer

\end{lstlisting}

TODO

\subsection{Contracts}

TODO verification: Hoare logic \\
TODO

\subsection{Graphs}

TODO

\subsection{Prototype implementation}

TODO

TODO migrate prototype to the its own repo
