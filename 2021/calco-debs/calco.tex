\documentclass[sigconf]{acmart}

\usepackage{graphicx}
\usepackage{algorithm} % for algorithms
\usepackage{algpseudocode}
\usepackage{booktabs} % For formal tables
\usepackage{amsthm} % For claims
\usepackage{bbm} % indicator function
\usepackage{acmcopyright}

% listings
\usepackage{xcolor,listings}
\usepackage{textcomp}
\lstset{upquote=true}

% plots
\usepackage{pgfplots}

% table
\usepackage[flushleft]{threeparttable} % http://ctan.org/pkg/threeparttable
\usepackage{booktabs,caption}

\theoremstyle{remark}

\settopmatter{printacmref=false, printccs=true, printfolios=true}
\pagestyle{empty} % removes running headers

\newcommand{\PicScale}{0.5}
\newcommand {\FlameStream} {FlameStream}
\begin{document}

% \copyrightyear{2019}
% \acmYear{2019}
% \setcopyright{rightsretained}
% \acmConference[DEBS '19]{DEBS '19: The 13th ACM International Conference on Distributed and Event-based Systems}{June 24--28, 2019}{Darmstadt, Germany}
% \acmBooktitle{DEBS '19: The 13th ACM International Conference on Distributed and Event-based Systems (DEBS '19), June 24--28, 2019, Darmstadt, Germany}\acmDOI{10.1145/3328905.3332514}
% \acmISBN{978-1-4503-6794-3/19/06}

% TODO dataflows: is space needed
\title{Calco: a contract-based approach to specify distributed dataflows}

% \author{Alexander Chernokoz, Darya Sharkova, Artem Trofimov, Nikita Sokolov, Ekaterina Gorshkova, and Boris Novikov}
% \affiliation{%
% \institution{$^5$ ITMO University}
%   \city{St. Petersburg}
%   \country{Russia}
% }
% \affiliation{%
% \institution{$^1$Higher School of Economics}
% }
% \author{Artem Trofimov,$^ {1,2}$    Mikhail Shavkunov,$^3$    Sergey Reznick,$^4$     Nikita Sokolov,$^{5}$   Mikhail Yutman,$^3$ \\   Igor E. Kuralenok,$^1$    and  Boris Novikov$^ {3}$ }
% \affiliation{%
% \institution{$^1$JetBrains Research}
% }
% \affiliation{%
% \institution{$^2$Saint Petersburg State University}
% }
% \affiliation{%
% \institution{$^3$National Research University Higher School of Economics}
% }
% \affiliation{%
% \institution{$^4$ Kofax}
% }
% \affiliation{%
% \institution{$^5$ ITMO University}
%   \city{St. Petersburg}
%   \country{Russia}
% }
% \email{\string{trofimov9artem, mv.shavkunov, sergey.reznick, faucct, myutman, ikuralenok\string}@gmail.com, borisnov@acm.org}

\begin{abstract}

    There are two general ways to define computations: imperative and declarative.
    The first one is more transparent for programmers, while the second one is more suitable for complex optimizations.
    One declarative definition can have multiple implementations so that a system automatically chooses the most optimal one.
    Accordingly, distributed dataflow can be specified in two ways: defining concrete execution graph and SQL.
    A concrete graph does not capture high-level information about the problem it solves, so a distributed processing system cannot permute operations for performance purposes.
    Hence, graph optimization is generally a programmer concern, but real-world graphs can consist of many nodes that make it hard to optimize them manually.
    Moreover, most of the necessary information for the graph cost evaluation is available only in runtime.
    SQL is popular for data analytics, and its optimization is well-researched.
    However, it is inconvenient or sometimes impossible to use SQL for general data management tasks, such as ETL, machine learning pipelines, etc.

    In this work, we introduce a novel approach to specify distributed dataflows.
    Our method is based on declarative specifications of user-defined operations that we call contracts.
    Such specifications allow us to automatically generate execution graphs to choose the most optimal one.
    A contract can describe an arbitrary operation or a dataflow part, so this approach combines the transparency of the imperative approach and optimization possibilities of the declarative one.
    We implement a prototype and demonstrate automatic graphs generation on a real-world dataflow.
    We also outline the challenges that we face regarding the optimization problem.

\end{abstract}

% \begin{CCSXML}
% \begin{CCSXML}
% \begin{CCSXML}
% <ccs2012>
% <concept>
% <concept_id>10002951.10002952.10002953.10010820.10003208</concept_id>
% <concept_desc>Information systems~Data streams</concept_desc>
% <concept_significance>500</concept_significance>
% </concept>
% <concept>
% <concept_id>10002951.10003317.10003347.10003356</concept_id>
% <concept_desc>Information systems~Clustering and classification</concept_desc>
% <concept_significance>500</concept_significance>
% </concept>
% <concept>
% <concept_id>10002951.10003227.10003351.10003446</concept_id>
% <concept_desc>Information systems~Data stream mining</concept_desc>
% <concept_significance>300</concept_significance>
% </concept>
% </ccs2012>
% \end{CCSXML}

% \ccsdesc[500]{Information systems~Data streams}
% \ccsdesc[500]{Information systems~Clustering and classification}
% \ccsdesc[300]{Information systems~Data stream mining}

% \keywords{Data streams, text classification, reproducibility, exactly once}

\maketitle

\thispagestyle{empty}

\section{Introduction}
TODO cross domain optimization

TODO SQL is just not enough

TODO intersecting queries optimization (cross-query optimization)

TODO runtime information is needed for optimization, so it should be done automatically

TODO too many concrete graphs for the simple examples to code them manually

TODO

TODO

\section{Running example}
To consider possibilities and limitations of the our approach later we introduce an example machine learning streaming problem.

Let's suppose that we have an application that consists of the frontend and backend sides.
They both log user's queries and send logs to the log service.
For simplicity, let's assume that every user query produces one log entity from the frontend and one log entity from the backend, and they have equal and unique query id.

The log entities have the following structure (ts stands for the timestamp):

\begin{tabular}{|l|llll|}
    \hline
    \textbf{frontend} & version & queryId & userId & ts \\
    \hline
\end{tabular}

\vspace{0.1em}

\begin{tabular}{|l|lllll|}
    \hline
    \textbf{backend} & id & queryId & userId & ts & payload \\
    \hline
\end{tabular}

We want to calculate statistics about the time that logged users with new frontend client versions wait for the application answer per user session.
To evaluate when the session ends we will use machine learning model that requires frontend features and users features.

One of the possible solutions is represented on the image TODO. % TODO image ref
Obviously, this solution is not optimal.
Filters are applied to streams too late and number of partitions potentially can be reduced.
To be able to fix it safely we need to know about the requirements and impact of the each operation.

\begin{itemize}
    \item \textbf{joinByQueryId}: Requires queryId field in the elements of both streams.
    \item \textbf{addUsersFeatures}: Requires partition by userId. Users should not be filtered before to get valid statistics. Adds userFeatures field to elements.
    \item \textbf{addFrontFeatures}: Requires only new front versions, adds frontFeatures field to elements in the stream.
    \item \textbf{filterNewFronts}: Leaves only elements with new front versions.
    \item \textbf{modelInference}: Requires front and users features. Sets trigger that signals when to emit data from aggregation.
    \item \textbf{filterAuthorizedUsers}: Requires users features. Filters users that are authorized.
    \item \textbf{stats}: Requires authorized users from new front versions. Also needs session trigger to be set. Sends accumulated statistics to the dashboard.
\end{itemize}

% TODO SQL proof
We see here complex pipeline with stateful operations and streaming triggers management.
It is not possible to describe itin SQL because of it.
Moreover, we see non-trivial operation requirements here, so concrete execution graph specifying will not provide space for optimizations.

% TODO too much lets
Let's consider another pipeline that satisfies requirements: TODO. % TODO image ref
We can see that filters here are maximally close to data sources.
And after permutation of stateful operations we have one less data partition that is rather expensive operation.

In section TODO % TODO number of section
we specify this task using contract-based approach and see, how it helps in optimizations.

% TODO insert properly
% TODO ops naming
\begin{figure}
    \label{fig:running-example-suboptimal}
    \label{fig:running-example-optimal}
    \includegraphics[width=\linewidth]{images/debs-calco-example}
\end{figure}


\section{Calco overview}
% TODO Calco haskell
TODO

\section{Example evaluation}
% TODO Way of reasoning about problems by contracts in declarative way
% TODO Optimal graph in the generated set
% TODO Examples of graphs
% TODO Second example from SQL bench
% TODO Graphs generation time
TODO

\section{Discussion and future work}
% TODO General implementation
% TODO Better contracts interface (trello)
% TODO Smart advices

% TODO Preliminary cost evaluation
% TODO Runtime statistics gathering
% TODO Cost evaluation by statistics analysis
% TODO Actor entity that decides, whether to restructure execution graph
% TODO How to restructure execution graph in runtime

% TODO Only map-like elements in the stream
TODO

\section{Related work}
% TODO machine learning
% TODO graph queries processing
TODO

\section{Conclusion}
We presented a novel approach to specify distributed dataflows.
It is based on the contracts, preconditions and postconditions of the graph operations.
We considered a small machine leaning streaming pipeline example, discussed ways of its optimization and saw that our approach defines the most optimal execution graph too.
% TODO We demonstrated the prototype?
Also we outlined optimization challenges that we faced.
Finally, we showed ideas for the future work enhancements.

We proposed to specify dataflow as CGraph (contracted graph).
CGraph is a pair of environment and semantics, where environment maps nodes to their contracts, and semantics is a set of nodes that produce dataflow target side-effects.
We provide an algorithm that generates all concrete execution graphs that satisfy contracts and semantics.

We discussed the example that SQL implementation seems to be unnatural.
And the only possible way to specify it is execution graph, that represents rather imperative approach then declarative, hence do not provide enough space for optimizations.
We specified the example problem by CGraph and pointed out that the most optimal graph exists in the set of concrete execution graphs, generated from CGraph.
% TODO Way of reasoning about problems by contracts in declarative way

We pointed out the optimization problems that we face.
% TODO Preliminary cost evaluation
% TODO Runtime statistics gathering
% TODO Cost evaluation by statistics analysis
% TODO Actor entity that decides, whether to restructure execution graph
% TODO How to restructure execution graph in runtime

In future TODO
% TODO General implementation
% TODO Better contracts interface (trello)
% TODO Smart advices


\bibliographystyle{ACM-Reference-Format}
\bibliography{../../bibliography/flame-stream}

\end{document}

\endinput
