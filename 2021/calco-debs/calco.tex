\documentclass[sigconf]{acmart}

\usepackage{graphicx}
\usepackage{algorithm} % for algorithms
\usepackage{algpseudocode}
\usepackage{booktabs} % For formal tables
\usepackage{amsthm} % For claims
\usepackage{bbm} % indicator function
\usepackage{acmcopyright}

% listings
\usepackage{xcolor,listings}
\usepackage{textcomp}
\lstset{upquote=true}

% plots
\usepackage{pgfplots}

% table
\usepackage[flushleft]{threeparttable} % http://ctan.org/pkg/threeparttable
\usepackage{booktabs,caption}

\theoremstyle{remark}

\settopmatter{printacmref=false, printccs=true, printfolios=true}
\pagestyle{empty} % removes running headers

\newcommand{\PicScale}{0.5}
\newcommand {\FlameStream} {FlameStream}
\begin{document}

% \copyrightyear{2019}
% \acmYear{2019}
% \setcopyright{rightsretained}
% \acmConference[DEBS '19]{DEBS '19: The 13th ACM International Conference on Distributed and Event-based Systems}{June 24--28, 2019}{Darmstadt, Germany}
% \acmBooktitle{DEBS '19: The 13th ACM International Conference on Distributed and Event-based Systems (DEBS '19), June 24--28, 2019, Darmstadt, Germany}\acmDOI{10.1145/3328905.3332514}
% \acmISBN{978-1-4503-6794-3/19/06}

% TODO dataflows: is space needed
\title{Calco: a contract-based approach to specify distributed dataflows}

% \author{Alexander Chernokoz, Darya Sharkova, Artem Trofimov, Nikita Sokolov, Ekaterina Gorshkova, and Boris Novikov}
% \affiliation{%
% \institution{$^5$ ITMO University}
%   \city{St. Petersburg}
%   \country{Russia}
% }
% \affiliation{%
% \institution{$^1$Higher School of Economics}
% }
% \author{Artem Trofimov,$^ {1,2}$    Mikhail Shavkunov,$^3$    Sergey Reznick,$^4$     Nikita Sokolov,$^{5}$   Mikhail Yutman,$^3$ \\   Igor E. Kuralenok,$^1$    and  Boris Novikov$^ {3}$ }
% \affiliation{%
% \institution{$^1$JetBrains Research}
% }
% \affiliation{%
% \institution{$^2$Saint Petersburg State University}
% }
% \affiliation{%
% \institution{$^3$National Research University Higher School of Economics}
% }
% \affiliation{%
% \institution{$^4$ Kofax}
% }
% \affiliation{%
% \institution{$^5$ ITMO University}
%   \city{St. Petersburg}
%   \country{Russia}
% }
% \email{\string{trofimov9artem, mv.shavkunov, sergey.reznick, faucct, myutman, ikuralenok\string}@gmail.com, borisnov@acm.org}

\begin{abstract}

    There are two general ways to define computations: imperative and declarative.
    The first one is more transparent for programmers, while the second one is more suitable for complex optimizations.
    One declarative definition can have multiple implementations so that a system automatically chooses the most optimal one.
    Accordingly, distributed dataflow can be specified in two ways: defining concrete execution graph and SQL.
    A concrete graph does not capture high-level information about the problem it solves, so a distributed processing system cannot permute operations for performance purposes.
    Hence, graph optimization is generally a programmer concern, but real-world graphs can consist of many nodes that make it hard to optimize them manually.
    Moreover, most of the necessary information for the graph cost evaluation is available only in runtime.
    SQL is popular for data analytics, and its optimization is well-researched.
    However, it is inconvenient or sometimes impossible to use SQL for general data management tasks, such as ETL, machine learning pipelines, etc.

    In this work, we introduce a novel approach to specify distributed dataflows.
    Our method is based on declarative specifications of user-defined operations that we call contracts.
    Such specifications allow us to automatically generate execution graphs to choose the most optimal one.
    A contract can describe an arbitrary operation or a dataflow part, so this approach combines the transparency of the imperative approach and optimization possibilities of the declarative one.
    We implement a prototype and demonstrate automatic graphs generation on a real-world dataflow.
    We also outline the challenges that we face regarding the optimization problem.

\end{abstract}

% \begin{CCSXML}
% \begin{CCSXML}
% \begin{CCSXML}
% <ccs2012>
% <concept>
% <concept_id>10002951.10002952.10002953.10010820.10003208</concept_id>
% <concept_desc>Information systems~Data streams</concept_desc>
% <concept_significance>500</concept_significance>
% </concept>
% <concept>
% <concept_id>10002951.10003317.10003347.10003356</concept_id>
% <concept_desc>Information systems~Clustering and classification</concept_desc>
% <concept_significance>500</concept_significance>
% </concept>
% <concept>
% <concept_id>10002951.10003227.10003351.10003446</concept_id>
% <concept_desc>Information systems~Data stream mining</concept_desc>
% <concept_significance>300</concept_significance>
% </concept>
% </ccs2012>
% \end{CCSXML}

% \ccsdesc[500]{Information systems~Data streams}
% \ccsdesc[500]{Information systems~Clustering and classification}
% \ccsdesc[300]{Information systems~Data stream mining}

% \keywords{Data streams, text classification, reproducibility, exactly once}

\maketitle

\thispagestyle{empty}

\section{Introduction}
There are two common options to define dataflows in state-of-the-art distributed stream processing systems.
The first option is defining a logical {\em execution graph}.
An execution graph is a directed graph, where nodes represent operations and edges denote data flows.
This mechanism is robust and suitable for complex dataflows (TODO synonym) but has limited optimization abilities.
A processing system can apply only local physical optimizations because it cannot ensure that the restructured graph is equivalent to the original one.

Another way is declarative: the user defines the result that he aims to obtain, and the execution graph is generated automatically by the processing system.
The declarative approach is commonly implemented using SQL.
SQL is based on relational algebra that includes a set of operations and query transformation rules.
This way, a system can obtain multiple equivalent execution graphs and choose optimal one using a {\em cost model}.
Unfortunately, SQL is not rich enough to express some user-defined operations, e.g., complex machine learning pipelines~\cite{PROOF} (TODO).

In this work, we present a declarative framework called {\em Calco} to specify distributed dataflows.
Our framework is bases on the ideas of the contract programming~\cite{REF} (TODO) and aim to solve the following problems:

{\bf Custom dataflows optimization}: user can annotate a custom operation with {\em contracts} which captures operation properties and allows the system to apply global optimization, e.g., to permute such operations.

{\bf Cross-domain optimization}: SQL can be automatically translated into Calco contracts and optimized together with the custom user-defined operations.

Optimization problem can be separated into two tasks: equivalent graphs generation and developing a cost model.
In this work we focus on the first one.

TODO

\section{Running example}
TODO SQL cross query optimization
TODO \\


\section{Calco overview}
% TODO Calco haskell
TODO

\section{Example evaluation}
% TODO Way of reasoning about problems by contracts in declarative way
% TODO Optimal graph in the generated set
% TODO Examples of graphs
% TODO Second example from SQL bench
% TODO Graphs generation time
TODO

\section{Discussion and future work}
% TODO General implementation
% TODO Better contracts interface (trello)
% TODO Smart advices

% TODO Preliminary cost evaluation
% TODO Runtime statistics gathering
% TODO Cost evaluation by statistics analysis
% TODO Actor entity that decides, whether to restructure execution graph
% TODO How to restructure execution graph in runtime
TODO

\section{Related work}
% TODO machine learning
% TODO graph queries processing
TODO

\section{Conclusion}
% We have observed existing ways to specify computations in distributed data processing frameworks: concrete graph definition and SQL.
% Also we have outlined disadvantages and limitations of these approaches with explanations.

% We have presented a novel approach to specify distributed dataflows.
% It is based on the contracts, preconditions, and postconditions of the custom graph operations (their semantics).
% Such specification has been called CGraph.
% CGraph is a pair of environment and semantics, where the environment maps nodes to their contracts, and semantics is a set of nodes that produce dataflow target side-effects.
% This representation allows us to generate all possible graphs with the needed semantics using defined operations.
% So having a cost evaluation function we can choose the most optimal graph.

% TODO example

% We have discussed challenges about making CGraph specification API comfortable enough to solve real-world tasks and maintain these solutions.
% Finally, we have studied problems that we will encounter in future work regarding the optimization: cost evaluation and the adaptivity loop implementation.

TODO


\bibliographystyle{ACM-Reference-Format}
\bibliography{../../bibliography/flame-stream}

\end{document}

\endinput
