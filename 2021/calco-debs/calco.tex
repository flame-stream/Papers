\documentclass[sigconf]{acmart}

\usepackage{graphicx}
\usepackage{algorithm} % for algorithms
\usepackage{algpseudocode}
\usepackage{booktabs} % For formal tables
\usepackage{amsthm} % For claims
\usepackage{bbm} % indicator function
\usepackage{acmcopyright}

% listings
\usepackage{xcolor,listings}
\usepackage{textcomp}
\lstset{upquote=true}

% plots
\usepackage{pgfplots}

% table
\usepackage[flushleft]{threeparttable} % http://ctan.org/pkg/threeparttable
\usepackage{booktabs,caption}

\theoremstyle{remark}

\settopmatter{printacmref=false, printccs=true, printfolios=true}
\pagestyle{empty} % removes running headers

\newcommand{\PicScale}{0.5}
\newcommand {\FlameStream} {FlameStream}
\begin{document}

% \copyrightyear{2019}
% \acmYear{2019}
% \setcopyright{rightsretained}
% \acmConference[DEBS '19]{DEBS '19: The 13th ACM International Conference on Distributed and Event-based Systems}{June 24--28, 2019}{Darmstadt, Germany}
% \acmBooktitle{DEBS '19: The 13th ACM International Conference on Distributed and Event-based Systems (DEBS '19), June 24--28, 2019, Darmstadt, Germany}\acmDOI{10.1145/3328905.3332514}
% \acmISBN{978-1-4503-6794-3/19/06}

\title{Calco: a contract-based approach to specify distributed dataflows} % TODO dataflows: is space needed

% \author{Alexander Chernokoz, Darya Sharkova, Artem Trofimov, Nikita Sokolov, Ekaterina Gorshkova, and Boris Novikov}
% \affiliation{%
% \institution{$^5$ ITMO University}
%   \city{St. Petersburg}
%   \country{Russia}
% }
% \affiliation{%
% \institution{$^1$Higher School of Economics}
% }
% \author{Artem Trofimov,$^ {1,2}$    Mikhail Shavkunov,$^3$    Sergey Reznick,$^4$     Nikita Sokolov,$^{5}$   Mikhail Yutman,$^3$ \\   Igor E. Kuralenok,$^1$    and  Boris Novikov$^ {3}$ }
% \affiliation{%
% \institution{$^1$JetBrains Research}
% }
% \affiliation{%
% \institution{$^2$Saint Petersburg State University}
% }
% \affiliation{%
% \institution{$^3$National Research University Higher School of Economics}
% }
% \affiliation{%
% \institution{$^4$ Kofax}
% }
% \affiliation{%
% \institution{$^5$ ITMO University}
%   \city{St. Petersburg}
%   \country{Russia}
% }
% \email{\string{trofimov9artem, mv.shavkunov, sergey.reznick, faucct, myutman, ikuralenok\string}@gmail.com, borisnov@acm.org}

\begin{abstract}

    There are two general ways to define computations: imperative and declarative.
    The first one is more transparent for programmers, while the second one is more suitable for complex optimizations.
    One declarative definition can have multiple implementations so that a system automatically chooses the most optimal one.
    Accordingly, distributed dataflow can be specified in two ways: defining concrete execution graph and SQL.
    A concrete graph does not capture high-level information about the problem it solves, so a distributed processing system cannot permute operations for performance purposes.
    Hence, graph optimization is generally a programmer concern, but real-world graphs can consist of many nodes that make it hard to optimize them manually.
    Moreover, most of the necessary information for the graph cost evaluation is available only in runtime.
    SQL is popular for data analytics, and its optimization is well-researched.
    However, it is inconvenient or sometimes impossible to use SQL for general data management tasks, such as ETL, machine learning pipelines, etc.

    In this work, we introduce a novel approach to specify distributed dataflows.
    Our method is based on declarative specifications of user-defined operations that we call contracts.
    Such specifications allow us to automatically generate execution graphs and safely permute operations to choose the most optimal graph.
    A contract can describe an arbitrary operation or a dataflow part, so this approach combines the transparency of imperative approach and optimization possibilities of declarative one.
    We implement a prototype and demonstrate automatic graphs generation on a real-world dataflow.
    We also outline the challenges that we face regarding the optimization problem.

\end{abstract}

% \begin{CCSXML}
% \begin{CCSXML}
% \begin{CCSXML}
% <ccs2012>
% <concept>
% <concept_id>10002951.10002952.10002953.10010820.10003208</concept_id>
% <concept_desc>Information systems~Data streams</concept_desc>
% <concept_significance>500</concept_significance>
% </concept>
% <concept>
% <concept_id>10002951.10003317.10003347.10003356</concept_id>
% <concept_desc>Information systems~Clustering and classification</concept_desc>
% <concept_significance>500</concept_significance>
% </concept>
% <concept>
% <concept_id>10002951.10003227.10003351.10003446</concept_id>
% <concept_desc>Information systems~Data stream mining</concept_desc>
% <concept_significance>300</concept_significance>
% </concept>
% </ccs2012>
% \end{CCSXML}

% \ccsdesc[500]{Information systems~Data streams}
% \ccsdesc[500]{Information systems~Clustering and classification}
% \ccsdesc[300]{Information systems~Data stream mining}

% \keywords{Data streams, text classification, reproducibility, exactly once}

\maketitle

\thispagestyle{empty}

\section{Conclusion}
We presented a novel approach to specify distributed dataflows.
It is based on the contracts, preconditions and postconditions of the graph operations.
We considered a small machine leaning streaming pipeline example, discussed ways of its optimization and saw that our approach defines the most optimal execution graph too.
% TODO We demonstrated the prototype?
Also we outlined optimization challenges that we faced.
Finally, we showed ideas for the future work enhancements.

We proposed to specify dataflow as CGraph (contracted graph).
CGraph is a pair of environment and semantics, where environment maps nodes to their contracts, and semantics is a set of nodes that produce dataflow target side-effects.
We provide an algorithm that generates all concrete execution graphs that satisfy contracts and semantics.

We discussed the example that SQL implementation seems to be unnatural.
And the only possible way to specify it is execution graph, that represents rather imperative approach then declarative, hence do not provide enough space for optimizations.
We specified the example problem by CGraph and pointed out that the most optimal graph exists in the set of concrete execution graphs, generated from CGraph.
% TODO Way of reasoning about problems by contracts in declarative way

We pointed out the optimization problems that we face.
% TODO Preliminary cost evaluation
% TODO Runtime statistics gathering
% TODO Cost evaluation by statistics analysis
% TODO Actor entity that decides, whether to restructure execution graph
% TODO How to restructure execution graph in runtime

In future TODO
% TODO General implementation
% TODO Better contracts interface (trello)
% TODO Smart advices


\bibliographystyle{ACM-Reference-Format}
\bibliography{../../bibliography/flame-stream}

\end{document}

\endinput
