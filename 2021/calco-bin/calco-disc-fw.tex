In this section we discuss advantages and concerns of the proposed contract-based approach.

\subsection{Eto horosho}

We see that CGraph specification duplicates constraints and impact of operations that we decided to use to solve the problem.
As the programmer anyway should take this information into consideration, so it is not hard to write contracts and they describe available operations naturally.

Also we see that as we can generate all possible graphs using provided CGraph, we can choose the most optimal one if we can evaluate cost.

TODO

\subsection{Contract specification interface}

It is not convenient to write all contracts manually.
For example if some operations need incoming streams not to be filtered, adding new filter property make programmer to add it to this operations manually.


TODO

\subsection{Optimization}



TODO General implementation \\
TODO Better contracts interface (trello) \\
TODO Smart advices \\
 \\
TODO Preliminary cost evaluation \\
TODO Runtime statistics gathering \\
TODO Cost evaluation by statistics analysis \\
TODO Actor entity that decides, whether to restructure execution graph \\
TODO How to restructure execution graph in runtime \\
 \\
TODO Only map-like elements in the stream \\
TODO
