% We propose to define a distributed dataflow by specifying a set of equivalent execution graphs using a representation called {\em CGraph}.
% An optimal graph can be chosen from this set using a cost-model.

\subsection{Abstract contracts}

% Graph operations can be user-defined, so we need to add more information about an operation to be aware of the graph equivalency.
% Such specifications we call contracts.
% Input contract specifies requirements that input data should satisfy, output contract defines how an operation changes the input data.

Graph operations can be user-defined, hence we cannot deduce their properties automatically to be able transform the graph preserving its equivalency.
So we need to ask the user to specify such properties manually.
Such specifications we call contracts.
We define them abstractly to be able to vary their implementations.

There are two kinds of contracts: input and output.
Input contract specifies requirements that input data should satisfy, output contract defines how an operation changes the input data.
We call an information about the data as a {\em data state}.

Input contract is a data type with the {\em match} function defined: it takes an input data state, input contract and returns boolean true if an input data state satisfies an input contract.
So the type class for input contract is defined as follows:

\begin{lstlisting}[language=Haskell]
class InCont s i where
  match :: s -> i -> Bool
\end{lstlisting}

Graph nodes can be of three types: data sources, one-arity operations and two-arity operations.
So there are three definitions of output contracts.
Output contract of the data source should have function to get a data state from it.

\begin{lstlisting}[language=Haskell]
class OutCont s o where
  toState :: o -> s
\end{lstlisting}

Output contract of the one-arity operation defines, how the operation changes the input data, hence there should be a function {\em update1} that takes an input data state, an output contract and returns an output data state:

\begin{lstlisting}[language=Haskell]
class OutCont1 s o where
  update1 :: s -> o -> s
\end{lstlisting}

Output contract of the two-arity operation defines how the node transforms the data of two sources:

\begin{lstlisting}[language=Haskell]
class OutCont2 s o where
  update2 :: (s, s) -> o -> s
\end{lstlisting}

TODO

\subsection{Contracts implementation example}

% Input and output data can be described using two sorts of information: attributes that every data element has and properties of such attributes.
% Input contract consists of three sets: the required attributes, the required properties, and the prohibited properties.
% Output contract consists of two sets: the attributes and the new properties.
% Properties are denoted just as strings, like {\em "reliablePerson"} or {\em "popularItem"}.
% We say that contracts are satisfied if output contracts match the input contracts within all connected operations in the execution graph.

Input and output data can be described using two sorts of information: attributes that every data element has and properties of such attributes.
Properties are denoted just as strings, like {\em "reliablePerson"} or {\em "popularItem"}.

\begin{lstlisting}[language=Haskell]
type Attr = String
type Prop = String

data State = State
  { attrs :: Set Attr
  , props :: Set Prop }

empty :: State
empty = State { attrs = Set.empty
              , props = Set.empty }

union :: State -> State -> State
union s1 s2 = State
  { attrs s1 `Set.union` attrs s2
  , props s1 `Set.union` props s2 }
\end{lstlisting}

% Input contract can be defined as tuple of three sets:
% \begin{enumerate}
%     \item set of attributes that are required in the input stream,
%     \item set of properties that are required in the input stream,
%     \item set of properties that are prohibited in the input stream.
% \end{enumerate}

Input contract consists of three sets: the required attributes, the required properties, and the prohibited properties:

\begin{lstlisting}[language=Haskell]
data InContImpl = InContImpl
  { attrsI  :: Set Attr
  , propsI  :: Set Prop
  , propsI' :: Set Prop }

instance InCont State InContImpl where
  match :: State -> InContImpl -> Bool
  match s i =
       attrsI c `Set.isSubsetOf` attrs s
    && propsI c `Set.isSubsetOf` props s
    && propsI' c `Set.disjoint`  props s
\end{lstlisting}

Output contract consists of two sets: the new attributes and the new properties.

\begin{lstlisting}[language=Haskell]
data OutContImpl = OutContImpl
  { attrsO :: Set Attr
  , propsO :: Set Prop }

instance OutCont State OutContImpl where
  toState :: OutContImpl -> State
  toState o = update1 empty

instance OutCont1 State OutContImpl where
  update1 :: State -> OutContImpl -> State
  update1 s o = State
    { attrs = attrs s `Set.union` attrsO o
    , props = Set.union
        (propsO c)
        (props s `Set.difference` propsO' c)
    }

instance OutCont2 State OutContImpl where
  update2 :: (State, State)
          -> OutContImpl -> State
  update2 (s1, s2) o =
    update1 (s1 `Set.union` s2)
\end{lstlisting}

TODO

\subsection{CGraph}

{\em Environment} is a set of {\em nodes}, annotated with {\em input} and {\em output contracts}.
Nodes can be data sources, one-arity operations, and two-arity operations.

\begin{lstlisting}[language=Haskell]
data Env i o o1 o2 = Env
  { sources :: Map NodeName o
  , ops1 :: Map NodeName (i, o1)
  , ops2 :: Map NodeName (i, i, o2) }
\end{lstlisting}

Some nodes produce particular side effects that form the result of the running graph (writes data to the storage, displays some statistics on the dashboard, etc.).
Such nodes should exist in all generated execution graphs.
We call the nodes that produce such side effects a {\em graph semantics}.

\begin{lstlisting}[language=Haskell]
type Semantics = Set NodeName
\end{lstlisting}

CGraph is simply a pair of environment and semantics.

\begin{lstlisting}[language=Haskell]
type CGraph = (Env, Semantics)
\end{lstlisting}

Graphs with satisfied contracts that include all semantics nodes form the desired set of equivalent graphs. The implementation of the execution graphs generation algorithm is rather effective despite the brute force approach because graphs that do not satisfy contracts can be ejected. The algorithm is exponential in the size of the semantics set, which is often small enough.

TODO

\subsection{Work scheme}

All concrete graphs that correspond to the given CGraph can be generated.
So having the cost function we can choose the most optimal one and run it.
But having the execution statistics from the runtime, cost function can be much more precise.
Thus we should continuously recompute the concrete graphs costs using the updating statistics.
When the running graph becomes enough less optimal then another graph to compensate the reconfiguration cost, running graph should be dynamically reconfigured.

\subsection{Prototype}

We have implemented a prototype in Haskell.
It consists of:
\begin{enumerate}
    \item contracts definition and CGraph definition,
    \item function that checks if a concrete graph corresponds to the given CGraph,
    \item function that generates all concrete graph that correspond to the given CGraph (runs rather fast because of early enumeration branch truncation),
    \item set of basic operations that emulate such set of the Apache Beam framework,
    \item evaluation function that runs the concrete graph.
\end{enumerate}

TODO ref https://github.com/flame-stream/halco

TODO(add, rewrite)

TODO
