% This is samplepaper.tex, a sample chapter demonstrating the
% LLNCS macro package for Springer Computer Science proceedings;
% Version 2.20 of 2017/10/04
%
\documentclass[runningheads]{llncs}
%
\usepackage{graphicx}
% Used for displaying a sample figure. If possible, figure files should
% be included in EPS format.
%
% If you use the hyperref package, please uncomment the following line
% to display URLs in blue roman font according to Springer's eBook style:
% \renewcommand\UrlFont{\color{blue}\rmfamily}

% listings
\usepackage{xcolor,listings}
\usepackage{textcomp}
\lstset{upquote=true}

\begin{document}
%
\title{Calco: Contract-based Approach\\ to Specify Dataflows}
%
%\titlerunning{Abbreviated paper title}
% If the paper title is too long for the running head, you can set
% an abbreviated paper title here
%
\author{Andrey Stoyan\inst{1} \and
Artem Trofimov\inst{1} \and
Nikita Sokolov\inst{1} \and
Boris Novikov\inst{2} \and
Igor E. Kuralenok\inst{1}}
%
\authorrunning{A. Stoyan et al.}
% First names are abbreviated in the running head.
% If there are more than two authors, 'et al.' is used.
%
\institute{Yandex \and
National Research University Higher School of Economics}
%
\maketitle              % typeset the header of the contribution
%
\begin{abstract}
    Distributed dataflows can be specified in two ways: by defining an execution graph or specifying the desired result, e.g., using SQL.
    An execution graph consists of arbitrary operations, so it cannot be automatically transformed for optimization.
    SQL is popular for data analytics, and its optimization is well-researched.
    However, it is sometimes inconvenient or impossible to use SQL for general data management tasks, such as ETL, machine learning pipelines, etc.
    In this work, we introduce a novel approach to specify distributed dataflows.
    Our method is based on declarative specifications of user-defined operations called contracts.
    Such specifications allow us to build a set of equivalent graphs, which form a space for optimization.
    We implement a graphs generation prototype and outline the challenges regarding the optimization problem.

\keywords{Optimization.}
\end{abstract}
%
%
%
\section{Introduction}
Distributed batch processing systems, such as Google MapReduce~\cite{Dean:2008:MSD:1327452.1327492}, Apache Hadoop~\cite{hadoop2009hadoop}, and Apache Spark~\cite{Zaharia:2016:ASU:3013530.2934664}, address the need to process large amounts of data (e.g., Internet-scale). The basic idea behind them is to independently process large data blocks (batches) that are collected from static datasets. The main advantages of these systems are the fault tolerance and scalability~\cite{borthakur2011apache} of massively parallel computations on commodity hardware. Batch processing ensures latency in terms of hours or even days~\cite{carbone2015apache, chang2014hawq, sun2023survey}. 

Batch processing is suitable for scenarios where data freshness is not a critical factor, such as in the collection of large datasets, exract-transform-load (ETL) problems, non-real-time data analysis, large-scale scientific computing, and log processing. In recent years, standalone batch processing systems are being replaced by batch processing subsystems within cloud-based data lakes and data warehouses. They may provide better performance due to transformations applied to data within the storage system itself (ELT)~\cite{stonebraker2024goes, akidau2024continuous}. However, these subsystems are still focused on scenarios with no strong requirements on latency. 

Distributed stream processing engines (SPEs) handle analytical, computational, and monitoring scenarios where data freshness is a strong requirement. These scenarios include online analytics (OLAP), short-term personalization, IoT, network monitoring. In these cases, data appear as a continuous, discrete, and potentially infinite stream of items, necessitating the continuous production of freshly updated results~\cite{fragkoulis2024survey, diro2024anomaly}. SPEs can be both standalone, such as Flink~\cite{carbone2015apache}, Spark Streaming~\cite{Zaharia:2012:DSE:2342763.2342773}, or Samza~\cite{Noghabi:2017:SSS:3137765.3137770}, and integrated into data lakes, such as Snowflake stream processing~\cite{akidau2024continuous}.

The low latency requirement raises two major issues related to data consistency. Firstly, an SPE requires advanced fault tolerance mechanisms to ensure quick recovery and data consistency during failures~\cite{Wang:2019:LSF:3341301.3359653, akidau2015streaming}. Secondly, there is the challenge of continuously producing a consistent resulting stream during processing. Determining the precise moment when a data item is ready for release in a distributed environment can be complex~\cite{Tucker:2003:EPS:776752.776780, DBLP:journals/pvldb/BegoliACHKKMS21}.

We acknowledge that, although there have been attempts to address these challenges, there is still room for improvement. In this thesis, we develop formal models that describe the shortcomings and limitations of current approaches. Additionally, we design more efficient techniques based on the theoretical insights gained from our analysis and modeling of existing methods. We demonstrate that the proposed methods are not only theoretically more efficient, but also outperforms state-of-the-art industrial alternatives.

\section{Research Methodology}
Our first step is to identify the actual challenges. The challenges have been identified throughout preparation of our prior publications~\cite{we2018adbis, trofimov2018consistency, we2018seim, we2018beyondmr, we2019debs, webirte, thepaper, 10.1145/3524860.3539809, trofimov2023bounding}. We briefly discuss them in Section~\ref{thesis-intro-challenges}. The corresponding literature is reviewed in Chapter~\ref{thesis-chapter-literature-review}.

After the identification of challenges, we design a formal model suitable to describe the corresponding domain. This step allows us to theoretically justify the challenges, compare existing approaches, highlight their advantages and limitations, and identify areas for improvement.

In the third step, we verify whether our theoretical findings align with experimental studies. At this stage, our goal is to develop technical implementations and ensure that our theoretically more efficient approaches are feasible in practice.

\section{Novelty}
The novelty of this work lies in addressing open challenges and achieving results that surpass the current state-of-the-art. The theoretical models we have designed describe a wider range of real-life problems and scenarios than existing alternatives. Additionally, our practical approaches and algorithms outperform those found in current research and industry standards. 

More specifically, this work brings the following novelty:
\begin{enumerate}
    \item Formal model for delivery guarantees, and a demonstration based on this model that deterministic SPEs can theoretically outperform non-deterministic ones in terms of latency.
    \item Experimental demonstration that deterministic SPE outperforms state-of-the-art alternatives in terms of latency.
    \item Formal model for substream management, and a demonstration based in this model of the lower bound of extra network traffic for detecting substream termination.
    \item Implementation of the substream management framework reaching the lower bound of extra network traffic and demonstration that it outperforms state-of-the-art alternatives.
\end{enumerate}
The main contributions of this work are outlined in more detail in Section~\ref{thesis-intro-contributions}.

\section{Open Challenges}
\label{thesis-intro-challenges}
\input{Chapters/Introduction/intro-open-challenges}

\section{Primary Contributions}
\label{thesis-intro-contributions}
We highlight the main results of this work in the following contributions, which address the challenges detailed in Section~\ref{thesis-intro-challenges}.

\subsection{Formal Model for Delivery Guarantees}

We introduced a formal framework for modeling consistency properties for any stream processing system~\cite{thepaper}. It was shown that the property of determinism is tightly connected with the concept of exactly-once. We proved that non-deterministic systems must persistently save a state of non-commutative operations before output delivery in order to achieve exactly-once.

We demonstrated that most of the state-of-the-art stream processing systems~\cite{Carbone:2017:SMA:3137765.3137777, Zaharia:2012:DSE:2342763.2342773, Akidau:2013:MFS:2536222.2536229, apache:storm:trident} use one of the following approaches to overcome this problem: 

\begin{itemize}
    \item Inherit exactly-once from batch processing using small-sized batches (micro-batching)
    \item Apply distributed transaction control protocols which guarantee that states are saved before delivery of elements affected by these states
    \item Write results of an operation to external storage on each input element
\end{itemize}

All these methods experience difficulties with working under low-latency requirements (less than a second). In the first case, latency cannot be lower than the batching period, in the second case, the distributed two-phase commit may result in a significant increase of latency, while in the third case latency is bounded below by the duration of external writes. The formal framework along with the mentioned theoretical insights are covered in Chapter~\ref{thesis-chapter-delivery-guarantees}.

\subsection{Exactly-once Based on Lightweight Determinism}

Using our formal inference, we designed the \textit{drifting state} technique~\cite{we2018adbis}, a lightweight optimistic approach for ensuring deterministic processing. As demonstrated earlier, deterministic systems can theoretically be more efficient for exactly-once processing in terms of latency, so that we adapted drifting state to achieve exactly-once guarantee~\cite{thepaper}. Our protocols offer several important features:

\begin{itemize}
    \item Elements are processed in a pure streaming fashion, without the need for input buffering.
    \item Business-logic computations, state snapshotting, and the delivery of output items operate asynchronously and independently.
    \item Exactly-once processing semantics are maintained.
\end{itemize}

We implemented the prototype of the proposed technique to examine its performance~\cite{we2018adbis, we2018seim, thepaper}. Our experiments demonstrated that the introduced protocols for fault tolerance are scalable and provide remarkably low overhead within different computational layouts. The comparison with the industrial stream processing solution indicated that our prototype could provide lower latency under exactly-once requirement. The drifting state technique along with exactly-once implementation on top of it and experimental study are detailed in Chapter~\ref{thesis-chapther-optimistic}.

\subsection{Formal Model for Substream Management}

We formalized the problem of substream management and highlighted its main features~\cite{10.1145/3524860.3539809, trofimov2023bounding}. Particularly, we introduced the notions of {\em soft bound} and {\em firm bound} required by various stream processing scenarios. Some applications that apply substream management systems do not require any particular properties of termination events. In this case, we denote the guarantee provided by such events as {\em soft bound} because termination events indicate that the substream ended some time ago. However, other problems require a {\em firm bound}: guarantee that the substream ends {\em exactly} after the termination event. For example, a commonly used snapshotting protocol~\cite{2015arXiv150608603C, jacques2016consistent} relies on an {\em epoch}. An epoch is a special substream that must be processed atomically. Therefore, the SPE requires the termination event for a given epoch to occur immediately after the last processing event belonging to that epoch. Otherwise, the snapshot can be inconsistent, capturing elements from multiple epochs.

We also provided several insights into the performance of substream management systems. We showed that the network overhead induced by a substream management system cannot be lower than linear in terms of the number of computational nodes. We demonstrated that the state-of-the-art punctuations framework, as applied in many production-scale SPEs~\cite{tucker2003exploiting}, incurs quadratic network overhead with respect to the number of computational nodes, indicating that it is far from the lower bound. Chapter~\ref{thesis-chapter-substreams-consistency} is dedicated to the substream formal model and the mentioned performance insights.

\subsection{Substream Management Approach Reaching Lower Bound of Network Traffic Overhead}

We designed and implemented a new substream management technique called \tracker~\cite{10.1145/3524860.3539809, trofimov2023bounding} that reaches the theoretical lower bound of network traffic overhead for this task. Our approach has the following features:

\begin{itemize}
    \item Efficiency due to the lower bound of service traffic overhead
    \item Suitability for problems that require non-linear executions: graph traversing, iterative algorithms, etc.
    \item Suitability for substreams consisting of a few elements
    \item Scalability due to the distribution of extra network traffic from operators between multiple nodes
\end{itemize}

These properties of the \tracker\ framework create a possibility to apply substream management in new applications; this includes smart caching of operator state and latency-conscious windowed joins.

We also demonstrated that \tracker\ outperforms state-of-the-art substream management mechanisms in real-life scenarios. The design of the \tracker\ framework along with its implementation details and experimental study are discussed in Chapter~\ref{thesis-chapter-tracker}.

\section{Published Work}
Most of the material in this dissertation is derived from work that has already been published and peer-reviewed. Particularly, Chapter~\ref{thesis-chapter-delivery-guarantees} and Chapter~\ref{thesis-chapther-optimistic} include content from papers on optimistic technique for achieving determinism and the exactly-once approach on top of that~\cite{we2018seim, we2018adbis, thepaper, we2018beyondmr, trofimov2018consistency, webirte, we2019debs}. Chapter~\ref{thesis-chapter-substreams-consistency} and Chapter~\ref{thesis-chapter-tracker} are based on papers on \tracker\ framework and substream management~\cite{10.1145/3524860.3539809, trofimov2023bounding}.

The complete list of published material along with assigned contributions is the following:

\begin{enumerate}
    \item Artem Trofimov, Nikita Sokolov, Nikita Marshalkin, Igor Kuralenok, and Boris Novikov. \textbf{Bounding substreams in distributed stream processing}. Information Systems, Volume 117, 2023. \newline
    
    \textit{Contribution:} implementation of the main parts of the substream management framework and the formal model for the substream management domain, leading the writing of the paper.

    \item Artem Trofimov, Nikita Sokolov, Nikita Marshalkin, Igor Kuralenok, and Boris Novikov. \textbf{Substream management in distributed streaming dataflows}. In Proceedings of the 16th ACM International Conference on Distributed and Event-Based Systems (DEBS '22), pp. 55-66. ACM, 2022. \newline
    
    \textit{Contribution:} implementation of the main parts of the substream management framework and the formal model for the substream management domain, leading the writing of the paper.

    \item Igor Kuralenok, Artem Trofimov, Nikita Marshalkin, and Boris Novikov. \textbf{Deterministic Model for Distributed Speculative Stream Processing}. In European Conference on Advances in Databases and Information Systems (pp. 233-246). Cham: Springer International Publishing, 2018. \newline
    
    \textit{Contribution:} refinement of the mechanism for ensuring determinism, conducting the experiments, co-authoring the paper.

    \item Igor Kuralenok, Nikita Marshalkin, Artem Trofimov, and Boris Novikov. \textbf{An optimistic approach to handle out-of-order events within analytical stream processing}. In Third Conference on Software Engineering and Information Management (SEIM-2018) (pp. 22-29), 2018. \newline
    
    \textit{Contribution:} participation in implementation of the mechanism for handling out-of-order elements, conducting the experiments, co-authoring the paper.

    \item Artem Trofimov, Igor Kuralenok, Nikita Marshalkin, and Boris Novikov. \textbf{Delivery, consistency, and determinism: rethinking guarantees in distributed stream processing}. arXiv preprint arXiv:1907.06250, 2019. \newline
    
    \textit{Contribution:} implementation of the main parts of the formal model for delivery guarantees and the mechanism for achieving exactly-once, leading the writing of the paper.

    \item Artem Trofimov, Nikita Sokolov, Mikhail Shavkunov, Igor Kuralenok, and Boris Novikov. \textbf{Distributed Classification of Text Streams: Limitations, Challenges, and Solutions}. In Proceedings of Real-Time Business Intelligence and Analytics (pp. 1-6), 2019. \newline
    
    \textit{Contribution:} problem setup, conducting the experiments, leading the writing of the paper.

    \item Artem Trofimov. \textbf{Consistency Maintenance in Distributed Analytical Stream Processing}. In New Trends in Databases and Information Systems: ADBIS 2018 Short Papers and Workshops, AI* QA, BIGPMED, CSACDB, M2U, BigDataMAPS, ISTREND, DC, Budapest, Hungary, September, 2-5, 2018, Proceedings 22. Springer International Publishing, 2018. \newline
    
    \textit{Contribution:} leading the writing of the paper.

    \item Igor Kuralenok, Artem Trofimov, Nikita Marshalkin, and Boris Novikov. \textbf{Flamestream: model and runtime for distributed stream processing}. In Proceedings of the 5th ACM SIGMOD Workshop on Algorithms and Systems for MapReduce and Beyond (pp. 1-2), 2018 \newline
    
    \textit{Contribution:} conducting the experiments, co-authoring the paper.

    \item Artem Trofimov, Mikhail Shavkunov, Sergey Reznick, Nikita Sokolov, Mikhail Yutman, Igor Kuralenok, and Boris Novikov. \textbf{Reproducible and reliable distributed classification of text streams}. In Proceedings of the 13th ACM International Conference on Distributed and Event-based Systems (pp. 264-265), 2019. \newline
    
    \textit{Contribution:} problem setup, leading the writing of the paper.
\end{enumerate}

\section{Dissertation Outline}
The rest of the dissertation is structured as follows: Chapter~\ref{thesis-chapter-literature-review} covers the literature review of the consistency and fault tolerance in distributed stream processing. In Chapter~\ref{thesis-chapter-delivery-guarantees} we formalize the delivery guarantees and demonstrate that deterministic systems can be theoretically more efficient for exactly-once than non-deterministic in terms of latency. In Chapter~\ref{thesis-chapther-optimistic} we indtroduce a lightweight approach for achievening determinism and exactly-once and show that it outperforms state-of-the-art alternatives. Chapter~\ref{thesis-chapter-substreams-consistency} covers the formal model for the substream consistency and the estimation of the lower bound for extra network traffic required by substream management mechanisms. In Chapter~\ref{thesis-chapter-tracker} we present the substream management technique reaching the lower bound of extra network traffic and demonstrate its practical feasibility. In Chapter~\ref{thesis-chapter-conclusion}, we conclude by summarizing our findings and discussing future directions.

\section{Running example}
\input{example.tex}

\section{Calco overview}
\input{overview.tex}

\section{Implementation and challenges}
\input{eval.tex}

\section{Conclusion}
\input{conclusion.tex}

\bibliographystyle{splncs04}
\bibliography{bibliography/flame-stream}

\end{document}
