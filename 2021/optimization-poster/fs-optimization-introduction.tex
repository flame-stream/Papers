\label {fs-optimization-introduction}

Modern day data analytics commonly requires online processing of continuously changing data from unbounded streams. A standard way of defining a stream processing pipeline is an execution graph. An alternative would be a declarative approach, which SQL is a popular example of.

For each declarative query, which is translated into a graph upon execution, there can be multiple execution graphs. The best execution graph is selected during \textit{optimization}. In databases, query optimization routinely employs a cost-based approach, which estimates a cost function value for each considered plan and selects the plan (which is a graph with nodes representing query operators) with the minimum cost value. The cost function is typically estimated based on relation cardinality and operator selectivity. Therefore, cost-based optimization requires knowledge of statistical information \cite{Neumann2018optimization}. However, obtaining such knowledge in streaming systems presents certain difficulties.

Efforts to optimize streaming queries execution focus on finding a suitable mapping from a logical graph to a physical graph \cite{grulich2020grizzly, gedik2009code}; such optimizations are local, and in order to perform global optimization, the planner needs to optimize the logical graph as well. The problem of logical level declarative query optimization is currently relevant and presents a challenge.

In this work, we present a detailed analysis of the problem of streaming SQL queries optimization and the challenges in implementing its solution. We also describe preliminary experiments that we have conducted in order to demonstrate feasibility of streaming SQL optimization.  
