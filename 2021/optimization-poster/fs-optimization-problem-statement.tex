\label {sec:fs-optimization-problem-statement}

To illustrate the problem of streaming SQL query optimization, let us consider an example query, written using the NEXMark \cite{tucker2008nexmark} benchmark model simulating an online auction:

    \begin{lstlisting}[
           language=SQL,
           showspaces=false,
           basicstyle=\ttfamily,
           numbers=left,
           numberstyle=\tiny,
           commentstyle=\color{gray},
           caption={The proposed NEXMark-based query}, 
           captionpos=b
        ]
SELECT P.name, B.price, A.itemName 
  FROM Person P 
    INNER JOIN Bid B 
      ON B.bidder = P.id 
    INNER JOIN Auction A 
      ON A.seller = P.id
\end{lstlisting}

This query contains two join operators, so there are at least two ways to execute this query: e.g. first performing the join between \texttt{Person} and \texttt{Auction}, then joining the result with \texttt{Bid}, or vice-versa. Which join order is more optimal depends on the statistical properties of the data, such as the arrival rate for each of the \texttt{Person}, \texttt{Auction}, \texttt{Bidder} streams, and such statistical properties have a tendency to change over time. Therefore, an execution graph that was once optimal might become inefficient after some time.

Various DBMS have employed a technique known as adaptive optimization. This technique utilizes an \textit{adaptivity loop}, during which the system measures and evaluates data characteristics and uses these evaluations to select a new query plan better suited to the current data \cite{deshpande2007adaptive}. A similar technique could be useful in streaming queries optimization. Thereby, in this paper we consider the problem of adaptive optimization of streaming SQL queries in distributed stream processing systems. 

Adaptive database optimization is not applicable to streaming SQL queries as is: in databases, such optimization is applied during the execution of a long-running query, with the queried data remaining the same, while in streaming systems the data is continuously updated, and the same query is executed on each subsequent window; therefore, it would make sense for streaming systems to optimize the execution graph between the windows. This presents certain challenges in implementing logical level streaming query optimization.

