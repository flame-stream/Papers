\documentclass[sigconf]{acmart}

\usepackage{graphicx}
\usepackage{algorithm} % for algorithms
\usepackage{algpseudocode}
\usepackage{booktabs} % For formal tables
\usepackage{amsthm} % For claims
\usepackage{bbm} % indicator function
\usepackage{acmcopyright}


% table
\usepackage[flushleft]{threeparttable} % http://ctan.org/pkg/threeparttable
\usepackage{booktabs,caption}

\theoremstyle{remark}

\settopmatter{printacmref=false, printccs=true, printfolios=true}
\pagestyle{empty} % removes running headers

\newcommand{\PicScale}{0.5}
\newcommand {\FlameStream} {FlameStream}
\begin{document}

% \copyrightyear{2019} 
% \acmYear{2019} 
% \setcopyright{rightsretained} 
% \acmConference[DEBS '19]{DEBS '19: The 13th ACM International Conference on Distributed and Event-based Systems}{June 24--28, 2019}{Darmstadt, Germany}
% \acmBooktitle{DEBS '19: The 13th ACM International Conference on Distributed and Event-based Systems (DEBS '19), June 24--28, 2019, Darmstadt, Germany}\acmDOI{10.1145/3328905.3332514}
% \acmISBN{978-1-4503-6794-3/19/06}

\title {Towards Adaptive SQL Query Optimization in Distributed Stream Processing}

% \author{Artem Trofimov,$^ {1,2}$    Mikhail Shavkunov,$^3$    Sergey Reznick,$^4$     Nikita Sokolov,$^{5}$   Mikhail Yutman,$^3$ \\   Igor E. Kuralenok,$^1$    and  Boris Novikov$^ {3}$ }
% \affiliation{%
% \institution{$^1$JetBrains Research}
% }
% \affiliation{%
% \institution{$^2$Saint Petersburg State University}
% }
% \affiliation{%
% \institution{$^3$National Research University Higher School of Economics}
% }
% \affiliation{%
% \institution{$^4$ Kofax}
% }
% \affiliation{%
% \institution{$^5$ ITMO University}
%   \city{St. Petersburg} 
%   \country{Russia}
% }
% \email{\string{trofimov9artem, mv.shavkunov, sergey.reznick, faucct, myutman, ikuralenok\string}@gmail.com, borisnov@acm.org}

\begin{abstract}

Distributed stream processing is widely adopted for real-time data analysis and management. SQL is becoming a common language for robust streaming analysis due to the introduction of time-varying relations and event time semantics. However, query optimization in state-of-the-art stream processing engines (SPEs) remains limited: runtime adjustments to execution plans are not applied. This fact restricts the optimization capabilities because SPEs lack the statistical data properties before query execution begins. Moreover, streaming queries are often long-lived, and these properties can be changed over time. 

Adaptive optimization, used in databases for queries with insufficient existing data statistics, can fit the streaming scenario. In this work, we explore the main challenges that SPEs face during the adjustment of adaptive optimization: retrieving and predicting statistical data properties, execution graph migration, misfit of SPEs programming interfaces, etc. We demonstrate the feasibility of the proposed approach within an extension of the Nexmark streaming benchmark and outline our further work on this topic.

\end{abstract}

% \begin{CCSXML}
% \begin{CCSXML}
% \begin{CCSXML}
% <ccs2012>
% <concept>
% <concept_id>10002951.10002952.10002953.10010820.10003208</concept_id>
% <concept_desc>Information systems~Data streams</concept_desc>
% <concept_significance>500</concept_significance>
% </concept>
% <concept>
% <concept_id>10002951.10003317.10003347.10003356</concept_id>
% <concept_desc>Information systems~Clustering and classification</concept_desc>
% <concept_significance>500</concept_significance>
% </concept>
% <concept>
% <concept_id>10002951.10003227.10003351.10003446</concept_id>
% <concept_desc>Information systems~Data stream mining</concept_desc>
% <concept_significance>300</concept_significance>
% </concept>
% </ccs2012>
% \end{CCSXML}

% \ccsdesc[500]{Information systems~Data streams}
% \ccsdesc[500]{Information systems~Clustering and classification}
% \ccsdesc[300]{Information systems~Data stream mining}

% \keywords{Data streams, text classification, reproducibility, exactly once}

\maketitle

\thispagestyle{empty}

\section{Introduction}

\section{A running example}

\subsection{Query}

\subsection{Implementation}

\subsection{Idea}

\section{Challenges}

\subsection{Fetching and predicting data statistics}

\subsection{Statistics integration in query processing}

\subsection{In-flight execution graph migration}

\section {Preliminary experiments}

\section{Discussion}

\section{Related work}

\section{Conclusion}

\bibliographystyle{ACM-Reference-Format}
\bibliography{../../bibliography/flame-stream}

\end {document}

\endinput
