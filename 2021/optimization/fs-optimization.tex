\documentclass[sigconf]{acmart}

\usepackage{graphicx}
\usepackage{algorithm} % for algorithms
\usepackage{algpseudocode}
\usepackage{booktabs} % For formal tables
\usepackage{amsthm} % For claims
\usepackage{bbm} % indicator function
\usepackage{acmcopyright}

% listings
\usepackage{xcolor,listings}
\usepackage{textcomp}
\lstset{upquote=true}

% plots
\usepackage{pgfplots}

% table
\usepackage[flushleft]{threeparttable} % http://ctan.org/pkg/threeparttable
\usepackage{booktabs,caption}

\theoremstyle{remark}

\settopmatter{printacmref=false, printccs=true, printfolios=true}
\pagestyle{empty} % removes running headers

\newcommand{\PicScale}{0.5}
\newcommand {\FlameStream} {FlameStream}
\begin{document}

% \copyrightyear{2019} 
% \acmYear{2019} 
% \setcopyright{rightsretained} 
% \acmConference[DEBS '19]{DEBS '19: The 13th ACM International Conference on Distributed and Event-based Systems}{June 24--28, 2019}{Darmstadt, Germany}
% \acmBooktitle{DEBS '19: The 13th ACM International Conference on Distributed and Event-based Systems (DEBS '19), June 24--28, 2019, Darmstadt, Germany}\acmDOI{10.1145/3328905.3332514}
% \acmISBN{978-1-4503-6794-3/19/06}

\title {Towards Adaptive SQL Query Optimization in Distributed Stream Processing}

% \author{Darya Sharkova}
% \authornote{Both authors contributed equally to this research.}
% \affiliation{%
%   \institution{National Research University Higher School of Economics}
%   \city{Saint Petersburg}
%   \country{Russia}
% }
% \email{sharkovadarya@gmail.com}

% \author{Alexander Chernokoz}
% \authornotemark[1]
% \affiliation{%
%   \institution{ITMO University}
%   \city{Saint Petersburg}
%   \country{Russia}
% }
% \email{chernokoz@hotmail.com}

% \author{Artem Trofimov}
% \affiliation{%
%   \institution{Yandex}
%   \city{Saint Petersburg}
%   \country{Russia}
% }
% \email{trofimov9artem@gmail.com}

% \author{Nikita Sokolov}
% \affiliation{%
%   \institution{Yandex}
%   \city{Saint Petersburg}
%   \country{Russia}
% }
% \email{faucct@gmail.com}

% \author{Ekaterina Gorshkova}
% \affiliation{%
%   \institution{JResearch Software}
%   \city{Prague}
%   \country{Czech Republic}
% }
% \email{cathy@jresearch.org}

% \author{Igor Kuralenok}
% \affiliation{%
%   \institution{Yandex}
%   \city{Saint Petersburg}
%   \country{Russia}
% }
% \email{ikuralenok@gmail.com}

% \author{Boris Novikov}
% \affiliation{%
%   \institution{National Research University Higher School of Economics}
%   \city{Saint Petersburg}
%   \country{Russia}
% }
% \email{borisnov@acm.org}

\begin{abstract}

Distributed stream processing is widely adopted for real-time data analysis and management. SQL is becoming a common language for robust streaming analysis due to the introduction of time-varying relations and event time semantics. However, query optimization in state-of-the-art stream processing engines (SPEs) remains limited: runtime adjustments to execution plans are not applied. This fact restricts the optimization capabilities because SPEs lack the statistical data properties before query execution begins. Moreover, streaming queries are often long-lived, and these properties can be changed over time. 

Adaptive optimization, used in databases for queries with insufficient existing data statistics, can fit the streaming scenario. In this work, we explore the main challenges that SPEs face during the adjustment of adaptive optimization: retrieving and predicting statistical data properties, execution graph migration, misfit of SPEs programming interfaces, etc. We demonstrate the feasibility of the proposed approach within an extension of the Nexmark streaming benchmark and outline our further work on this topic.

\end{abstract}

% \begin{CCSXML}
% \begin{CCSXML}
% \begin{CCSXML}
% <ccs2012>
% <concept>
% <concept_id>10002951.10002952.10002953.10010820.10003208</concept_id>
% <concept_desc>Information systems~Data streams</concept_desc>
% <concept_significance>500</concept_significance>
% </concept>
% <concept>
% <concept_id>10002951.10003317.10003347.10003356</concept_id>
% <concept_desc>Information systems~Clustering and classification</concept_desc>
% <concept_significance>500</concept_significance>
% </concept>
% <concept>
% <concept_id>10002951.10003227.10003351.10003446</concept_id>
% <concept_desc>Information systems~Data stream mining</concept_desc>
% <concept_significance>300</concept_significance>
% </concept>
% </ccs2012>
% \end{CCSXML}

% \ccsdesc[500]{Information systems~Data streams}
% \ccsdesc[500]{Information systems~Clustering and classification}
% \ccsdesc[300]{Information systems~Data stream mining}

% \keywords{Data streams, text classification, reproducibility, exactly once}

\maketitle

\thispagestyle{empty}

\section{Introduction}
\label {fs-optimization-introduction}

Modern day data analytics commonly requires online processing of continuously changing data from unbounded streams. A standard way of defining a stream processing pipeline is an execution graph. An alternative would be a declarative approach, which SQL is a popular example of.

For each declarative query, which is translated into a graph upon execution, there can be multiple execution graphs. The best execution graph is selected during \textit{optimization}. In databases, query optimization routinely employs a cost-based approach, which estimates a cost function value for each considered plan and selects the plan (which is a graph with nodes representing query operators) with the minimum cost value. The cost function is typically estimated based on relation cardinality and operator selectivity. Therefore, cost-based optimization requires knowledge of statistical information \cite{Neumann2018optimization}. However, obtaining such knowledge in streaming systems presents certain difficulties.

Efforts to optimize streaming queries execution focus on finding a suitable mapping from a logical graph to a physical graph \cite{grulich2020grizzly, gedik2009code}; such optimizations are local, and in order to perform global optimization, the planner needs to optimize the logical graph as well. The problem of logical level declarative query optimization is currently relevant and presents a challenge.

In this work, we present a detailed analysis of the problem of streaming SQL queries optimization and the challenges in implementing its solution. We also describe preliminary experiments that we have conducted in order to demonstrate feasibility of streaming SQL optimization.  


\section{Problem statement}
\label {sec:fs-optimization-problem-statement}

This section illustrates the problem of streaming SQL query optimization using a running example of a query for a streaming system.

We are using the NEXMark benchmark~\cite{tucker2008nexmark} for our query. The NEXMark benchmark suite, designed for queries over continuous data streams, is an extension of the XMark benchmark~\cite{schmidt2002xmark} adopted for use with streaming data. 
The NEXMark scenario simulates an online auction system with three kinds of entities: people selling items or bidding on items, items submitted for auction, and bids on items. 
These kinds of entities will be referred to as \texttt{Person}, \texttt{Auction}, and \texttt{Bid} respectively. 
The original NEXMark benchmark includes eight queries that utilize the full spectrum of SQL features, but none of them contain more than one join operator.  Unfortunately, the system used in our experiments can optimize the order of joins only. 
We extended the benchmark with  the following query based on the NEXMark model:  



\begin{lstlisting}[
           language=SQL,
           showspaces=false,
           basicstyle=\ttfamily,
           numbers=left,
           numberstyle=\tiny  %,
   %        commentstyle=\color{gray},
%        caption={The proposed NEXMark-based query in the Apache Calcite SQL dialect}, 
   %        captionpos=b
        ]
SELECT P.name, P.city, P.state, 
       B.price, A.itemName 
  FROM Person P 
    INNER JOIN Bid B 
      ON B.bidder = P.id 
    INNER JOIN Auction A 
      ON A.seller = P.id
\end{lstlisting}

This query selects all the people who have joined the auction as both bidders and sellers. 
For each such person their name, city and state of residence are selected, as well as the price of each of their bids and the name of each item they are selling at the auction. 

This query contains two join operators, which means that there are at least two ways to execute this query.

One of the logical   plans (with substituted variable names omitted; using Apache Beam transforms as operators) for our example query looks like  follows: 

\begin{lstlisting}[
           showspaces=false,
           basicstyle=\ttfamily,
           numbers=left,
           numberstyle=\tiny %,
 %          commentstyle=\color{gray},
 %          caption={Logical plan}, 
 %          captionpos=b
        ]
LogicalProject(name, city, state, 
               price, itemName)
  LogicalJoin(condition, joinType=inner) 
    LogicalJoin(condition, joinType=inner)
      BeamIOSourceRel(table=Person)
      BeamIOSourceRel(table=Bid)
    BeamIOSourceRel(table=Auction)
\end{lstlisting}

A physical plan derived from this logical plan produced for Flink executor is as follows:

\begin{lstlisting}[
           showspaces=false,
           basicstyle=\ttfamily,
           numbers=left,
           numberstyle=\tiny %,
   %        commentstyle=\color{gray},
     %      caption={Physical plan}, 
   %        captionpos=b
        ]
BeamCalcRel(name, city, state, price, itemName)
  BeamCoGBKJoinRel(condition, joinType=inner)
    BeamIOSourceRel(table=Bid)
    BeamCoGBKJoinRel(condition, joinType=inner)
      BeamIOSourceRel(table=Person)
      BeamIOSourceRel(table=Auction)
\end{lstlisting}

An alternative is  the following physical plan for this query, with the two join operators in a different order:

\begin{lstlisting}[
           showspaces=false,
           basicstyle=\ttfamily,
           numbers=left,
           numberstyle=\tiny  %,
    %       commentstyle=\color{gray},
    %       caption={Alternative physical plan}, 
   %        captionpos=b
        ]
BeamCalcRel(name, city, state, price, itemName)
  BeamCoGBKJoinRel(condition, joinType=inner)
    BeamIOSourceRel(table=Auction)
    BeamCoGBKJoinRel(condition, joinType=inner)
      BeamIOSourceRel(table=Person)
      BeamIOSourceRel(table=Bid)        
\end{lstlisting}


In unbounded data streams, elements can be grouped into windows based on event time or the number of tuples in each window, and each window can be processed similarly to a SQL table. 
Cost-based optimization requires statistical knowledge about the data, such as the cardinality of each window, which can be inferred from the element arrival rate in the case of streaming data. 
In our example, the first plan, where \texttt{Person} and \texttt{Bid} are joined first, and then the result is joined with \texttt{Auction}, would be preferable if the arrival rate of auctions significantly exceeded the arrival rate of bids, meaning that while many items have been getting put up for Auction, not many sellers have been making bids. 
If, however, after some time, sellers started making many bids, the second execution plan would have a lesser cost value. Thus, as data statistics change for the query execution, a previously optimal plan might become inefficient. 

Due to the imprecision of data statistics, it is well known that optimal in terms of cost function plan is not necessarily optimal in actual resource consumption. To address this issue, several techniques known as adaptive query processing were developed~\cite{deshpande2007adaptive}. Under these techniques, the execution is paused, and the query is re-optimized with more precise statistics, and then execution is resumed with the new plan. As the repeated optimization consumes a certain amount of resources itself, the adaptive optimization makes sense for relatively long-running queries and is hardly applicable in data streams. 

We use a different kind of adaptivity in our approach: the statistics collected during previous query executions are used to re-optimize the query for subsequent executions. As soon as the new plan changes, the query execution on subsequent windows is switched to the new plan. 

This, as well as other specifics of SPEs, presents particular challenges in implementing global optimization of streaming queries, which we explore in the following section.



\section{Challenges}
\label {sec:fs-optimization-challenges}
In this section, we present the challenges in adapting SQL optimization techniques to data streams. We categorize each challenge as either research or engineering.  The former category includes challenges that require further work and we are not sure in the final outcome, while the latter includes everything related to incorporation of our approach into a production system. 

\subsection{Fetching and predicting data statistics}
Cost-based optimization requires statistical information on data in order to calculate cost function values for each plan. However, upon starting a streaming query execution, no information about the data, such as its arrival rate, is available. 
Therefore, to properly apply cost-based optimization to streaming SQL queries, it is necessary to collect data statistics over the course of query execution. 
Moreover, since we possess no definitive knowledge about the arriving data, we need to predict statistics for each next window based on statistics for previous windows to utilize an optimal plan for the upcoming windows. To this end, we identify two challenges in using statistics for streaming query optimization:

\begin{itemize}
    \item \textbf{Engineering}:
    Statistical information on stream elements, such as their arrival rate, needs to be collected during execution at runtime without seriously affecting the performance of a distributed SPE. % which can present certain challenges since there is no centralized point at which to collect statistics, so we would need to aggregate it somehow, which can affect the performance of the system itself
    \item \textbf{Research}
We need techniques to predict statistics for upcoming windows based on statistics collected for previous windows.    
We expect that previous window statistics would present a decent baseline. 
However, this assumption requires further investigation. 
\end{itemize}

Popular SPEs and frameworks for defining streaming workflows do not offer any statistics fetching or predicting. 
For example, the Apache Beam framework passes constant values to the query planner instead of any actual data statistics to
 Apache Calcite, a dynamic data management framework that implements its SQL processing functionality.



\subsection{Using statistics for streaming query optimization}
The API of the current state-of-the-art systems typically utilizes a consecutive approach to building a graph for query execution. First, a planner builds a logical graph. Then another component converts it to a physical graph, makes local optimizations, and finally passes it to an executor.
Such separation leaves no opportunity to pass any data from the physical level to the logical level and adapt the graph to the new data. Therefore, making any runtime adjustments to execution plans impossible. 

The progress in applying various local optimizations to the execution graphs at the physical level (see˜\cite{grulich2020grizzly} or Google Cloud Dataflow optimizer), there are no significant results in the logical level optimization yet, and the logical level allows to use of more complex optimizations than the physical level. 
Moreover, in this paper, we discuss distributed stream processing engines, which require additional consideration for query execution. To this end, we identify the following challenges:

\begin{itemize}
    \item \textbf{Engineering}: 
    it is necessary to adapt the API of current SPEs to pass statistics collected or predicted during the query execution at the physical level to the query planner at the logical level.
    
    \item \textbf{Research}: 
    the relational algebra and the planner cost model require extension with new operators specific to distributed systems.
\end{itemize}

For example, a join operation should broadcast a stream with a low arrival rate of new elements to all the nodes in the system. 
In contrast, a high arrival rate suggests distributing different keys across different nodes. 
Therefore, the algebra should include a new \textit{distribution} operator, and the cost model should include the estimate of its cost. The cost model should consider the latency due to communication between nodes as well. Such cost models have been well-researched in distributed databases˜\cite{kossmann2000thestate} but not in distributed streaming systems.




\subsection{Execution graph migration in runtime}
% well i have nothing to say about this one
Streaming data is ever-changing: new data continues arriving indefinitely, and the statistics for the next window elements might differ significantly from the statistics for previous windows. Even if the optimal query execution plan was selected based upon statistics accumulated for previous windows, the new data statistics might be different enough to render the previous plan no longer optimal. In order to adapt to the changes in data, it's necessary to identify the moment in time in which the previous plan is no longer optimal for the current or upcoming data and to migrate the execution to a new graph. % this is terribly grammatically wrong 
We identify the following challenges in migrating the execution graph in runtime:
\begin{itemize}
    \item \textbf{Technical}: the mechanics of graph migration, particularly for stateful operations such as joins and aggregations, in runtime have been researched \cite{zhu2004dynamic}; % TODO references
    however, most current popular SPEs do not provide such functionality. In order to support graph migration in stream processing frameworks such as Apache Beam, one such strategy is the parallel track strategy described in \cite{zhu2004dynamic}: a second graph can start the query execution along the first one, and the first one can terminate execution once the current window has been processed, so that all the subsequent windows are only processed by the new execution graph.
    % something something about state and migration. i don't remember
    \item \textbf{Research}: identifying the point in time at which it's feasible and beneficial to perform the graph migration is an open problem. First of all, it's necessary to be able to estimate the costs of graph migration at the current point in time. Secondly, we need to establish what qualifies as substantial enough change in statistics to warrant graph migration. % I don't know what we propose in order to do this.
\end{itemize}

\section {Preliminary experiments}
\label {sec:fs-optimization-experiments}

\section{Related work}
\label {sec:fs-optimization-related-work}




\section{Conclusion}
\label {sec:fs-optimization-conclusion}
In this paper, we introduced a technique of adaptive optimization of streaming SQL queries in distributed stream processing engines. 
We highlighted the differences between adaptive optimization for database and stream queries. 
We argued the necessity of adaptive query optimization at both the logical and the physical levels.
We identify the following challenges in the realization of  optimization techniques for streaming queries: 
fetching and predicting data statistics, 
using predicted statistics for the query optimization, 
and migrating the execution graph in runtime once the statistics have changed enough that the previous execution plan becomes suboptimal. 

We presented a running example of a streaming SQL query and performed experiments on this query and its different possible execution graphs to demonstrate the possibility of a gain in performance should the graph be adapted to the data statistics. 


\bibliographystyle{ACM-Reference-Format}
\bibliography{../../bibliography/flame-stream}

\end {document}

\endinput
