\documentclass[14pt]{matmex-diploma-custom}
\setdefaultlanguage{english}


\usepackage{graphicx}
\usepackage{algorithm} % for algorithms
\usepackage{algpseudocode}
\algblockdefx[Process]{Process}{EndProcess}[2][Unknown]{{\bf Process} {\it #2}}{}
\algblockdefx[Event]{Event}{EndEvent}[2][Unknown]{{\bf upon event #2 do}}{}

\usepackage{booktabs} % For formal tables
%\usepackage{cite} %for multiple refs

%\usepackage{amssymb} %for nice emptyset
\usepackage{amsthm} % For claims
\theoremstyle{remark}


\newcommand{\PicScale}{0.5}
\newcommand {\FlameStream} {FlameStream}


% Currently, large-scale distributed stream processing is a hot area of research. While state-of-the-art distributed stream processing systems are able to provide low-latency under at-most-once and at-least-once guarantees, achieving exactly-once semantics is still a challenging problem. An important reason behind this fact was the lack of an efficient idempotent model for stream processing. In this work we apply a low-overhead deterministic model to the problem of exactly-once. We demonstrate the lightweight protocols which use the property of determinism for achieving system-wide idempotence, and, therefore, exactly-once. Our experiments show that proposed approach can significantly outperform an alternative industrial solution.

% В настоящее время, распределенная обработка потоков данных является актуальной областью исследований. В то время как современные распределенные системы потоковой обработки позволяют получить низкую задержку при  «at-most-once» и «at-least-once» гарантиях, достижение «exactly-once» семантики все еще остается сложной задачей. Важная причина этого — отсутсвие эффективной идемпотентной модели для потоковой обработки. В этой работе мы применяем детерминированную модель с маленькими накладными расходами для остижения «exactly-once». Мы определяем легковесные протоколы, которые используют свойство детерминизма для получения идемпотентности на уровне системы, и, следовательно, «exactly-once» семантики. Наши эксперименты показывают, что предложенный подход может значительно опережать по производительности альтернативную промышленную систему.

\begin{document}

\filltitle{ru}{
    chair              = {Математическое обеспечение и администрирование информационных систем\\Кафедра информационно-аналитических систем},
    title              = {Поддержка согласованности в системах потоковой обработки данных},
    type               = {diploma},
    author             = {Трофимов Артем Владимирович},
    supervisorPosition = {д.\,ф.-м.\,н., профессор},
    supervisor         = {Новиков Б.\,А.},
    reviewerPosition   = {к.\,ф.-м.\,н.},
    reviewer           = {Кураленок И.\,Е.},
    year               = {2018}
}
\filltitle{en}{
    chair              = {Software and Administration of Information Systems\\Chair of Analytical Information Systems},
    title              = {Consistency maintenance in stream processing systems},
    author             = {Trofimov Artem},
    supervisorPosition = {professor},
    supervisor         = {Boris Novikov},
    supervisorPosition = {professor},
    reviewer           = {Igor Kuralenok}
}

\maketitle
\tableofcontents

\section* {Introduction}
%%% fs-state-intro - Introduction

\label {fs-intro-seciton}

Distributed batch processing systems, such as Google's MapReduce~\cite{Dean:2008:MSD:1327452.1327492}, Apache Hadoop~\cite{hadoop2009hadoop}, and Apache Spark~\cite{Zaharia:2016:ASU:3013530.2934664}, address a need to process vast amounts of data (e.g., Internet-scale). 
The basic   idea behind them is to independently process large data blocks (batches) that are collected from static datasets. 
%  These engines are able to run in a massively parallel fashion on clusters consisting of thousands of commodity computational units. 
The main advantages of these systems are fault tolerance and scalability~\cite{borthakur2011apache} of massively parallel computations on commodity hardware.

However, there are plenty of scenarios where processing results are most valuable at the time of data arrival, for example, IoT, news processing, financial analysis, anti-fraud, network monitoring, etc. 
Such problems cannot be directly addressed by classical MapReduce~\cite{Doulkeridis:2014:SLA:2628707.2628782}. 
State-of-the-art stream processing systems, such as Flink \cite{carbone2015apache}, Samza \cite{Noghabi:2017:SSS:3137765.3137770}, Storm \cite{apache:storm}, Spark Streaming~\cite{Zaharia:2012:DSE:2342763.2342773},   provide a computational model addressing these needs.
%   aim at filling this gap by introducing another computational models. 
%  These systems receive a record or a set of records, updates internal state if any, and sends out new records. % можно выпилить

One of the most challenging tasks for streaming systems is to provide guarantees on data processing. 
Streaming systems must release output elements before processing has finished because input data is assumed to be unbounded. 

In distributed stream processing, consistency is usually described in terms of delivery guarantees: {\em at-most-once}, {\em at-least-once}, and {\em \eo}~\cite{carbone2015apache}. 
These guarantees describe a contract regarding {\em which data} will be  processed and released in case of failures. 
\Eo\ is the strongest and the most valuable guarantee from the user perspective as it ensures that input elements are processed atomically and are not lost. These notions are seemingly simple but shadow  the dependency   of  an output item on the {\em system state} as well as on the  input item. 
Streaming systems face a need to recover computations consistently with previous input data, the current system state, and with the already delivered elements.
This requirement makes failure recovery mechanisms somewhat complex. 
%  The above implies that a system can process each element exactly once, but in practice, it can release completely invalid results due to inconsistencies in the state or in-flight elements.

This complication is resolved  by most of the existing stream processing engines. 
Flink ensures atomicity of  state updates and   delivery using a protocol based on distributed transactions. 
%   This protocol prevents inconsistencies but leads to a significant latency increase. 
Google MillWheel~\cite{Akidau:2013:MFS:2536222.2536229} enforces consistency between state and output elements by writing results of each operation to persistent external storage. 
%  The lower bound of latency is a duration of all external writes within routes of an input element and its descendants. 
Micro-batching engines like Storm Trident~\cite{apache:storm:trident} and Spark Streaming~\cite{Zaharia:2012:DSE:2342763.2342773} process data in small-sized blocks. 
Each block is atomically processed on each stage of a data flow  providing properties similar to batch processing. 
%   The main downside of this approach is high latency, about a few seconds~\cite{7530084, 7474816}. 
%Therefore,  it is clear that state-of-the-art systems are aware of the issues regarding exactly once, but pay performance price for their resolution.
The price for~\eo\ delivery is a high latency  observed in these implementations (e.g.~\cite{7530084, 7474816}).


The huge gap between the notion of~\eo\ and the properties of its implementations indicates the lack of formalization. 
Misunderstandings of streaming delivery guarantees frequently cause debates and discussions~\cite{JerryPengStreamIO, PaperTrail}. Without a formal model, it is hard to observe similarities and distinctions between existing solutions and to recognize their limitations. In this work, we introduce a formal model of stream processing that captures delivery guarantees existing in most of the state-of-the-art systems.

Another property that can be easily obtained in batch processing systems but is hard to achieve in streaming engines is {\em a determinism}. 
The determinism means that repeated runs of the system on the same data produce the same results. It is commonly considered as a challenging task~\cite{Zacheilas:2017:MDS:3093742.3093921}. On the other hand, this property is desirable, because it implies reproducibility and predictability. Intuitively, determinism is connected with delivery guarantees~\cite{Stonebraker:2005:RRS:1107499.1107504}, but, to the best of our knowledge, this relation has not been deeply investigated. 

% We demonstrate that determinism is tightly connected with the implementation of delivery guarantees. In a deterministic system, a state of a non-commutative operation can be reprocessed consistently with the previous output elements. Hence, there is no need to save the results of non-commutative operations before output delivery. This property opens a wide range for performance optimizations.

In this work, we demonstrate that the property of determinism can mitigate an overhead on~\eo\. 
In order to prove the feasibility of efficient~\eo\ over determinism, we introduce  fault tolerance protocols on top of {\em drifting state} model~\cite{we2018adbis}. 
%  This optimistic technique provides determinism with low overhead. 
We show that lightweight determinism together with the results of the formal inference provides for~\eo\ for a negligible  cost.
%   It is verified by the experiments on a realistic  problem.

The contributions of this paper are the following: 
\begin{itemize}
    \item Formal model of a distributed stream processing  and   a   definition of  delivery guarantees 
    \item Demonstration that in non-deterministic systems providing~\eo\, the lower bound of latency is the duration of state snapshotting
    \item Techniques for lightweight implementation of~\eo\ guarantee on top of a deterministic engine
    \item Study of the practical feasibility of the proposed approaches
\end{itemize}

The rest of the paper is structured as follows: 
in section~\ref{fs-preliminaries} the notion of consistency applied to a stream processing is discussed, 
we introduce our formal framework in section~\ref{fs-formalism}, 
existing implementations of~\eo\ in terms of the proposed formal framework are described in section~\ref{fs-eo-impl}, 
implementation details of~\eo\ over determinism are mentioned in section~\ref{fs-consistency-section}, 
experiments that demonstrate the feasibility of the proposed concept are detailed in section~\ref{fs-experiments-seciton}, 
and we discuss prior works on the topic in section~\ref{fs-related-seciton}. 


\section {Related work}
 %%% fs-state-related - Related work

\label {fs-related-seciton}

Techniques for providing exactly-once~\cite{Carbone:2017:SMA:3137765.3137777, Akidau:2013:MFS:2536222.2536229, Zaharia:2012:DSE:2342763.2342773} are discussed   in details in sections~\ref{fs-intro-seciton} and~\ref{fs-eo-impl}. Many prior works in the field of stream processing do not consider exactly-once maintenance. 
For instance, Aurora~\cite{Abadi:2003:ANM:950481.950485} and Borealis~\cite{abadi2005design} do not provide any guarantees on data at all. Some other systems provide only partial consistency. Apache Storm~\cite{apache:storm} supports message tracking mechanism that prevents the loss of data. 
However, exactly-once semantics is not provided, because duplicates are still possible. Twitter Heron, that was presented as the next generation of Apache Storm~\cite{Kulkarni:2015:THS:2723372.2742788}, does not provide for exactly-once as well. 
Samza~\cite{Noghabi:2017:SSS:3137765.3137770} also implements fairly similar to Storm model and has the same consistency guarantees.

Prior works on stream processing formalization concentrate on operations specification rather than delivery guarantees. Logical foundation for specifying streaming computations is discussed in~\cite{alur2018interfaces}. Declarative algebraic notations for the streaming queries are introduced in~\cite{halle2014formalization}. Another direction in streaming formalization is designing frameworks to define operations semantics~\cite{beck2018lars}.

A comprehensive analysis of different approaches to consistency and fault tolerance in stream procesing is provided in~\cite{Wang:2019:LSF:3341301.3359653}.  The authors observe that, in order to provide consistency guarantees, execting systems have to choose between checkpoints and lineage. The checkpoints tend to be expensive during the normal execution, while lineage usually results in expensive failure recovery. In terms of the above classification our work belongs to the tatter group, as it relies on the total ordering of input items and  re-plays  all items that weren't delivered before failure. 

THe authors of \cite{Wang:2019:LSF:3341301.3359653} also describe an approach that allows non-deterministic processing during normal execution but reproduce exactly same lineage during recovery.  This results in effectively deterministic computation. In contrast with our work, the lineage is produced dynamically during the execution.  This approach still esults in significant computational overheas, but  the lineage may be saved asynchonously and thus produces ònly minor impact on the latency. 


\section {Efficient deterministic model}
%%% fs-state-model - Model

\label {fs-model-section}

This section contains the high-level view of our model and a few implementation details.

\subsection{Data flow}
Define stream and data items.

\subsection{Computational flow}
Flow is represented in the form of a graph.

\subsection{Ordering assumptions}
The main idea: we define a total ordering on items to achieve determinism.

\subsection{Supported operations}
Map/grouping and their properties. Grouping's window can be > 2: e.g. merge of 3 streams.

\subsection{Stateful transformations}
Pipeline for any stateful transformation using only map and grouping. 

\subsection{Consistency guarantees}
At most once, at least once, exactly once. Exactly once is achieved if failures do not occur.

\subsection{Implementation notes}

\begin{itemize}
    \item Physical operations and hash units: state is identified by hash unit
    \item Fronts and rears
    \item Deterministic model is real! There are extra items, they can be cleared at the barrier, if we know min time  %<-- Here we introduce the barrier!
    \item Acker - min time provider. Acker observes progress of the full system. Can play the role of the master %<-- Here we introduce min time notifications
\end{itemize}

\section{Fault tolerance}
%%% fs-state-consistency - Fault tolerance

\label {fs-consistency-seciton}

In this section we describe how consistency guarantees can be provided within the proposed model in case of failures. Initially, we define three mechanisms which are necessary both for consistent processing and for fast and reliable recovery. After that, we demonstrate how they behave in recovery processes and why consistency semantics is preserved.

It is worth to mention that we rely strongly on the deterministic properties of the proposed model. Particularly, we expect that several independent runs with the same input data produce exactly the same result. This property is guaranteed by the fact that all supported operations produce deterministic results up to the order of input items. These features allow us to build efficient techniques for barrier flushing and operations' state management, which almost do not depend on each other.  

\subsection{Input replay}
Input replay functionality is commonly used to restore computational process after a failure in stream processing systems [?]. The key idea here is that in case of failure, previously released input data can be produced again. Usually, replay is started not from the beginning, but from some determined point. Like other stream processing solutions, our system requires from producers an ability to replay input data. The only difference is that the initial point of replay is defined in terms of the global time. Practically, the role of global times can be played by any monotonically increasing sequence, e.g. offsets in Apache Kafka or the values of a logical clock. Therefore, this requirement is not a strong limitation for real-life deployments.

\subsection{Barrier flushing}
Notifications of the new minimal time within the stream are sent by the acker with monotonically increasing global times. Barrier receives these notifications and releases output items with monotonically increasing global times as well. Hence, if barrier knows the global time of the last released item $GT_{last}$, it can process data in idempotent fashion simply by filtering out any items with a global time less than or equal to $GT_{last}$. 

Therefore, the barrier has its own state - $GT_{last}$, and this state is applied for avoiding duplicates in case of failure and subsequent input replay. Exactly-once semantics is possible only if there are no any inconsistencies between released items and $GT_{last}$, so there is a need to atomically output items and update $GT_{last}$. To solve this problem, we require the following output protocol with data consumer:

\begin{itemize}
    \item When minimal time is received, barrier send special output bundle to the data consumer. This bundle contains all corresponding output items and $GT_{last}$. The consumer must acknowledge that it received the bundle
    \item Barrier does not send new output bundle until the previous one is not acknowledged
    \item Consumer must return last received bundle on barrier's request 
\end{itemize}

This protocol guarantees that $GT_{last}$ and released items are always consistent with each other. It implies that barrier can request the last released bundle and fetch $GT_{last}$ after recovery to avoid duplicates and preserve exactly-once semantics. It should be noted that for at least once semantics, contract with the consumer can be relaxed: it is not required to store the last received bundle.

\subsection{State snapshotting}
Actually, in case of failures, a producer can replay all previously sent data items on recovery without loss of exactly-once semantics. Such behavior is achieved because processing results are deterministic and all possible duplicates are filtered out at the barrier. However, this approach can be memory and time demanding. 

Therefore, there is a need to periodically take a snapshot of operations' state and to replay on recovery from the global time that corresponds to the previous snapshot. In order to start processing after recovery from global time $GT$, two conditions must be satisfied:
\begin{enumerate}
    \item All stateful operations can restore the state that could be reached after consuming all data items with global times in range $[0..GT)$ 
    \item All bundles that contain items with global times in range $[0..GT)$ have been already acknowledged by data consumer 
\end{enumerate}

It is convenient to piggyback on the acker's notifications of the minimal time within the stream for triggering state saving. Let $GT_{min}$ be the last received minimal time within the stream. Grouping's buckets in practice are lists sorted by global time. Because of the guarantees provided by minimal time notification, these lists cannot be changed at any position before the item that corresponds to $GT_{min}$. At the same time, because of the grouping semantics, state that could be reached after consuming all data items with global times in range $[0..GT_{min})$ is just a window-sized sublist that is located in the immutable part of the bucket. Thereby, state snapshotting can be done asynchronously not only with state snapshotting of other operations but even with the computational process of the current operation.

Considering the properties mentioned above, the protocol of state snapshotting is the following:

\begin{itemize}
    \item On some minimal time event, acker decides to initiate state snapshotting. The decision is made by its internal logic, i.e. by the elapsed time since the last snapshot. Acker sends a request for the new snapshot along with minimal time notification
    \item When grouping operation receives the request, it asynchronously saves the state and sends back the acceptance message   
    \item When barrier receives the request, it waits until producer acknowledges all in-flight bundles, and then sends back the acceptance message to acker
    \item When acker receives all acceptance messages, it writes the global time of the snapshot to ZooKeeper 
\end{itemize}

It is worth to note that any persistent key-value storage can be applied as a storage for a state. Hash unit of the corresponding grouping operation concatenated with received minimal time can be used as a key. Waiting until barrier sends out all in-flight bundles is the only dependency between state snapshotting and barrier flushing mechanisms, that does not practically influence end-to-end processing latency. 

This protocol satisfies two proposed above conditions regarding the global time that is written to ZooKeeper. Hence, this global time can be safely used as a resuming point after recovery.

\subsection{Recovery protocols}
We can handle package loss and node failure. Network partitioning is out of the scope of stream processing.

\subsubsection{Package loss}
Acker determines. Replay.

\subsubsection{Node failure}
The state can be restored. Replay.

\subsubsection{Barrier failure}
Barrier's state can be restored. Replay.

\section {Experiments}
%%% fs-state-experiments - Experiments

\label {fs-experiments-seciton}
The series of experiments were performed in order to analyze the overall performance of the system's prototype. We apply building an inverted index as stream processing task for the evaluation because it satisfies the following properties:

\begin{itemize}
    \item The computational pipeline of the task contains network shuffle that can violate the ordering constraints. Therefore, this task is able to fairly estimate the performance of the proposed deterministic model
    \item Consistency guarantees are strongly required because the inconsistent index does not make sense for many applications
\end{itemize}

Building inverted index can be implemented as a MapReduce transformation: 

\begin{itemize}
    \item Map phase includes conversion of input documents into the key-value pairs {\it (word; word positions within the page)}
    \item Reduce phase consists of combining word positions for the corresponding word into the single index structure 
\end{itemize}

We assume that reduce phase outputs the change records of the inverted index structure, to make this algorithm suitable for stream processing systems. It implies that each input page triggers the output of the corresponding change records of the full index. 

Notably, building an inverted index in a streaming manner can be the halfway task between the generation of documents and consuming index updates by search infrastructure. In the real-world, such scenario can be found in freshness-aware systems, e.g., news processing engines.
 
In \FlameStream\ this algorithm is implemented as the typical conversion of MapReduce transformation, which is shown in section~\ref{fs-model-section}. Inverted index structure plays the role of an accumulator, and the accumulator map produces the most recent changes of this structure if any.

By the notion of {\it latency} we assume the time between two events: 

\begin{enumerate}
    \item Input page is taken into the stream
    \item All the change records for the page leave the stream
\end{enumerate}

Our experiments were performed on the cluster of Amazon EC2 micro instances with 1GB RAM and 1 core CPU. We used 10000 Wikipedia articles as a dataset. 

\subsection{Overhead and scalability}
Show that our model is scalable and provides low overhead.

\subsection{Comparison against Apache Flink}
Show that our prototype outperforms Flink with at least once/exactly once mode on. Explain such behavior.


\section* {Conclusion and future work}
%%% fs-state-conclusion - Conclusion

\label {fs-conclusion-seciton}

We introduced a formal, conceptual framework for modeling consistency properties for any stream processing system. We demonstrated how the behavior of state-of-the-art research and industrial systems could be described in terms of the proposed framework. It was shown that the property of determinism is tightly connected with the concept of of exactly-once. We proved that non-deterministic systems must persistently save a state of non-commutative operations before output delivery in order to achieve exactly-once. Most of the state-of-the-art stream processing systems use one of the following approaches to overcome this problem: 

\begin{itemize}
    \item Inherit exactly-once from batch processing using small-sized batches (micro-batching)
    \item Apply distributed transaction control protocols which guarantee that states are saved before delivery of elements affected by these states
    \item Write results of an operation to external storage on each input element
\end{itemize}

All these methods experience difficulties with working under low-latency requirements (less than a second). In the first case, latency cannot be lower than the batching period, in the second case, the distributed two-phase commit may result in a significant increase of latency, while in the third case latency is bounded below by the duration of external writes.

Using our formal inference, we designed mechanisms for achieving exactly-once on top of {\em drifting state} technique introduced in our previous work~\cite{we2018adbis}. Drifting state provides inexpensive determinism due to optimistic nature and low overhead on a single buffer per any stateful data flow. Because of the determinism, the protocols provide the following features:

\begin{itemize}
    \item Elements are processed in a pure streaming manner without input buffering
    \item The processes of business-logic computations, state snapshotting and delivery of output items work asynchronously and independently
    \item Exactly-once is preserved
\end{itemize}

We implemented the prototype of the proposed technique to examine its performance. Our experiments demonstrated that the introduced protocols for fault tolerance are scalable and provide remarkably low overhead within different computational layouts. Furthermore, the comparison with the industrial stream processing solution indicated that our prototype could provide significantly lower latency under exactly-once requirement.

% Regarding future work, we plan to design a distributed version of ~\Acker\ and to adopt the drifting state for iterative stream processing. This task seems natural for our model because it already handles cycles in a data flow. 

\setmonofont[Mapping=tex-text]{CMU Typewriter Text}
\bibliographystyle{ugost2008ls}
\bibliography{../../bibliography/flame-stream}
\end {document}


\endinput
you can put whatever here

