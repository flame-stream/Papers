In this section, we remind main concepts of stream processing, that we use further in this paper. Basically, a distributed stream processing system is a shared-nothing distributed runtime, that can handle a potentially unlimited sequence of input items. Each item can be transformed several times before the final result is released from the system. Elements can be processed one-by-one or grouped into small input sets, usually called {\em micro-batches}. 

An element has been {\em entered} the system if the system is aware of it and takes some kind of responsibility for its processing. This concept can be implemented in a different way in different systems. For example, in Flink the fact that the element has been entered means that the element has arrived at {\em Source} vertex. In Spark Streaming, element enters, when it is read or received by an input agent also called {\em Source}. 

An element has {\em left} the system if an element has been released to a consumer. Since that moment system cannot observe it anymore. This concept can also be mapped differently onto various systems. For instance, in Spark Streaming element leaves when it is pushed to output operation, .e.g., written to HDFS or file. In Flink element leaves when it leaves from {\em Sink} vertex.   

It is important to note that input and output elements cannot be directly matched to each other due to a possibility of complex transformations within the system. For instance, a single input element can be transformed into multiple ones. After the split, resulting elements are able to be processed in completely different ways and even influence each other. Hence, in general, it is impossible to determine which input element triggers an output one.

The way how a system transforms input elements is usually defined in the form of a {\em logical graph}. A logical graph is a graph, typically provided by a user, where vertices are {\em business-logic} operations and edges determine the order of processing. Each user-defined operation may be {\em stateless} or {\em stateful}. States of operations are usually managed by the system itself to prevent inconsistencies.

The main difference between the state of user-defined operation and an ordinary element is that the state is consumed, updated, and produced by the same operation. In~\cite{we2018adbis} it is shown that operations state can practically be an ordinary data flow element for any stateful operation. Therefore, we can assume that the states of operations are just special elements in a data flow.

% In order to make our mathematical model independent of any implementation, we consider streaming consistency guarantees as correspondences between input and output streams and system state. If we formulate guarantees only in these terms, they can be applied to any system. However, the question of what to consider as a system state is a little bit sophisticated. At a very high-level, we can define a state as a {\em information} about input elements, which have been previously entered the system.  

% An obvious idea is to say that system state is a state of user-defined operations, i.e., so-called {\em business logic} state. The main purpose of the states of operations is to accumulate the information about input items. It allows a system to not store all previous input elements in order to process a new one in a stateful user-defined operation. However, output items can be affected not only by input ones and operations state but also by other elements, which are currently in the system. For instance, if cycles are allowed in a logical graph, cycling elements can affect output ones, but they do not belong to a state. In-flight elements can also influence the result, being not in the state. These examples demonstrate the evident fact that information about input elements can be contained not only in the states of operations but in data flow elements as well.
