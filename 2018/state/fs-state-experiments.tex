%%% fs-state-experiments - Experiments

\label {fs-experiments-seciton}

\subsection{Setup}
The series of experiments were performed in order to analyze the overall performance of the system's prototype. We apply building an inverted index as stream processing task for the evaluation because it satisfies the following properties:

\begin{itemize}
    \item The computational pipeline of the task contains network shuffle that can violate the ordering constraints. Therefore, this task is able to fairly estimate the performance of the proposed deterministic model
    \item Consistency guarantees are strongly required because the inconsistent index does not make sense for many applications
\end{itemize}

Building inverted index can be implemented as a MapReduce transformation: 

\begin{itemize}
    \item Map phase includes conversion of input documents into the key-value pairs {\it (word; word positions within the page)}
    \item Reduce phase consists of combining word positions for the corresponding word into the single index structure 
\end{itemize}

We assume that reduce phase outputs the change records of the inverted index structure, to make this algorithm suitable for stream processing systems. It implies that each input page triggers the output of the corresponding change records of the full index. 

Notably, building an inverted index in a streaming manner can be the halfway task between the generation of documents and consuming index updates by search infrastructure. In the real-world, such scenario can be found in freshness-aware systems, e.g., news processing engines.
 
In \FlameStream\ this algorithm is implemented as the typical conversion of MapReduce transformation, which is shown in section~\ref{fs-model-section}. Inverted index structure plays the role of an accumulator, and the accumulator map produces the most recent changes of this structure if any.

By the notion of {\it latency} we assume the time between two events: 

\begin{enumerate}
    \item Input page is taken into the stream
    \item All the change records for the page leave the stream
\end{enumerate}

Our experiments were performed on the cluster of 10 Amazon EC2 micro instances with 1GB RAM and 1 core CPU. We used 10000 Wikipedia articles as a dataset. Documents per second input rate is 50. RocksDB~\cite{rocksdb} is used as a storage for the state.

\subsection{Overhead and recovery}
The performance of the proposed deterministic model within the same stream processing task is deeply analyzed in~\cite{we2018seim}. In this paper we aim to evaluate the overhead on providing consistency guarantees and the time needed for the full recovery.

Figure~\ref{performance} shows the latencies of \FlameStream\ within distinct times between checkpoints, and at most once, at least once, and exactly once consistency semantics. As expected, the overhead on at least once and exactly once semantics is low (less than 10 ms) and it does not depend on the time between checkpoints. Slight overhead can be explained by the fact that asynchronous state snapshotting is executed on single-core nodes. The time between checkpoints does not influence latency because barrier flushing and state snapshotting mechanisms are no connected in our model. The latencies under at least once and exactly once semantics are the same because the only difference between them regarding our model is in contract with data consumer.

System's behavior in case of failure and recovery is demonstrated in Figure~\ref{recovery}. It is shown that the system is able to perform recovery processes in an adequate time. Existing latency fluctuations are caused by replay process, JVM restart, etc.

\begin{figure}[htbp]
  \centering
  \includegraphics[width=0.58\textwidth]{pics/comparison}
  \caption{The latencies of \FlameStream\ during failure and recovery}
  \label {recovery}
\end{figure}

\subsection{Comparison against Apache Flink}
One of the most important goals of the experiments is the performance comparison with an industrial solution regarding latency. Apache Flink has been chosen for evaluation because it is state-of-the-art stream processing system that provides similar functionality and achieves low latency in the real-world scenarios~\cite{S7530084}. 

For Apache Flink, the algorithm for building the inverted index is adopted by the usage of {\it FlatMapFunction} for map step and stateful {\it RichMapFunction} for reduce step and for producing the change records. Order enforcing before reduce is implemented using custom {\it ProcessFunction} that buffers all input until corresponding low watermark is received. Watermarks are sent after each document. The network buffer timeout is set to 0 to minimize latency.

{\it FsStateBackend} with the local file system is used for storing the state, because {\it RocksDBStateBackend} requires saving state to RocksDB on each update that leads to an additional overhead. {\it FsStateBackend} stores state on the disk only on checkpoints and do not provide an additional overhead against RocksDB storage in \FlameStream, so it is fairer to use it rather than {\it RocksDBStateBackend} for comparison purposes.

In this paper, we compare $50^{th}$, $75^{th}$, $95^{th}$, and $99^{th}$ percentile of distributions, which clearly represent the performance from the perspective of the users' experience.

Figure~\ref{performance} demonstrates the comparison of latencies between \FlameStream\ and Flink within distinct times between checkpoints, and different consistency semantics. At the initial point, \FlameStream\ provides lower latency for at most once semantics. Such behavior is explained by the features of the optimistic approach for handling out-of-order items and is investigated in details in~\cite{we2018seim}. The latencies of both \FlameStream\ and Flink are slightly higher under at least once semantics. As it was mentioned above, it can be explained by the single-core configuration of the nodes. For at most once and at least once semantics, the latencies of \FlameStream\ and Flink do not vary a lot because both systems do not buffer output items for a long time before releasing. However, for exactly once semantics, Flink's latency is dramatically higher and it directly depends on the time between checkpoints. Nevertheless, such behavior is expected, because unlike \FlameStream, Flink needs to take state snapshot and release output items within a single transaction in order to preserve exactly once semantics. As it was discussed above, it is the consequence of the lack of determinism in the computational model. Therefore, output items cannot be released until the transaction is committed and this fact significantly increases the latency. 

\begin{figure}[htbp]
  \centering
  \includegraphics[width=0.58\textwidth]{pics/comparison}
  \caption{The comparison in latencies between FlameStream and Flink regarding different consistency semantics within 10 nodes, 50 documents per second rate, and distinct delays between checkpoints}
  \label {performance}
\end{figure}