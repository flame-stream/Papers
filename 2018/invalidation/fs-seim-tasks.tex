%%% fs-seim-tasks - Tasks that require in-order input

\label {fs-tasks}

In this section we outline common problems that require the order persistence of input items and describe a couple of computation scenarios, which can be found in many real-life projects.

\subsection{Tasks requiring complete event retrieval}
The processing of the single event could be split into multiple independent parts that are executed in parallel. After execution finishes the results must be combined into a single cumulative item. This task could be naturally implemented using order guarantees: the final part of the task could be flagged and receiving the flagged result guarantees that the rest of the operation is completed. Unfortunately, as it was shown in Figure~\ref{break-order-dataflow}, independent processing via different paths can lead to reordering.

As an example, we can mention the computation of inverted index. Pipeline shown in Figure~\ref{break-order-dataflow} can be applied for the task. In this case, operation 1 accepts documents from the input and for each word produces corresponding positions. Operation 2 accepts pairs of word and positions and computes changelog of the inverted index for each word. In order to produce changes for each document in the form of single update, there is a need for retrieval all changelogs regarding the document.

\subsection{Tasks that depend on the order of input events}
This class includes all non-commutative operations. Such tasks strictly require the order of input items, because there are no any other methods to compute a valid result.

Generally, this class of examples includes all windowed aggregations. Moving average calculation over a numerical stream is a typical case. Even if values inside window could be arbitrary reordered, the order between windows is required to ensure that incomplete windows are not produced.