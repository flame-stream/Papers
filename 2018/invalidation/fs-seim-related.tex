%%% fs-seim-related - Related work

\label {fs-related}

Research works on this topic consider different methods of handling out-of-order items. Most of them are based on buffering.

K-slack technique can be applied, if network delay is predictable \cite{Babu:2004:EKC:1016028.1016032, Li:2007:ESP:1270388.1270975}. The key idea of the method is the assumption that an event can be delayed for at most K time units. Such assumption can reduce the size of the buffer. However, in the real-life applications, it is very uncommon to have any reliable predictions about the network delay.

IOP and OOP architectures, that are mentioned in the section~\ref{fs-typical}, are popular within research works and industrial applications. IOP architecture is applied in \cite{Cranor:2003:GSD:872757.872838, Abadi:2003:ANM:950481.950485, Arasu:2006:CCQ:1146461.1146463, Ding:2004:EWJ:1031171.1031189, Hammad:2003:SSW:1315451.1315478, Hammad:2005:OIE:1116877.1116897}. OOP approach is introduced in \cite{Li:2008:OPN:1453856.1453890} and it is widely used in the industrial stream processing systems. For instance, Flink \cite{carbone2015apache} and Millwheel \cite{Akidau:2013:MFS:2536222.2536229} apply OOP approach.

Regarding optimistic techniques, there is much less scientific and industrial activity. In \cite{Wei:2009:SSO:1559845.1559973} so-called {\it aggressive} approach is proposed. This approach provides the idea that operation can immediately output {\it insertion} message on the first input. After that, if that message became invalid, because of the arrival of out-of-order items, an operation can send {\it deletion} message to remove the previous result and then send new insertion item. The idea of deletion messages is very similar to our tombstone items. However, authors describe their idea in an abstract way and do not provide any techniques to apply their method for arbitrary operations.

Yet another optimistic strategy is detailed in \cite{Li2011}. Nevertheless, this method is probabilistic and supports only the limited number of query operators.