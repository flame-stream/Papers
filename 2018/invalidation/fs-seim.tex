% IEEE Paper Template for US-LETTER Page Size (V1)
% Sample Conference Paper using IEEE LaTeX style file for US-LETTER pagesize.
% Copyright (C) 2006-2008 Causal Productions Pty Ltd.
% Permission is granted to distribute and revise this file provided that
% this header remains intact.
%
% REVISION HISTORY
% 20080211 changed some space characters in the title-author block
%
\documentclass[10pt,conference,letterpaper]{IEEEtran}
\usepackage{times,amsmath,epsfig}

\usepackage{graphicx}
\usepackage{algorithm} % for algorithms
\usepackage{algpseudocode}
\usepackage{booktabs} % For formal tables
%\usepackage{amsthm} % For claims  ???????

%
\title{An optimistic approach to handle out-of-order events within analytical stream processing}
%
\author{%
% author names are typeset in 11pt, which is the default size in the author block
{Igor E. Kuralenok{\small $~^{\#1}$}, 
    Nikita Marshalkin{\small $~^{\#2}$},
    Artem Trofimov{\small $~^{\#3}$}, 
        Boris Novikov{\small $~^{\#4}$} }%
% add some space between author names and affils
\vspace{1.6mm}\\
\fontsize{10}{10}\selectfont\itshape
% 20080211 CAUSAL PRODUCTIONS
% separate superscript on following line from affiliation using narrow space
$^{\#}$\,JetBrains Research\\
Saint Petersburg, Russia\\
\fontsize{9}{9}\selectfont\ttfamily\upshape
%
% 20080211 CAUSAL PRODUCTIONS
% in the following email addresses, separate the superscript from the email address 
% using a narrow space \,
% the reason is that Acrobat Reader has an option to auto-detect urls and email
% addresses, and make them 'hot'.  Without a narrow space, the superscript is included
% in the email address and corrupts it.
% Also, removed ~ from pre-superscript since it does not seem to serve any purpose
$^1$\,ikuralenok@gmail.com   %  \\
$^2$\,marnikitta@gmail.com    %  \\
$^3$\,trofimov9artem@gmail.com   % \\
$^4$\,borisnov@acm.org 
}
%
\newcommand{\PicScale}{0.5}
\newcommand {\FlameStream} {FlameStream}

\begin{document}
\maketitle
%
\begin{abstract} 
Abstract
\end{abstract}

% NOTE keywords are not used for conference papers so do not populate them
% \begin{keywords}
% keyword-1, keyword-2, keyword-3
% \end{keywords}
%
\section {Introduction}
State-of-the-art stream processing systems like Flink \cite{carbone2015apache} can provide low-latency and high-throughput in distributed environment. However, there are tasks which require the order of input elements. Even if input items arrive monotonically, they can be reordered within system by shuffling and/or asynchronous operations. The typical way to achieve in-order processing is to lock operation's input for some time to ensure that there are no out-of-order items. Such technique can lead to significantly large latency, especially if processing pipeline contains several operations that require ordered input. Alternative option is to handle out-of-order items as a special case in business logic. The main disadvantage of this way is the complexity of business logic maintanance. In this paper we propose an optimistic approach, which can reduce waiting time for multiple operations with order constraints, but does not affect the complexity of business logic.

\section{Stream processing dataflow}
Here we should introduce preliminaries: stream, items, meta-information. How stream processing is typically organized: shared-nothing, queue-based.

\section{Tasks that require in-order input}
Introduce some problems. Why ordering is so important? Grouping should be mentioned :)

\section{Typical solutions}
How this problem can be solved using Flink, Storm, etc.? Observe several types of windows introduced in Google Dataflow \cite{Akidau:2015:DMP:2824032.2824076}.

\section{Optimistic approach}
Introduce the main idea of our technique. Draw parallels with transactions/transactional memory. 

Working example - grouping. 

Discuss limitations.

Barrier as a single point of locking. 

\section{Experiments}
Show that our idea works. Problem: how to perform experiments without introducing our MapReduce model?

\section{Related work}
Seems to be related: \cite{4279071}, \cite{Wei:2009:SSO:1559845.1559973}, \cite{Mutschler:2014:ASP:2659232.2633686}.

\section{Conslusion}
We are winners :)

\bibliographystyle{IEEEtran}
\bibliography{../../bibliography/flame-stream}

\end{document}

\endinput
