%%% fs-seim-optimistic - Optimistic approach

\label {fs-optimistic}

The main idea of our method is to represent stateful transformations as a sequence of map and windowed grouping operations and handle out-of-order items within them. 

Following our approach, we make some assumptions about stream processing system, that is suitable for applying it. Such system must support meta-information on data items, allow cycles in logical graph, and its set of predefined operations must be sufficient to express map and windowed grouping operations.

In this section, firstly, we show that any stateful transformation can be implemented using the combination of windowed grouping and map operations. After that, we demonstrate an optimistic approach to handle out-of-order items within these operations. Eventually, the limitations of such technique are discussed.

\subsection{Semantics of map and windowed grouping operations}

\subsubsection{Map}
Map transforms input item into a sequence of its derivatives according to a function on item's payload provided by user. This sequence can consist of any number of items or even be empty.

\subsubsection{Windowed grouping}
Windowed grouping is a stateful operation with a numeric parameter {\it window}. It is supposed that payloads of input items of grouping operation have key-value form. The state of this operation is represented by a set of buckets, one for each key. Windowed grouping has the following semantics:

\begin{itemize}
    \item Each input item is put into corresponding bucket at position defined by its meta-information
    \item The output of grouping operation is a window-sized tuple of last items in the corresponding bucket. If bucker size is less than window, the output contain full bucket
\end{itemize}

The following example illustrates the semantics of the windowed grouping operation. In this example, payloads of input items are represented as natural numbers: 1, 2, 3, etc. The hash function returns 1 if the number is even and 0 otherwise. If the window is set to 3, the output is:

\[(1), (2), (1|3), (2|4), (1|3|5), (2|4|6), (3|5|7), (4|6|8)...\]

\subsection{Stateful transformations using defined operations}
Figure ~\ref{stateful-schema} shows the part of the logical pipeline, that can be used for stateful transformation. The input of windowed grouping operation is supposed to be ordered. There are several functional steps to produce output and update state. These steps can be detailed by considering two cases:

\begin{itemize}
    \item When the first item arrives at grouping, it is inserted into the empty bucket. Grouping outputs single-element tuple, and then it is sent to combined map. Combined map generates state object and sends it back to the grouping in the form of ordinal data item. This new state item is assigned by meta-information of the item in tuple. At the same time, combined map can generate some output and send it further down the stream.
    \item When new regular input item arrives at windowed grouping, it is inserted into the corresponding bucket's tail. Because of the fact that grouping accepts items in the right order, input item is grouped into the tuple with previously generated state item. The next map operation combines new item and previous state into the new one. New state is sent back to the grouping with the meta-information of the newly arrived item. As in the preceding case, combined map can generate some additional output.
\end{itemize}

\begin{figure}[htbp]
  \centering
  \includegraphics[width=0.48\textwidth]{pics/stateful-schema}
  \caption{The part of the logical pipeline for stateful transformation}
  \label {stateful-schema}
\end{figure}

\subsection{Handling out-of-order events}

\subsection{Limitations}




