%%% fs-seim-typical - Typical solutions

\label {fs-typical}

There are two the most common methods that are used to implement order-sensitive operators: in-order processing (IOP) \cite{Arasu:2006:CCQ:1146461.1146463, Cranor:2003:GSD:872757.872838, hammad2004optimizing} and out-of-order processing (OOP) \cite{Li:2008:OPN:1453856.1453890}.

\subsection{IOP}

According to IOP approach, each operation must enforce the total order on elements. Simple operators, such as projection and selection, apply function and propagate output items down the stream without any additional buffering. However, even stateless operations may require buffering. Figure~\ref{iop} shows the union operation that combines multiple streams into the one. Both input streams are ordered, as predecessors must meet ordering constraint. Nevertheless, if there is arrival time skew between input streams, the union must buffer the earlier stream to produce ordered output. It is known that IOP is memory demanding and has unpredictable latencies and limited scalability \cite{Li:2008:OPN:1453856.1453890}.

\begin{figure}[htbp]
  \centering
  \includegraphics[width=0.30\textwidth]{pics/iop}
  \caption{IOP union operation. Due to delay of the upper stream operation must buffer elements}
  \label {iop}
\end{figure}

\subsection{OOP}

OOP is an architecture of streaming systems that does not require order maintenance if it is not needed. In the case of ordering requirements, OOP buffers input items similar to the IOP approach. To flush buffers, OOP systems use progress indicators such as punctuations \cite{Tucker:2003:EPS:776752.776780}, low watermarks \cite{Akidau:2013:MFS:2536222.2536229}, or heartbeats \cite{Srivastava:2004:FTM:1055558.1055596}. They go through the stream as ordinal items, but do not trigger business-logic of operations. Each punctuation carries meta-information and promises that there are no any elements with lesser meta-information. Therefore, punctuations must be monotonic, but data items between two consecutive punctuations can be arbitrarily reordered. Punctuations are periodically yielded by data sources.

As an example of OOP approach, timed window operation can be noted. Window operation buffers partial aggregates until punctuation for specific time arrives. After that, window flushes corresponding buffers and propagates punctuation to the next operation down the stream.

OOP resolves some of the downsides of the IOP but it has several flaws too. Even if the input stream is totally ordered, the operation must wait for the punctuation. Figure~\ref{oop} illustrates such case. Bottom window is complete but must be blocked until the punctuation for item 11 arrives. Another issue of OOP is that periodical flushes can result in load bursts and latency increasing. 

\begin{figure}[htbp]
  \centering
  \includegraphics[width=0.48\textwidth]{pics/oop}
  \caption{OOP sliding window, range=3, slide=1. Operation must block lower window until next punctuation arrival }
  \label {oop}
\end{figure}
