\documentclass[sigconf]{acmart}

\usepackage{graphicx}
\usepackage{algorithm} % for algorithms
\usepackage{algpseudocode}
\usepackage{booktabs} % For formal tables
\usepackage{amsthm} % For claims
\usepackage{bbm} % indicator function

% table
\usepackage[flushleft]{threeparttable} % http://ctan.org/pkg/threeparttable
\usepackage{booktabs,caption}

\theoremstyle{remark}

\settopmatter{printacmref=true, printccs=true, printfolios=true}
\pagestyle{empty} % removes running headers

\newcommand{\PicScale}{0.5}
\newcommand {\FlameStream} {FlameStream}
\begin{document}

% \copyrightyear{2019} 
% \acmYear{2019} 
% \acmConference[BIRTE 2019]{Real-Time Business Intelligence and Analytics}{August 26, 2019}{Los Angeles, CA, USA}
% \acmBooktitle{Real-Time Business Intelligence and Analytics (BIRTE 2019), August 26, 2019, Los Angeles, CA, USA}
% \acmPrice{15.00}
% \acmDOI{10.1145/3350489.3350491}
% \acmISBN{978-1-4503-7660-0/19/08}

\title {Distributed Change Detection in Data Sreams}

\author{Mikhail Yutman}
\affiliation{%
  \institution{Yandex}
  \city{Saint Petersburg}
  \country{Russia}}
\email{myutman@yandex-team.ru}

\author{Artem Trofimov}
\affiliation{%
  \institution{Yandex}
  \city{Saint Petersburg}
  \country{Russia}}
\email{tomato@yandex-team.ru}

\author{Igor Kuralenok}
\affiliation{%
  \institution{Yandex}
  \city{Saint Petersburg}
  \country{Russia}}
\email{solar@yandex-team.ru}

\author{Boris Novikov}
\affiliation{%
  \institution{National Research University Higher School of Economics}
  \city{Saint Petersburg}
  \country{Russia}}
\email{borisnov@acm.org}

\begin{abstract}

% High-frequent stream processing is a very important research area. Powerful streaming algorithms are widely used in banking and financial analytics as the data is too huge to store it and also arrives at high frequencies so it needed to be processed as soon as possible to make a timely decision. Change-point detection on a streaming data is used to track changes in credit card transactions, stock market ticks or computer network traffic. By the way, data can be produced too frequent, so it's desirable to distribute change-point detection over multiple nodes. Also values of statistics can be dependent and it's a challenge. In this paper we introduce a distributed stream processing algorithm for change-point detection and try to explore importance of dependencies between incoming statistics.

Change detection algorithms are widely used in data flows that monitor suspicious activity in credit card transactions, stock market ticks, or computer network traffic. Another application is to track unexpected fluctuations in data used for training online machine learning models. To make these data flows scalable, it is natural to run them on distributed stream processing systems such as Flink, Storm, Spark Streaming, etc. Usually, we don't have an information about probability distribution of data. However, change detection algorithms either aim to a case when all data stream items are handled by a single worker or require an information about probability distribution of data. In this work, I complement an existing efficient change detection algorithm that doesn't need information about probability distribution of data to be suitable for a distributed stream processing engine. However, scalability has its price. Experiments demonstrated an emerging trade-off between scalability together with false alarm probability on the one side, and change detection accuracy together with detection latency on the other side.

% To  By the way, data can be produced too frequent, so it's desirable to distribute change-point detection over multiple nodes. Also values of statistics can be dependent and it's a challenge. In this paper we introduce a distributed stream processing algorithm for change-point detection and try to explore importance of dependencies between incoming statistics.

\end{abstract}

\maketitle

\thispagestyle{empty}

\section {Introduction}
The change-point detection problem on a stream data is formulated in a following way. There is a stream of elements sampled from a probability distribution refers to a family of distributions $F(\theta)$ with parameter $\theta$. Changes happen after elements $M_1, M_2, \dots$ so that parameter of distribution is $\theta_0, \theta_1, \theta_2, \dots$ in the element ranges \\ $\left[ 0; M_1 \right), \left[ M_1; M_2 \right), \left[M_2; M_3 \right), \dots$ correspondingly. The problem is to detect change in distribution parameter with the smallest possible latency (number of elements between change and detection).

Lots of stream processing algorithms can be sensitive to an unexpected change in a data stream distribution. For example, in stream classification problem change of tf-idf features can lead to a massive misclassification. However, most stream processing algorithms use adaptation to changing distribution rather than applying a forgetting factor to an old data or ignoring it. Thus, we have to be able to detect change occurrences in the stream as soon as possible so as to quickly adapt the algorithm to a different distribution.

As change-point detection problem can be applied in large scale stream processing systems where data arrives with high frequency, a scalable solution for this problem is needed to be presented. Possible way of organising distibuted solution for change-point detection is to present two algorithms: an algorithm for processing element on a single node which collects the statistics and an algorithm for fusing statistics from nodes. In this work, the existing efficient algorithm was taken as a basis for a single-node algorithm and different simple rules were used for fusing statistics. 


\section {Motivation}
Most change-point detection algorithms are not intended for usage in distributed mode. Experimentally obtained results show that algorithm used in single-node mode can really slow the stream processing graph.

\begin{figure}[h]
    \centering
    \includegraphics[scale=0.7]{throughput.png}
\end{figure}

``Decentralized'' line depicts throughput of system with distributed algorithm of change-detection while ``Centralized'' corresponds to throughput of system with single-node algorithm.

Thus, development an efficient distributed change-point detection algorithm is a practically important problem.

\section {Algorithm}
\subsection{Single-node algorithm}

Single-node algorithm should process elements from stream and return a statistic that characterise a state of the part of the stream passed through this node.

We use the window technique from the paper \cite{kifer2004detecting} which involves two windows: reference that stands at the beginning of the stream and sliding that moves as soon as new elements arrive. We calculate maximum value of distance between reference and sliding window for all positions of sliding window. More precisely we calculate the following statistic and return it as a result fo this node $$S_t = \max_{1 \le i \le t - (n + m) + 1} KS(reference, sliding_i)$$, where $$KS(X, Y) = \max_x \left|F_X(x) - F_Y(x)\right|$$ is Kolmogorov-Smirnov statistic for emperical distributions $F_X$ and $F_Y$ of samples $X$ and $Y$.

\subsection{Fusion rule}

Fusion rule should recieve statistics from different nodes and decide wheter change has already happened or not.

We apply a simple multivariable function such as maximum, median or mean. We compare the obtained value to the precomputed threshold and if it's greater than we think that change has occured.



% \section {Evaluation}
% \label {fs-discussion}

In this section, we discuss and demonstrate the main pitfalls which arise with the na\"ive data flow. The purpose of our evaluation is not to compare various machine learning models, but to investigate the applicability of stream processing systems as a {\em tool} for building text classification pipelines. We concentrate on a question on how distributed stream processing features may affect reproducibility and reliability of classification results.

For experiments, we used an implementation of the proposed data flow on top of Apache Flink. The experiments were performed on a single 4 core CPU 8GB RAM machine with 2 Flink workers. Such setting is chosen to show that issues can be faced even in a deployment with a limited asynchrony. As a dataset, we used an open corpus of news articles from Russian media resource lenta.ru~\cite{lentaru}. This dataset contains documents, which are labeled by one of 90 different topics such as {\em sport}, {\em politics}, {\em science}, etc. In the experiments, we generated a stream consisted of articles from the dataset.

We used multinomial logistic regression model as a classifier. We considered the multi-label classification problem: each news article can be labeled by multiple topics. For instance, text about novel research in sports food may be denoted as 70\% about sport, 20\% about science and 10\% about food. 

\subsection{Reproducibility}

Users of popular open-source machine learning libraries like sklearn used to obtain results which are unbiased by an execution environment. Migration to batch processing systems like Hadoop or Spark usually do not cause many extra issues, because these engines mostly hide effects of asynchronous processing from a user and provide deterministic results. On the contrary, most distributed stream processing engines are non-deterministic due to processing model aiming to low latency. Therefore, the main challenge regarding reproducibility of streaming machine learning pipelines is to achieve predictable results, while keeping low processing latency. 

\subsubsection{Online training}
An issue regarding the pipeline is that the training process may be time-consuming. If training and prediction processes run consecutively, there will be significant latency spikes, e.g. if a training process lasts for several minutes, then spikes may be 10 000 times greater than the latency for prediction. However, without synchronization, there will be no reproducible correspondence between texts and applied model. It is almost impossible to achieve the same results within a new run on the same data because the training time becomes a hidden parameter that influences output. For instance, assume that we make two runs. On the first run model update consumes 70 seconds, but on the second run 75 seconds due to extra CPU load. If training and predicting are not synchronized, more unlabeled input elements are processed by an outdated model in the second case, so the distribution of news topics may be different between these two runs. We propose two solutions for the issues in question:

\begin{itemize}
    \item Use online learning algorithms. In this case, model updating is smooth and its synchronization with training does not cause latency spikes. We discuss this approach in details in the next section.
    \item Consider model parameters as special input elements that are stored with other input elements in a persistent queue, e.g. using Kafka~\cite{kreps2011kafka}. To reproduce results, there is just a need to replay elements from this queue.
\end{itemize}

\subsubsection{Races in the data flow}
The issue is that there is a race between documents in the data flow before IDF update. Hence, IDF features of the words in articles may vary from run to run. For example, let us consider two documents stream: the first one contains word {\em cat}, while the second consists of {\em cat} and {\em dog}. If the first document is processed before the second, IDF for the word {\em cat} within TF-IDF features of the second document will be 1, while otherwise, it will be 0. This issue is more sophisticated than the previous one, but can also make results irreproducible. 

To show how this behavior affects text classification results we made 10 runs on a stream consisted of 10 000 news articles. We compared the most probable 1,2,3,4,5 obtained labels for the same documents between runs. Our comparison was order-sensitive: if top 2 labels for the document on the first run is [sport (50\%), science (20\%)], but on the second run [science (50\%), sport (20\%)], then we denoted these results as varied. 

Table~\ref{race_table} demonstrates the results of the experiment. As we can see, approximately 56 out of 10 000 articles obtained distinct top 1 label. With the growth of considering the top, the percent of varied results significantly increases: 1270 articles achieved different top 5 labels on the average. These results indicate that the issue may influence the classification results and makes them hardly reproducible between independent runs. The solution to this problem is to determine the order of input documents and preserve this order before IDF computation. An implementation of this fix is specific for a concrete streaming engine.

\begin{table}[htbp]
\caption{Effects of races in the data flow}
\begin{threeparttable}
\begin{tabular}{lcl}
Top labels for comparison    & \% of varied results & std    \\
\hline
1   &   0.56    &   0.06    \\
2   &   2.38    &   0.14    \\
3   &   5.27    &   0.22    \\
4   &   9.27    &   0.35    \\
5   &   13.7    &   0.53    \\
\end{tabular}
\end{threeparttable}
\label{race_table}
\end{table}

\subsection{Fault tolerance}

As we demonstrated above, if machine learning pipeline is run on multiple computational units, it brings issues with reproducibility. Unfortunately, it is not the only challenge: computational nodes and the network may fail and potentially cause shifted or even incorrect results. Hence, in large-scale production deployments it is important to ensure that nodes and network failures do not {\em invisibly} influence outcome.






% \section {Screaming challenges}
% \label {fs-solution}

\subsection{Reproducibility and Fault Tolerance}

As it was demonstrated in the previous section, {\em exactly once} and determinism are desirable properties for the text classification data flow. On the other hand, one of the key performance metrics in streaming applications is latency, so there is a need to achieve as small latency as possible. Table~\ref{comparison} shows if a system supports exactly once, built-in determinism, and low latency (less than 500 ms). To the best of our knowledge, among open systems only~\FlameStream\ provides for both low latency, determinism, and exactly once. This property is achieved using optimistic order enforcement that implies system-wide idempotence. The details of this approach are discussed in~\cite{we2018adbis, we2018beyondmr, we2018seim}. Therefore, implementation of the text classification data flow on top of~\FlameStream\ can potentially resolve the trade-offs between reliability and performance.

\begin{table}[htbp]
\caption{Support of exactly once, built-in determinism, and low latency (less than 500ms) by stream processing systems}
\begin{threeparttable}
\begin{tabular}{lccc}
System & Exactly-once & Determinism & Latency    \\
\hline
Storm  &    --      &   --       &   low            \\
Heron  &    --      &   --       &   low            \\
Samza  &    --      &   --       &   low            \\
Spark Streaming    &    +       &   +        &   high           \\
Flink              &    +       &   -        &   high$^*$       \\
MillWheel          &    +       &   +        &   NA             \\
FlameStream        &    +       &   +        &   low            \\
\end{tabular}
* with enabled {\em exactly once}~\cite{we2018beyondmr}
\end{threeparttable}
\label{comparison}
\end{table}

\subsection{Concept Drift}

Streaming applications often experience {\em concept drift}: the statistical properties of the target, which the machine learning model is trying to predict, change over time. Regarding news articles classification, concept drift may lead to the following effects:

\begin{itemize}
    \item Meaning of some terms is changing over time. For example, word {\em goal} in the article published during the football world cup is most likely related to football. On the other hand, during the world hockey cup, it rather belongs to the hockey topic.
    \item News topic may appear and vanish over time. For example, some important events can transform into a separate news topic for a time as it happens with large political or sports forums. However, after some time such topics can disappear form a news agenda.
\end{itemize}

A streaming classifier must handle such behavior in order to automatically fit in rapidly changing news data. We propose a modification of the data flow for prediction with online training that aims to handle concept drift. Training is a separate branch within the logical graph presented in section~\ref{fs-framework}. The modified data flow is shown in Figure~\ref{training_graph}. Assume that the input stream consists of two types of elements: pre-labeled and raw. The latter elements must be labeled by a classifier and delivered to end-user. For already labeled text its features are sent to a {\em Partial fit} vertex instead of the {\em Classifier}. {\em Partial fit} vertex updates machine learning model and sends it to {\em Text Classifier} vertice.

\begin{figure}[htbp]
  \centering
  \includegraphics[scale=0.44]{pics/logical-graph}
  \caption{Data flow with online training}
  \label {training_graph}
\end{figure}

A potential issue regarding the pipeline is that the training process may be time-consuming. If training and prediction processes run consecutively, there will be significant latency spikes, e.g. if a training process lasts for several minutes, then spikes may be thousands times greater than the latency for prediction. However, without synchronization, there will be no reproducible correspondence between texts and applied model. It is almost impossible to achieve the same results within a new run on the same data because the training time becomes a hidden parameter that influences output. 

For instance, assume that we make two runs. On the first run model update takes 70 seconds, but on the second run 75 seconds due to extra CPU load. If training and predicting are not synchronized, more unlabeled input elements are processed by an outdated model in the second case so the distribution of news topics may be different between these two runs. In order to solve this issue, we propose using efficient online learning algorithms, e.g. FTRL proximal~\cite{mcmahan2013ad}. In this case, model updating is smooth and its synchronization with training does not cause latency spikes.

% \section{Related Work}
% \label{fs-related}

While the classification of text streams is a well-studied problem~\cite{zhang2008one, tampakas2005}, it is challenging to solve this task at scale. There are plenty of projects that apply batch or micro-batch processing systems to distributed text classification~\cite{semberecki2016distributed, 8029336, baltas2016apache, svyatkovskiy2016large}. However, as it was mentioned above, these methods are not suitable for streaming classification due to low latency requirements. 

General problems regarding machine learning at scale are formulated by the TFX project team~\cite{Baylor:2017:TTP:3097983.3098021}. Among them are continuous training, reliability, reproducibility, etc. It is shown that these problems are hard even if an environment is reliable and fault tolerant. In this work, we argue that these goals are even harder to achieve using state-of-the-art distributed stream processing systems.

Despite the fact that machine learning on top of distributed stream processing is a hot topic~\cite{qiu2016survey}, previous works in this field do not consider issues related to the delivery guarantees and distributed environment~\cite{khumoyun2016real}. A SAMOA framework~\cite{morales2015samoa} provides implementations of several popular algorithms which can be executed on Flink, Storm or Samza but does not take responsibility for reproducible results and correctness in case of failures. Therefore, SAMOA delegates the work on enforcing reproducibility and fault tolerance on a developer. Spark MLlib~\cite{meng2016mllib} is another popular library for adaptation of machine learning pipelines to scalable batching or micro-batching data flows. While Spark Streaming can produce reproducible and reliable results, it is not able to provide latency less than a second~\cite{karimov2018benchmarking, S7530084}.

The problem of non-deterministic and non-reproducible data flows within distributed stream processing is also have been studied in recent years.  As we demonstrated above, the key problem regarding determinism enforcement is races in a data flow. Popular methods to handle such races is {\em in-order} and {\em out-of-order} processing approaches~\cite{Li:2008:OPN:1453856.1453890}. Both these techniques require buffering before each order-sensitive operations until there is a guarantee that all elements are properly ordered. A mechanism that allows a user to control the trade-off between determinism and latency is proposed in~\cite{Doulkeridis:2014:SLA:2628707.2628782}. However, this technique provides for low latency only with a {\em some level} of determinism. An optimistic approach to handle out-of-order elements approach is introduced in~\cite{we2018seim}. Basically, this method mitigates a need for buffering before each operation and reduces the latency of the whole data flow. An adaptation of this approach for machine learning pipelines relates to our future work.

% Techniques for achieving exactly once the delivery guarantee is discussed in plenty of works~\cite{Carbone:2017:SMA:3137765.3137777, Akidau:2013:MFS:2536222.2536229, Zaharia:2012:DSE:2342763.2342773}. However, most of them have a high performance overhead. A promising technique to obtain exactly once was proposed in~\cite{we2018beyondmr}. The key idea behind it is .

\section{Conclusion and Future Work}
\label {fs-lightbulbs-conclusion}

In this work, we extended a popular algorithm for change detection in streaming data~\cite{kifer2004detecting} to make it suitable for distributed stream processing systems. The quality of the resulting algorithm was experementally confirmed.

The experimentally obtained quality is comparable with the state-of-the-art and slightly decreases with the growing number of computational nodes while detection latency increases sublinearly. It is explained by the fact that the multinode algorithm needs to process more elements to ``warm-up'' before it starts to detect changes. Experiments within distributed stream processing system demonstrated linear increase of throughput, which means that the algorithm is pretty scalable.

We demonstrated that our extension of single-node algorithm makes dataflows with change detection scalable with non-essential cost. Thus, it potentially increases the applicability of change detection to large streaming deployments.

Regarding our future work, we plan to investigate more data partitioning methods and how they affect the accuracy of our change detection algorithms. We also plan to modify the algorithm to make it more stable in the case when streaming elements are partitioned depending on their values.

\bibliographystyle{ACM-Reference-Format}
\bibliography{bibliography/flame-stream}

\end {document}

\endinput
