\documentclass{vldb}

\usepackage[utf8]{inputenc}

\usepackage{graphicx}
\usepackage{url}
\usepackage{multirow}
\usepackage{array}
\usepackage{hyperref}
\usepackage{algorithm} % for algorithms
% \usepackage{algorithmicx}
% \usepackage{algorithm2e} % for algorithms
\usepackage{algpseudocode}
% \usepackage{booktabs} % For formal tables
\algdef{SE}[SUBALG]{Indent}{EndIndent}{}{\algorithmicend\ }%
\algtext*{Indent}
\algtext*{EndIndent}

\usepackage{balance}  % for  \balance command ON LAST PAGE  (only there!)
\usepackage{caption}
\usepackage{subcaption}
\usepackage{tikz}
\usepackage[flushleft]{threeparttable}
\usetikzlibrary{backgrounds, calc, positioning, fit, decorations.pathreplacing}
% \theoremstyle{remark}

\pagestyle{empty} % removes running headers

\newcommand{\PicScale}{0.5}
\newcommand {\FlameStream} {FlameStream}
\newcommand {\tracker} {trAcker}
\newcommand {\acker} {Acker}

\newtheorem{lemma}{Lemma}

% Include information below and uncomment for camera ready
\vldbTitle{Substreams Management in Distributed Streaming Dataflows}
\vldbAuthors{Nikita Sokolov, Artem Trofimov, Igor Kuralenok, Nikita Marshalkin, and Boris Novikov}
\vldbDOI{https://doi.org/10.14778/xxxxxxx.xxxxxxx}
\vldbVolume{12}
\vldbNumber{xxx}
\vldbYear{2019}

\begin{document}

\title {Substreams Management in Distributed Streaming Dataflows}

% \numberofauthors{5}

% \author{
% \alignauthor
% Nikita Sokolov\\
%     \affaddr{ITMO University}\\
%     \affaddr{Saint Petersburg, Russia}\\
%     \email{faucct@gmail.com}
% \alignauthor
% Artem Trofimov\\
%     \affaddr{Yandex}\\
%     \affaddr{Saint Petersburg, Russia}\\
%     \email{tomato@yandex-team.ru}
% \alignauthor
% Igor Kuralenok\\
%     \affaddr{Yandex}\\
%     \affaddr{Saint Petersburg, Russia}\\
%     \email{solar@yandex-team.ru}
% \and 
% \alignauthor
% Nikita Marshalkin\\
%     \affaddr{VK}\\
%     \affaddr{Saint Petersburg, Russia}\\
%     \email{n.marshalkin@corp.vk.com}
% \alignauthor
% Boris Novikov\\
%     \affaddr{National Research University Higher School of Economics}\\
%     \affaddr{Saint Petersburg, Russia}\\
%     \email{borisnov@acm.org}
% }

\maketitle

\begin{abstract}
%The majority of state-of-the-art stream processing systems faces a problem of obtaining notifications when the specific set of input elements have already been applied to all operators in a dataflow. Such notifications are commonly used to take consistent state snapshots or to release all descendants of an input element atomically while preserving exactly-once delivery guarantee. To generate these notifications, there is a need to track dependencies between input and their descendants and to inform when all descendants have been completely processed. In this work, we propose a method for tracking dependencies between streaming elements that can be applied for both cyclic (iterative) and acyclic dataflows. We demonstrate that our technique has low latency and throughput overhead within various setups and can significantly outperform methods employed by state-of-the-art stream processing systems.

% In distributed stream processing, it is hard to determine a correspondence between input and output elements due to complex transformations, filterings, and so on. On the other hand, mechanisms for taking consistent state snapshots and for transactional stream processing require understanding when a system has already processed a specific set of input elements and the elements induced by them. To achieve this understanding, one needs to track dependencies between items in a dataflow. In this work, we propose a method for monitoring dependencies between streaming elements that is applicable to cyclic dataflows, as well. Our technique is based on an agent that watches for the computational progress and broadcasts system notifications when certain elements processing is finished. These notifications indicate when a system completely processed all descendants of concrete input elements. We demonstrate that our technique provides low notification latency while having low throughput overhead on regular processing, and can outperform methods employed by state-of-the-art stream processing systems.

% In distributed stream processing, a system often needs to determine a correspondence between input and output elements, e.g. to take state snapshot affected by  or to release all descendants of an input element atomically while preserving exactly-once delivery guarantee. This task is hard due to complex transformations, filterings, producing multiple elements from a single, etc. 

% 1. На практике в основном работают с бесконечным потоком как со смесью конечных потоков. Примеры задач
% 2. Наибелее популярный сейчас способ - пунктуации, специальные элементы, которые инъектируются в поток и несут предикат. Они используют свойство упорядоченности элементов в рамках одного процесса. Проблемы пунктуаций - циклы, большой дополнительный трафик из-за броадкаста, неподходит для быстроизменяющихся свойств.
% 2. В этой статье мы предлагаем способ разделять элементы с помощью меток, которые приклеиваются к элементам потока. Потоковые операторы отправляют метки элементов, а также предикаты, которым эти элементы удовлетворяют специальному агенту. Агент аггрегирует все сообщения и бродкастит нотификации о том, что в потоке больше нет элементов, удовлетворяющих тому или иному предикату. В зависимости от того, как объявить порядок на метках, можно получить различные свойства, в т.ч. аналогичные пунктуациям и подходящие для снепшота состояния. 

% In a plenty of real-world streaming applications, potentially infinite input stream is considered as a mixture of finite substreams with distinct properties. For example, stateful operators can maintain unbounded

Stream Processing Engines (SPEs) handle a potentially infinite sequence of data elements. This sequence often can be viewed as a mixture of finite substreams with distinct properties: event times ranges, payload values, etc. Some operators ought to know since then there will be no longer elements with specified properties, i.e. when the substream ends. For example, stateful streaming operators should clear outdated state. Otherwise, they may run out of memory. Time window operators should release output after all elements within the specified time range arrived. Most state-of-the-art SPEs use {\em punctuations} to manage substreams. Punctuations approach is powerful, but has limitations: it does not support cyclic dataflows and is inefficient for small substreams due to high network traffic overhead.

In this work, we present a new substream management design that supports cyclic dataflows and tiny substreams (even substreams containing a single element). Within our method, all streaming elements are marked with ordered {\em labels}. Operators send to a distributed agent called {\em tracker} the properties of sent and received items with corresponding labels. Tracker aggregates data from all operators and propagates notifications, which indicate the end of a substream since the specified label. We demonstrate that our implementation provides the same semantics of notifications as punctuations, but implies lower network traffic overhead. Experiments show that our technique efficiently handles real-world cyclic dataflows and outperforms punctuations for small substreams.

% 

\end{abstract}

% \keywords{Data streams, exactly-once, drifting state, optimistic OOP}

\thispagestyle{empty}

\section {Introduction}
% \label {fs-acker-intro}

The processing of a data stream without insights into the properties of its data elements can be challenging. For example, it may be unclear when a system can prune outdated keyed state~\cite{Tucker:2003:EPS:776752.776780}, release windowed aggregations~\cite{Begoli:2019:OSR:3299869.3314040}, or create a state snapshot for an epoch~\cite{Carbone:2017:SMA:3137765.3137777}.

Each of these scenarios is a particular case of a problem of monitoring substreams emergence and termination that we call a {\em substream management problem}. A substream is a part of the stream such that all its elements satisfy some predicate. 
For example, in the case of state pruning, the predicate is {\em [a data element key equals to $K$]}, for time window aggregations, the predicate is {\em [a data element has a timestamp less than $T$]}, and for state snapshotting it is {\em [a data element belongs to the epoch $E$]}.

In this paper, we focus only on two signals: substream start and its termination. Tracking a start of a substream is a straightforward task: the first event of a substream will naturally trigger its start. On the contrary, generating a substream termination event is a challenging task, and various properties may be required by practical problems:
\begin{itemize}
    \item Deterministic windowed join\footnote{given the same sequences of input tuples, the same output tuples will be produced} requires an order of termination signals to respect the order of input elements (termination events from data producers)~\cite{najdataei2019stretch, gulisano2016scalejoin}.
    \item An epoch is a substream that an SPE should process atomically. A termination event for an epoch should arrive before any elements of the next epoch~\cite{2015arXiv150608603C}.
    \item State pruning problem does not require any specific properties from termination events. However, late termination event receiving may cause sub-optimal memory utilization.
\end{itemize}

A popular substream management method is the punctuations framework~\cite{tucker2003exploiting}. The main idea behind this framework is to divide the stream by injecting special elements called {\em punctuations} that define substreams ``borders''. These special elements are propagated via the same network channels as data elements. While the punctuation approach is robust and easy to implement, it has several limitations. 

Punctuations are not applicable for cyclic dataflows in a general case because elements belonging to a substream can remain in transit within a cycle for an uncertain time~\cite{carbone2018scalable}. The technique proposed in~\cite{Carbone:2017:SMA:3137765.3137777} mitigates this issue for the state snapshotting problem. The main idea of this technique is to include in a snapshot all in-transit elements (possibly from previous epochs) within a cycle and then resend them on rollback. However, it provides a solution for a specific problem that does not allow a system to determine a substream termination for cyclic dataflows using punctuations.

The high network overhead forms another limitation. Network traffic complexity for this method is $O(K|\Pi|^2)$, where $|\Pi|$ is the number of processes and $K$ is the number of substreams because each process should propagate punctuations to all output channels. This complexity boundary covers the worst case when all processes are interconnected. However, SPEs prefer to distribute the work among nodes evenly to ensure scalability~\cite{carbone2015apache, Kulkarni:2015:THS:2723372.2742788, Akidau:2013:MFS:2536222.2536229}. This load balancing implies that each process effectively occupies channels to all other processes. The worst-case complexity boundary is tight for scenarios when an execution graph contains at least one operator that repartitions data.

Substreams can be {\em fine-grained}: for example, each user session defines a substream. If there are a lot of small substreams, an inefficient substream management system can degrade the latency~\cite{DBLP:journals/pvldb/BegoliACHKKMS21} and the throughput of an SPE~\cite{Li:2008:OPN:1453856.1453890} or affect the performance of state checkpointing~\cite{zhang2021research}.

\begin{figure}[t]
  \centering
  \includegraphics[width=0.20\textwidth]{pics/tracker-scheme.pdf}
  \caption{\tracker\ framework: tracking agent aggregates information about substreams and produces NEOSS}
  \label{tracker_scheme}
\end{figure}

In this work we formalize the substream management problem and show that the network traffic overhead of the punctuations framework is far from the optimal. We also formally define properties of a substream management technique required by various problems such as state snapshotting to ensure that a newly proposed method satisfy them. 

We introduce a new substream management framework called \tracker. Figure~\ref{tracker_scheme} shows the high-level scheme of our method. 
Within this framework, we use a dedicated agent that receives information about substreams from the entire SPE and sends back {\em end-of-substream notifications} (NEOSS). 
NEOSS messages are propagated through this agent without broadcasting between processes, reducing the amount of extra traffic. Such propagation method is suitable for cyclic dataflows because there is no need to forward service traffic through the cycles.

Basic comparison between the \tracker\ framework and its alternatives is shown in Table~\ref{solutions-overview-table}. Regarding network traffic, $|\Pi|$ is the number of computational nodes and $K$ is the number of substreams. We can outline that the punctuations framework is the only substream management mechanism that supports arbitrary predicates for substreams, so we use it as a baseline approach in the experiments. The commonalities and differences between the \tracker\ framework and alternative solutions are detailed in Section~\ref{fs-acker-related}.

\begin{table}[t]
    \caption{An overview of substream management techniques}
    \label{solutions-overview-table}
    \begin{threeparttable}
        \centering
        \begin{tabular}{|>{\bfseries}c|c|c|c|c|c|} 
          \hline
          Method & Arbitrary predicates & Cycles & Traffic  \\ \hline \hline
          Punctuations & + & - & $O(K|\Pi|^2)$ \\ \hline
          MillWheel* & - & N/A & N/A \\ \hline
          Naiad* & - & + & $O(K|\Pi|^2)$ \\ \hline
          Acker & - & + & $O(K|\Pi|)$ \\ \hline
          \tracker\ & + & + & $O(K|\Pi|)$ \\ \hline
        \end{tabular}
        *progress tracker
    \end{threeparttable}
\end{table}

In summary, our contributions are as follows:
\begin{enumerate}
    \item We provide a formal model of substream management. This model allows us to compare the properties of various substream management systems.
    \item We present a novel substream management technique that achieves a lower bound of network traffic overhead.
    \item We demonstrate \tracker\ performance in comparison to a state-of-the-art approach on diverse workloads.
\end{enumerate}

The rest of the paper is organized as follows: Section~\ref{fs-acker-preliminaries} formalizes the substream management problem and indicates its main properties. In Section~\ref{fs-acker-tracker}, we introduce a general design of the \tracker\ framework and demonstrate the properties of this substream management solution. Section~\ref{fs-acker-impl} summarizes the implementation of \tracker\. In Section~\ref{fs-experiments}, we show that the proposed technique is scalable and can outperform alternatives employed in state-of-the-art stream processing engines. The relevant prior research is outlined in Section~\ref{fs-acker-related}. Finally, we discuss our conclusions in Section~\ref{fs-acker-conclusion}.

% \section{Background and Motivation}
% \label {fs-acker-motivation}

Distributed stream processing systems are shared-nothing runtimes which continuously ingest input elements, transform them according to pre-defined operators, and deliver output elements. We assume that an operator processes items one-by-one so that it can handle only one record at a time. Operators can be stateless and stateful. An output element of an operator may depend on the current state as well as on the corresponding input element. Operators are partitioned among workers and independently process their chunks of data items. Streaming operators are often defined just as a user-provided code and can be quite complicated: they can produce multiple output elements from a single input, filter out some items, or do transformations according to their current state. In general, it is hard to figure out which input element originated an output one. For instance, if an operator receives texts and splits them into the words, it is not possible, having only a word, to determine the document this word came from.

{\em Dependency tracking} is a process of matching streaming elements with its original input items and vice versa. Typically, this mechanism can notify the system when all descendants of some input elements are entirely or partially processed. For instance, input elements can be stored in persistent queues to be reprocessed in case of system failures. Items cannot be stored in these queues forever due to memory and disk limits. A dependency tracking technique can trigger garbage collection within these queues. As an example, notification, when the system has processed all words from particular documents, can cause removing these documents from the persistent storage. 

A good dependency tracking method should provide low notification latency as well as do not induce an overhead on regular processing. Another valuable property is a {\em granularity} of tracking: some methods can provide notifications for the specific input element and small parts of the pipeline, while others can only indicate when a set of input records went through the whole dataflow.

Further, in this section, we present several practical problems that require a dependency tracking mechanism. We demonstrate that this mechanism plays a crucial role in obtaining the correct and consistent processing results. After that, we review tracking techniques adopted in state-of-the-art stream processing systems and identify their limitations. 

\subsection{Completeness monitoring}
As we mentioned above, streaming systems often need to monitor the completeness of processing. One application is to purge input queues when some input elements are entirely processed. Another important problem that requires completeness notifications is to alert the user if some elements are lost.

A particular use-case of completeness monitoring is transactional processing. If input elements are split into multiple ones and system processes them independently, a user often assumes that the descendants will be handled atomically. In other words, if a system loses a single element, other items that depend on the same input will not affect a system state. As an illustration, let us consider a streaming pipeline that transfers money between accounts. Assume that input elements are bank transfer actions. An operator that updates an account balance is partitioned by the account identifier. In this case, one can split an input element into items that update source and destination accounts. Note that these items can be processed independently and possibly on different workers. If a system loses only one of these records, e.g., due to a network failure, the system state becomes inconsistent.

To process input items transactionally, a system must check that all descendants of an input item are not lost and rollback changes caused by survived elements otherwise. One can solve this problem with dependency tracking. If there is no notification that all descendants of an input item are completely processed for quite a long time, a system can initiate a recovery mechanism.

\subsection{State snapshotting}
Many streaming systems, including Flink~\cite{Carbone:2017:SMA:3137765.3137777} and Heron~\cite{Kulkarni:2015:THS:2723372.2742788}, apply state snapshotting mechanisms to ensure fault tolerance. The main idea behind this method is to periodically save the global system state (states of all operators) to persistent storage. In case of a failure, the system rollbacks a state from storage and reprocesses missed input elements. The main problem here is to determine which input elements should be reprocessed after the failure. If an input element has already influenced the system state, reprocessing of this element will cause state inconsistencies. The tricky point is that the input element can affect the system state {\em partially}. Returning to the example with bank transfer, the element that updates source account can be processed before failure, while the item that should deposit money to the destination account can be lost. If the system reprocesses the input item, it withdraws money from the source account twice.

To prevent the mentioned inconsistency, a system can snapshot state at time moments, when some specific set of input records have entirely affected it, including all descendant records~\cite{2015arXiv150608603C, thepaper}. Streaming systems use dependency tracking techniques to determine these moments. For instance, the tracking mechanism can notify each operator when it is safe to save its local state. The system commits global snapshot when all operators successfully saved their states affected by certain input elements only. Therefore, in case of a failure, a streaming engine can safely reprocess input records that did not influence the snapshot.

\subsection{Existing solutions}
We can consider the micro-batching model applied in Spark Streaming~\cite{Zaharia:2012:DSE:2342763.2342773} and Storm Trident~\cite{apache:storm:trident} as a dependency tracking mechanism. Within the micro-batching model, system groups all input elements into small sets called batches. System consecutively and atomically processes micro-batches and monitors the completeness of the computations. In case of failure, a current micro-batch can be entirely reprocessed, because it is guaranteed that an element in a batch may depend only on another record in the same batch. Therefore, the system can snapshot the state after each successfully handled micro-batch. Very small micro-batches are ineffective~\cite{Zaharia:2012:DSE:2342763.2342773}, so the main limitation of this technique is the low granularity of tracking.

Another popular approach bases on punctuations~\cite{Tucker:2003:EPS:776752.776780}. The main idea is to inject special input elements called {\em checkpoints} into a system. These items flow through the same network channels and separate groups of ordinary input records. Therefore, a streaming element can depend only on the input item that arrived between the surrounding checkpoints. If an operator receives a checkpoint, it is guaranteed that this operator has already received all descendants of certain input records. Hence, this event can initiate state snapshotting. While this approach is quite similar to the micro-batching technique, it provides more fine-grained tracking because of independent notifications for different parts of a pipeline. However, checkpoint-based methods do not support cyclic dataflows as well as can affect the throughput of a system in case of tracking for the individual input elements.

\acker\ is a mechanism for completeness tracking employed in Apache Storm~\cite{apache:storm}. The main idea is to enrich all data records with a random number. Each time an item is sent or received, its random number is XORed into the checksum. If all elements have been successfully processed, the checksum will be zero. In this paper, we build our dependency tracking mechanism based on the same idea. However, we substantially extend it to support the high granularity of processing, be scalable, and more efficient in terms of network traffic. In the next section, we discuss details about \acker\ and our method called \tracker\ .

% \section{Dependency tracking design}
% \label {fs-acker-design}

Formalize core concepts.

\subsection{Global time concept}
What is global time and how we use it.

\subsection{Tracking formalisms}
There are two cases: elements with increased global time can be generated in graph or cannot.

\subsection{\tracker\ 's mechanism}
XORS, grouping by Global Times, and monotonicity. 

\subsection{Examples}

\subsubsection{Completeness monitoring}

\subsubsection{State snapshotting}





% \section{Implementation}
% \label {fs-acker-impl}

\subsection{Centralized \tracker\ }

\subsection{Decentralized \tracker\ }


% \section {Experiments}
% \label {fs-acker-experiments}

We have tested the \tracker\ performance on a simple directed path graph of various length, which was shuffling processed elements between machines in each vertex. The \tracker\ of various configurations was compared with  tracking using watermarks and no tracking at all. Elements were tracked in windows of 1, 10 and 100.
Virtual machines used in experiments had single CPUs and 4 GB of RAM per machine. One machine called a Bench Stand was used to input data into the dataflow at a fixed rate while measuring the speed of it being processed via receiving output and notifications for data being processed. \tracker\ was running on machines excluded from the dataflow: a single one for centralized configuration and two for distributed configuration.

\subsection{Network traffic}

Network traffic was measured in number of separate service messages sent over the network. Local Acker was sending messages in batches.

% https://gist.github.com/faucct/032aaf6240db361d30a184b1d7bf3c8e

\subsection{Notification latency}

Notification latency was measured as a time between moments of Bench Stand receiving last elements in tracking windows and notifications for that window.

% https://gist.github.com/faucct/032aaf6240db361d30a184b1d7bf3c8e

\subsection{Scalability}

In those experiments we are reproducing a case in which the centralized \tracker\ was not holding the load, while the distributed \tracker\ was working. While using 100 machines running our dataflow we have failed to reproduce it. Still, we have been able to simulate it by increasing the number of Ack messages sent in a single batch 9 times.

% Надо подумать, какие графики тут нужны и подобрать, какую конфигурацию мы симулируем этим способом.

\subsection{Overhead on throughput}

In those experiments we show that tracking with \tracker\ in contrary to watermarks does not have a large overhead on throughput.

% Я забыл доснять эти эксперименты.

\subsection{State snapshotting}

In those experiments we are comparing granular tracking using centralized \tracker\ and watermarks. Processed elements are divided into snapshot windows. Pipeline vertices only process elements from a current snapshot window and buffer ones from a next snapshot window until they receive a notification that all elements from a current snapshot have been processed. When this happens vertices imitate snapshotting with a fixed duration sleep and continue to process elements from next snapshot window. We have measured a number of buffered elements and total time they have spent in buffer varying the snapshot duration.

% https://gist.github.com/faucct/6097d9d08197cb979b71721b16f8b6a3/

\subsection{Count iterations?}



% \section{Related Work}
% \label {fs-acker-related}

Most dependency tracking techniques employed in state-of-the-art stream processing systems are discussed in detail in Section~\ref{existing_solutions}. In this section we hightlight the differences between 

% Naiad~\cite{Murray:2013:NTD:2517349.2522738} uses a kind of similar to \tracker\ mechanism for tracking the progress of iterative computations. Within this method, each data item in a system is assigned with an {\em epoch} and a vector of logical timestamps called {\em loop counter}. Epoch is similar to our notion of {\em global time} concept but provided by an external user. The value on the $i$th position of the loop counter indicates the number of times this element went through the $i$th {\em loop context} (cycle) in a dataflow. Special distributed agents monitor for the items and their timestamps and notify when all elements reach some iteration number or all elements from an epoch are entirely processed. There are three main differences between the mentioned technique and the \tracker. Firstly, \tracker\ relies on the global identifier of an element provided by a system itself. Secondly, the protocol used in Naiad causes the enormous number of extra network messages that quadratically depend on the number of machines even with optimizations~\cite{Murray:2013:NTD:2517349.2522738}. Thirdly, Naiad's method uses counter updates (+1/-1) instead of XORs, hence update messages do not commutate. This fact complicates the implementation (especially distributed) of this method.

% The problems of transactional processing, providing for delivery guarantees, and fault tolerance, are extensively studied in recent years~\cite{Akidau:2013:MFS:2536222.2536229, Carbone:2017:SMA:3137765.3137777, thepaper, Wang:2019:LSF:3341301.3359653}. While state-of-the-art stream processing systems still provide high overhead on regular processing due to fault tolerance protocols, transactional processing, etc., we expect that this area will be studied further. As we mentioned above, a dependency tracking mechanism is an essential part of the solutions to these problems. Hence, \tracker\ can be applied to optimize the existing techniques.

\section {Conclusion}
% \label {fs-acker-conclusion}

In this work, we formulated and formalized a problem of dependency tracking between input and output elements in streaming dataflows. We demonstrated that state-of-the-art distributed stream processing systems face this problem in state snapshotting mechanisms~\cite{Carbone:2017:SMA:3137765.3137777, apache:storm}, the materialization of time-varying relations~\cite{Begoli:2019:OSR:3299869.3314040}, and atomic delivery of all descendants of an input item~\cite{we2018adbis}.  

To solve this problem, we proposed a mechanism that adopts ideas from the Apache Storm completion tracking mechanism called \acker. We extend each data item with a logical timestamp that denotes corresponding input items and track if data flow contains elements with specific timestamps. Our solution, called \tracker, provides the following features:
\begin{itemize}
    \item {\bf Fine-grained tracking:} \tracker\ efficiently watches and provides notifications that system completely processed some set of input items even for individual input elements.
    \item {\bf Cyclic graphs support:} proposed mechanism works for cyclic execution graphs, and that makes it suitable for iterative dataflows as well. 
    \item {\bf Scalability:} we introduced a decentralized version of \tracker\ that allows a system to distribute extra network traffic between all computational units. 
    \item {\bf Low overhead:} \tracker\ does not produce any significant performance penalty and does not affect the throughput of a distributed streaming dataflow.
\end{itemize}

We conducted a series of experiments and compared the proposed method with a baseline approach based on the checkpointing mechanism used in Apache Flink. We demonstrated that both centralized and decentralized implementations of \tracker\ provide lower notification latency that does not considerably degrade with an increase of a logical graph size or a cluster size. Experiments also showed that \tracker\ has lower throughput overhead in case of fine-grained tracking.

\bibliographystyle{abbrvurl}
% \bibliography{bibliography/flame-stream}
\bibliography{bibliography/flame-stream}

\end {document}

\endinput
