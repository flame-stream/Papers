\documentclass{vldb}

\usepackage{graphicx}
\usepackage{url}
\usepackage{hyperref}
\usepackage{algorithm} % for algorithms
\usepackage{algpseudocode}
% \usepackage{booktabs} % For formal tables
% `\usepackage{amsthm} % For claims
\usepackage{balance}  % for  \balance command ON LAST PAGE  (only there!)
\usepackage{caption}
\usepackage{subcaption}

% \theoremstyle{remark}

\pagestyle{empty} % removes running headers

\newcommand{\PicScale}{0.5}
\newcommand {\FlameStream} {FlameStream}
\newcommand {\tracker} {trAcker}
\newcommand {\acker} {Acker}

% Include information below and uncomment for camera ready
\vldbTitle{Tracking Dependencies in Distributed Streaming Dataflows}
\vldbAuthors{Nikita Sokolov, Artem Trofimov, Igor Kuralenok, Nikita Marshalkin, and Boris Novikov}
\vldbDOI{https://doi.org/10.14778/xxxxxxx.xxxxxxx}
\vldbVolume{12}
\vldbNumber{xxx}
\vldbYear{2019}

\begin{document}

\title {Keyed State Management in Distributed Streaming Dataflows}

\numberofauthors{5}

\author{
\alignauthor
Nikita Sokolov\\
    \affaddr{ITMO University}\\
    \affaddr{Saint Petersburg, Russia}\\
    \email{faucct@gmail.com}
\alignauthor
Artem Trofimov\\
    \affaddr{Yandex}\\
    \affaddr{Saint Petersburg, Russia}\\
    \email{tomato@yandex-team.ru}
\alignauthor
Igor Kuralenok\\
    \affaddr{Yandex}\\
    \affaddr{Saint Petersburg, Russia}\\
    \email{solar@yandex-team.ru}
\and 
\alignauthor
Nikita Marshalkin\\
    \affaddr{VK}\\
    \affaddr{Saint Petersburg, Russia}\\
    \email{n.marshalkin@corp.vk.com}
\alignauthor
Boris Novikov\\
    \affaddr{National Research University Higher School of Economics}\\
    \affaddr{Saint Petersburg, Russia}\\
    \email{borisnov@acm.org}
}

\maketitle

\begin{abstract}
%The majority of state-of-the-art stream processing systems faces a problem of obtaining notifications when the specific set of input elements have already been applied to all operators in a dataflow. Such notifications are commonly used to take consistent state snapshots or to release all descendants of an input element atomically while preserving exactly-once delivery guarantee. To generate these notifications, there is a need to track dependencies between input and their descendants and to inform when all descendants have been completely processed. In this work, we propose a method for tracking dependencies between streaming elements that can be applied for both cyclic (iterative) and acyclic dataflows. We demonstrate that our technique has low latency and throughput overhead within various setups and can significantly outperform methods employed by state-of-the-art stream processing systems.

%In distributed stream processing, it is hard to determine a correspondence between input and output elements due to complex transformations, filterings, producing multiple items from a single, and so on. On the other hand, mechanisms for taking consistent state snapshots and for transactional stream processing require notifications when a system has already processed a specific set of input elements together with all their descendants. To produce these notifications, one needs to track dependencies between items in a dataflow. In this work, we propose a method for monitoring dependencies between streaming elements applicable to cyclic dataflows as well. We enrich each data item with a logical timestamp that identifies corresponding input items. We build a scalable method to watch for the computational progress and to notify when dataflow contains no elements with specific timestamps. These notifications indicate when a system completely processed all descendants of concrete input elements. We demonstrate that our technique provides low notification latency while having almost no throughput overhead on regular processing, and can outperform methods employed by state-of-the-art stream processing systems.

% In distributed stream processing, a system often needs to determine a correspondence between input and output elements, e.g. to take state snapshot affected by  or to release all descendants of an input element atomically while preserving exactly-once delivery guarantee. This task is hard due to complex transformations, filterings, producing multiple elements from a single, etc. 

In distributed stream processing, it is often required to manage in-memory state, e.g. buffers for joining streams. This state is often partitioned by independent keys, for example by user session identifier. When these keys reach their end of life, the state related to them can no longer be reached and could be dumped to disk or even deleted. Unless freed, this state can grow limitless.

Another problem in distributed stream processing is uneven distribution of workload between machines as a result of most popular or hot keys being partitioned together.

In this work, we propose a method for monitoring an amount of in-flight elements per key, which can be used to solve the mentioned problems and is applicable to cyclic dataflows as well. We demonstrate that our method provides low notification latency and its memory footprint is less than ones of methods employed by state-of-the-art stream processing systems.

\end{abstract}

% \keywords{Data streams, exactly-once, drifting state, optimistic OOP}

\thispagestyle{empty}

\section {Conclusion}
\label {fs-acker-conclusion}

In this work, we formulated and formalized a problem of freeing independent in-memory state per surrogate and natural keys and a problem of detecting hot keys.

To solve the problems, we proposed a mechanism that adopts ideas from the Apache Storm completion tracking mechanism called \acker. We track number of dataflow elements with specific independent keys and notify when there are no more left or when the system load is unevenly distributed. Our solution, called \tracker, provides the following features:
\begin{itemize}
    \item {\bf Fine-grained tracking:} \tracker\ efficiently watches and provides notifications that the system has completely processed input items with a specific key.
    \item {\bf Cyclic graphs support:} proposed mechanism works for cyclic execution graphs, and that makes it suitable for iterative dataflows as well. 
    \item {\bf Low overhead:} \tracker\ does not produce any significant performance penalty and does not affect the throughput of a distributed streaming dataflow.
\end{itemize}

We have tested the proposed method in three stream processing tasks: text classification, User Session Log Service and Descendant Query.

We compared the proposed method of state freeing with two baseline approaches: Last Recently Used and Last Frequently Used caches, TTL and punctuations based on markers used in Apache Flink. We demonstrated that our method provide lower memory footprint.

We also tested the proposed method of hot keys tracking. We demonstrated that it is able to detect uneven load distribution and propose an even repartition.


\bibliographystyle{abbrvurl}
\bibliography{../../bibliography/flame-stream}

\end {document}

\endinput
